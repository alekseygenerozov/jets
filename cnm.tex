\documentclass[usenatbib,fleqn]{mnras}

\makeatletter
\newlength{\abovecaptionskip}%
\setlength{\abovecaptionskip}{10\p@}
\makeatother


\usepackage{threeparttable}
 

\usepackage{amsmath,amssymb}
\usepackage{cases}
\usepackage{mathrsfs}
\usepackage{graphicx}
\usepackage{epstopdf}
%\usepackage{hyperref}
\epstopdfsetup{outdir=./figures/}
\graphicspath{{./figures/}}
\usepackage{url}
%\usepackage{aas_macros}

\newcommand\lsim{\mathrel{\rlap{\lower4pt\hbox{\hskip1pt$\sim$}}
    \raise1pt\hbox{$<$}}}
\newcommand\gsim{\mathrel{\rlap{\lower4pt\hbox{\hskip1pt$\sim$}}
    \raise1pt\hbox{$>$}}}


\newcommand       \be          {\begin{eqnarray}}
\newcommand       \ee          {\end{eqnarray}}
\newcommand{\Mbh}[1][]{M_{\bullet#1}}
\newcommand{\Menc}{M_{\rm enc}}
\renewcommand{\th}{t_h}
\newcommand{\Msun}{{\rm M_\odot}}
\newcommand{\pyear}{{\rm yr}^{-1}}
\newcommand{\rs}{r_s}

% write title (with email and institute)
\title{The influence of the cicumnuclear environment on the radio emission from TDE jets}
\author[Generozov et al.]{ A. Generozov$^{1}$, P. Mimica$^{2}$,
  B. D. Metzger$^{1}$,
  D. Giannios$^{3}$, 
  N. Stone$^{1}$,
  M.A. Aloy$^{2}$ 
  \\
  $^{1}$Columbia Astrophysics Laboratory, Columbia University, 550 West 120th Street, New York, NY 10027\\
  $^{2}$Departamento de Astronomia y Astrofisica, Universidad de Valencia, E-46100 Burjassot, Spain\\
  $^{3}$Department of Physics and Astronomy, Purdue University, 525
  Northwestern Avenue, West Lafayette, IN 47907, USA}

\begin{document}
\maketitle
\begin{abstract}
  There are now dozens of candidates for tidal disruptions of stars by
  supermassive black holes (TDEs) at optical and x-ray wavelengths. A
  small fraction of these events, (e.g. {\it Swift} J1644+57) have radio
  synchrotron emission consistent with a powerful, relativistic jet
  shocking surrounding gas. The low detection rate of such events may
  mean that powerful jets are intrinsically rare in TDEs. However, it
  could also mean that typical nuclear gas densities are unfavorable
  for producing observable radio emission. We explore this
  possibility, constraining the range of gas densities which could be
  encountered by a TDE jet. We then calculate radio light curves for
  jets across the expected range of gas densities ($\sim$ 0.5-2000
  cm$^{-3}$ at $10^{18}$ cm). We find bright radio transients across
  this range of density profiles. Existing radio upper limits are
  often taken decades after the observed flare, well after the
  expected peak of the light curve for low density
  environments. Nonetheless they exclude powerful, {\it Swift} J1644+57-like
  jets for CNM densities less than 60 cm$^{-3}$ at $10^{18}$ cm. More
  stringent constraints would be possible with prompt follow-up of
  tidal disruption event candidates and would inform our understanding
  of the conditions necessary to launch jets.
\end{abstract}
\section{Introduction}
\label{sec:intro}
When a star in a galactic nucleus is deflected too close to the
central supermassive black hole (BH), it can be torn apart by tidal
forces.  During this tidal disruption event (TDE), roughly half of the
stellar debris remains bound to the BH, while the other half is flung
outwards and unbound from the system.  The bound material, following a
potentially complex process of debris circularization
(\citealt{Guillochon+2013},\citealt{Hayasaki+2013},\citealt{Hayasaki+2015},\citealt{Shiokawa+2015},\citealt{Bonnerot+2015}),
accretes onto the BH, creating a luminous flare lasting months to
years \citep{Hills1975, Carter+1982, Rees1988}.

Many TDE flares have now been identified at UV/optical
\citep{Gezari+2008, Gezari+2009, van-Velzen+2011, Gezari+2012,
  Arcavi+2014, Chornock+2014, Holoien+2014, Vinko+2015, Holoien+2016}
and soft x-ray wavelengths \citep{Bade+1996, Grupe+1999,
  Komossa&Greiner1999, Greiner+2000, Esquej+2007, Maksym+2010,
  Saxton+2012}. Beginning with the discovery of {\it Swift} J1644+57
(hereafter SwJ1644) in 2011, three additional TDEs have been
discovered by their hard x-ray emission (\citealt{Bloom+2011,
  Levan+2011, Burrows+2011, Zauderer+2011, Cenko+2012, Pasham+2015,
  Brown+2015}) {\bf Mostly same refs for optical/soft x-ray as
  Holoien+2016, with addition of Greiner+2000, and exclusion of
  Donley+2002 (who are looking at statistics of x-ray outbursts wo
  identifying a specific event)}.  Unlike the optical/UV and soft
X-ray TDEs, these events are the result of emission from a transient
relativistic jet beamed along our line of sight, similar to the blazar
geometry of active galactic nuclei (AGN).  In addition to their highly
variable X-ray emission, which likely originates from the base of the
jet, these events are characterized by radio synchrotron emission.
The latter, more slowly evolving, is powered by shocks formed at the
interface between the jet and surrounding circumnuclear medium (CNM)
\citep{Bloom+2011,Giannios&Metzger2011,Metzger+2012,Mimica+2015} {\bf
  other refs? did not see others in Mimica+2015}, analagous to the
afterglow of a gamma-ray burst {\bf AG change thermal to optical/UV
  and soft X-ray: X-ray emission may not be thermal (often times
  difficult to tell given spectral data.)}.

Although a handful of jetted TDE flares have been observed, their
volumetric rate is a very small fraction $\sim 10^{-5}-10^{-4}$ of the
rate of the observed TDE flare rate (e.g., \citealt{Burrows+2011},
\citealt{Brown+2015}), and an even smaller fraction of the
theoretically predicted TDE rate (\citealt{Stone&Metzger2016}).  One
potential explanation for this discrepancy is that the majority of
TDEs produce powerful jets, but their hard X-ray emission is
relativistically beamed into a small angle $\theta_{\rm b}$, making
them visible to only a small fraction of observers.  However, the
inferred beaming fraction $f_b \approx \theta_{b}^{2}/2 \sim
10^{-5}-10^{-4}$ would require $\theta_{\rm b} \sim 0.01$ and hence a
jet with a bulk Lorentz factor of $\Gamma \gtrsim 1/\theta_j \sim
100$, much higher than typically inferred for AGN jets or inferred for
SwJ1644 (\citealt{Metzger+2012}).

The low detection rate of hard X-ray TDEs may instead indicate that
jets are intrinsically rare in these events, or that the conditions of the
surrounding environment are unfavorable for producing bright emission.
Jets could be rare if they require, for instance, a highly
super-Eddington accretion rate (\citealt{De-Colle+2012}), a TDE from a
deeply plunging stellar orbit (\citealt{Metzger&Stone2015}), or a
particularly strong magnetic flux threading the star
(\citealt{Tchekhovskoy+2014,Kelley+2014}).  Alternatively, jet formation
or its X-ray emission could be suppressed if the disk undergoes
Lens-Thirring precession due to a misalignment between the angular
momentum of the BH and that of the disrupted star
(\citealt{Stone&Loeb2012}).  In the latter case, however, even a `dirty'
jet could still be produced, which emits luminous non-thermal radio
synchrotron emission as it interacts the surrounding CNM.  The radio
emission from an off-axis is predicted to be relatively isotropic
\citep{Giannios&Metzger2011,Mimica+2015}.

\citet{Bower+2013} and \citet{van-Velzen+2013} performed follow-up
radio observations of optical/UV and soft X-ray TDE flares timescales
of months to decades after the outburst (see Table 1 of
\citealt{Mimica+2015} for a compilation). They find no radio afterglows
definitively associated with the host galaxies of strong TDE
candidates.\footnote{There were radio detections for two ROSAT flares:
  RX J1420.4+5334 and IC 3599. However, for RX J1420.4+5334 the radio
  emission was observed in a different galaxy than was originally
  associated with the flare.  IC 3599 has shown multiple outbursts in
  the recent years, calling into question whether it is a true TDE at all
  \citep{Campana+2015}.}  \citet{Bower+2013} and
\citet{van-Velzen+2013} use a simple model for the radio emission as a
Sedov blast wave, to conclude that $\lesssim 10\%$ of TDEs produce
jetted TDE emission at a level similar to that produced in SwJ1644.  \citet{Mimica+2015} used 2D hydrodynamical models, coupled
with a synchrotron radiation transport calculation, to model the
on-axis emission from SwJ1644. They then extended this
model to off-axis viewing angles, and concluded that most previous
TDEs should have been detected if their jets were as powerful. 

Recently, a candidate TDE flare (ASSASN-14li) was observed to
have transient radio emission, consistent with either a weak
relativistic jet \citep{van-Velzen+2015} or a sub-relativistic outflow
\citep{Alexander+2015} with overall energy in the range of
$10^{48}-10^{49}$ ergs.  ASSASN-14li occurred in a very nearby galaxy,
and if other TDE candidates launched similar outflows, their radio
afterglows would be below existing upper limits. A priori it
is not obvious whether the lack of observed radio emission in most
thermal TDEs is due to (i) weak/no outflows or (ii) a nuclear gas
environment unfavorable for producing observable radio emission.

Previous works (\citealt{Bower+2013}; \citealt{van-Velzen+2013};
\citealt{Mimica+2015}) have generally assumed that all TDEs occur in a
similar environment as SwJ1644.  However, in general the
gas density in a galactic nucleus depends sensitively on the sources
of gas from stellar winds, and the sources of gas heating
(\citealt{Quataert2004,Generozov+2015}).  In a gamma-ray burst, the
environment the jet emerges from is relatively well-understood to be
the wind of the massive progenitor star, or the ISM of the host
galaxy.  However, the density of the CNM encountered by a TDE jet
could in principle be orders of magnitude higher.

In this paper we address the range of gas densities encountered by
jetted TDEs and estimate their synchrotron emission by means of
analytic calculations and numerical simulations.  We find that radio
emission should be detected for all plausible CNM densities, leading to the robust conclusion that most TDEs do not launch SwJ1644-like
jets.  This result has significant implications for the physics
of jet launching in TDEs and other accretion flows. 

In $\S\ref{sec:cnm}$ we use the formalism developed in
\citet{Generozov+2015} (hereafter GSM15) to calculate the CNM profiles
encountered by TDE jets for different assumptions about the stellar
population in the galactic nucleus.  A younger stellar population
produces significant wind mass loss from O and B stars. In contrast,
the rate of wind mass loss from an older population, which lacks these
massive stars, could be a $\sim$couple orders of magnitude smaller.
We show that the requirement of a physical stellar population limits
the gas density on a scale of $10^{18}$ cm to a range of $n_{18} \sim
0.5-2000$ cm$^{-3}$. We also show that the measured distribution of
accretion rates of low mass SMBHs in the local universe suggests that
the gas density in most galactic nuclei lie within a similar range.

As a second component of this paper, in $\S\ref{sec:jet}$ we explore
the expected synchrotron emission from SwJ1644-like
across the allowed range of CNM conditions.  We start in
$\S\ref{sec:analytic}$ with analytic conditions. We also present
analytic scaling relations for the time of peak and peak flux of the
radio light curve in Appendix~\ref{app:analyt}.  Then in
$\S\ref{sec:numerical}$ use both 1D and 2D hydrodynamic models to
simulate the jet propagating through the CNM and compare the results
in $\S\ref{sec:2D}$. In $\S\ref{sec:results}$ we show radio light
curves from our 1D models for a wide range of CNM densities, to
illustrate qualitatively how much varying the CNM density by itself
could change the radio light curve.  We summarize our conclusions in
$\S\ref{sec:conc}$.

\section{Range of CNM Densities}
\label{sec:cnm}

In this section we place constraints on the radial profile of the gas
density in galactic nuclei. In $\S\ref{sec:analy}$, we determine the
possible range of densities resulting from mass loss by stellar
winds. Then in $\S\ref{sec:empirical}$ we convert empirical
distributions for black hole luminosities into distributions of
nuclear gas densities.

\subsection{Analytic Constraints on CNM Density}
\label{sec:analy}

\subsubsection{Dynamical Model}
\label{sec:model}

The dominant source of gas in the CNM of quiescent galaxies is winds
from stars in the galactic nucleus. We model the hot phase of the ISM
using the 1D spherical hydrodynamic equation with mass and energy
injection from stellar winds (e.g. \citealt{Holzer+1970};
\citealt{Quataert2004})
\begin{align}
  &\frac{\partial \rho}{\partial t}+\frac{1}{r^2}\frac{\partial}{\partial r}\left(\rho r^2 v\right)=q \label{eq:drhodt}\\
  &\rho \left(\frac{\partial v}{\partial t} + v\frac{\partial
      v}{\partial r}\right) =-\frac{\partial p}{\partial r}- \rho\frac{GM_{\rm enc}}{r^{2}} -q v \label{eq:dvdt}\\
  &\rho T\left(\frac{\partial s}{\partial t} + v\frac{\partial
      s}{\partial
      r}\right)=q\left[\frac{v^2}{2}+\frac{v_w^2}{2}-\frac{\gamma}{\gamma-1}
    \frac{p}{\rho} \right] ,
\label{eq:model}
\end{align}
where $\rho = \mu m_p n$, $v$, $p$, and $s$ are the density, velocity,
pressure (assumed to be an ideal gas with $\mu = 0.62$), and specific
entropy, respectively.  The enclosed mass $\Menc = M_{\bullet} +
M_{\star}$ includes both the black hole mass $M_{\bullet}$ and
enclosed stellar mass $M_{\star} \propto \int \rho_{\star}r^{2}dr$, where $\rho_{\star}$ is the stellar density. At
the radius of the sphere of influence, $r_{\rm inf}$, the enclosed stellar and black masses are equal, $M_{\star}(r_{\rm inf})=\Mbh$.  We take $r_{\rm inf}=3.5 \Mbh[,7]^{0.6}$ (GSM15), where $\Mbh[,7]=\Mbh/10^7
\Msun$.

The mass source term $q =\eta \rho_\star/\th$ is the mass injection
rate per unit volume per unit time. The energy source term $\propto v_w^{2}$, parameterizes the
heating rate of the gas, as physically results from stellar wind
kinetic energy, supernovae, and black hole feedback.

GSM15 present analytic approximations for the
densities and temperatures of steady state solutions to
equation~\eqref{eq:model}. We apply these results across the
physically allowed range heating rates ($v_w$) and mass injection
rates ($\eta$), and obtain the corresponding range of gas densities.

\subsubsection{Stellar density profiles}
We assume a broken power law for the stellar density profile,
$\rho_{\star}$, motivated by Hubble measurements of the radial surface
brightness profiles for hundreds of nearby early type galaxies
\citep{Lauer+2007}.  The measured profile is well fit by the so-called
``Nuker'' law parameterization: i.e., a broken power law that smoothly
transitions from an inner power law slope, $\Gamma$, to an outer power
law slope, $\beta$, at a break radius, $r_b$.

Most galaxies have $0<\Gamma<1$, and are classified into two broad
categories: ``core" galaxies with $\Gamma<0.3$ and ``cusp" galaxies with
$\Gamma>0.5$. Assuming spherical symmetry and a constant mass-to-light
ratio, the inner stellar profile translates to a stellar density of
$\rho_\star\propto r^{-1-\Gamma}=r^{-\delta}$. Core galaxies have
$1<\delta<1.3$, while cusp galaxies have $1.5<\delta<2$.

Cusp-like stellar density profiles, are the most relevant to TDEs for
two reasons.  First, only a low mass black hole ($\Mbh\lsim 10^8
\Msun$) can disrupt a main sequence star, and low mass low mass black
holes are more commonly characterized by cusp-like profiles.
Additionally, as described in \citet{Stone&Metzger2016} a cuspy
stellar density profile results in a higher TDE rate per galaxy.  We
adopt a fiducial value of $\Gamma=0.7$ ($\delta=1.7$), motivated by
the rate-weighted average value of the inner stellar density profile for
the galaxies in \citet{Stone&Metzger2016} (their Table C).

L% ow mass black holes ($\Mbh\lsim 10^{8} \Msun$), which are more commonly characterized by cusp-like profiles, are the most relevant to TDEs for two reasons.  First, above this black hole mass the star is swallowed whole instead of undergoing a TDE.  Additionally, as described in \citet{Stone&Metzger2016} 
% a cuspy stellar density profile results in a higher TDE rate per galaxy.  We adopt a fiducial value of $\Gamma=0.7$ ($\delta=1.7$), motivated by the rate-weighted average value of the stellar density profile for the galaxies in \citet{Stone&Metzger2016} (their Table C).

 {\bf Recently changed from $\delta=1.8$, need to
  double check all expressions.}

\subsubsection{Gas density profiles}
The radio emission from TDE jets is sensitive to the density of gas
near the deceleration (or ``Sedov") radius, outside of which the jet
has swept up a gaseous mass exceeding its own.  For SwJ1644 the
deceleration radius of the jet was $\sim 10^{18}$ cm
\citep{Mimica+2015}; we therefore focus on determining the gas density
on parsec scales.

Given sufficiently strong heating, a one-dimensional steady-state
model for the CNM is characterized by an inflow-outflow structure.
The velocity passes through zero at the ``stagnation radius'', $\rs$.
Mass loss from stars interior to the stagnation radius is accreted,
while that outside of $\rs$ is unbound in an outflow from the nucleus.
Fig.~\ref{fig:profiles} shows example radial profiles of the
steady-state gas density calculated for a core and a cusp stellar
density profile. The stagnation radius is marked as a blue dot on each
profile.

As described in GSM15, the stagnation radius is approximately given by
\begin{equation}
r_s \simeq f(\delta) \frac{G M}{v_w^2},
\label{eq:rs}
\end{equation}
where $f(\delta)$ is a constant of order unity that depends on the slope of the stellar density profile $\delta$.  The gas density at the stagnation radius, $n(\rs)$, is determined by the rate at which stellar winds inject mass  interior to this,
\begin{equation}
\dot{M}=\frac{\eta M_{\rm \star}(\rs)}{t_h} \approx  3.0 \times 10^{-6} \Mbh[,7]^{0.22} \eta_{0.02} \left(\frac{r_s}{\rm
  pc}\right)^{1.3} \Msun \, {\rm yr}^{-1},
\label{eq:dotM}
\end{equation}
where $M_{\star}(\rs)$ is the total stellar mass enclosed within the stagnation radius and the second equality was derived for our fiducial value of $\delta=1.7$, where $\eta_{0.02}=\eta/0.02$ is normalized to a fiducial value characteristic of an old stellar population.

The density at the stagnation radius, $n(\rs)$, is estimated by equating the gas injected by stellar winds over a dynamical time at the stagnation radius, $t_{\rm dyn} (\rs)=(\rs^3/G \Mbh)^{1/2}$, to the gas mass enclosed at this location.  This gives
\begin{align}
  &\frac{4 \pi}{3} \rs^3 \mu m_p n(r_s) \simeq \dot{M} t_{\rm dyn}
  (\rs) \nonumber\\
  &\Rightarrow n(r_s) \simeq 0.21 \eta_{0.02} \Mbh[,7]^{-0.28} \left(\frac{r_s}{\rm
      pc}\right)^{-0.2} {\rm cm}^{-3}.
\label{eq:nrs}
\end{align}
Substituting equation~\eqref{eq:rs} for $r_s$, we obtain 
\begin{equation}
n(r_s) \simeq 0.2 \, v_{500}^{0.4} \eta_{0.02} \Mbh[,7]^{-0.48} {\rm cm}^{-3},
\label{eq:nrs2}
\end{equation}
where $v_{500}=v_w/\left(500 \,\mathrm{km\,\,
    s^{-1}}\right)$. 

Near the stagnation radius, GSM15 found that the radial gas profile
has a power-law slope of $k \approx (4\delta-1)/6$, which for our
fiducial value of $\delta$=1.7 gives $n \propto r^{-1}$.  A similar
density profile obtains for most radii of interest.\footnote{For cusp
  galaxies, the gas velocity asymptotes to a constant value between
  the stagnation radius and break radius (provided they are
  sufficiently far apart). In this region we instead expect $n\propto
  r^{1-\delta}$.}  Thus, we adopt $n(r)= n_{18} (r/10^{18} {\rm
  cm})^{-1}$ as our fiducial density profile, where $n_{18}$ is the
density at $r = 10^{18}$ cm.  In Appendix~\ref{app:core} we explore a
core-like density profile, to which we compare our results for the
jetted radio emission to the fiducial cusp case in
$\S\ref{sec:profileComp}$.

Combining our assumption of a $n\propto r^{-1}$ profile with
equation~\eqref{eq:nrs}, we obtain
\begin{equation}
  n_{18}\simeq 0.6 \left(\frac{r_s}{\rm pc}\right)^{0.8}
  \Mbh[,7]^{-0.28} {\rm cm^{-3}},
  \label{eq:n18}
\end{equation}
Substituting the stagnation radius (eq.~\ref{eq:rs}) into this expression gives
\begin{equation}
n_{18}\simeq 0.3 \Mbh[,7]^{0.52} \eta_{0.02} v_{500}^{-1.6} {\rm
  cm^{-3}}.
\label{eq:n182}
\end{equation} 

Although the gas density steepens near the break radius $r_b$ of the
stellar density profile (Fig.~\ref{fig:profiles}), this will only
affect the radio emission near its peak if $r_b$ lies inside of the
Sedov radius of the jet.  Figure \ref{fig:profiles} shows the Sedov
radius for different energy jets and the gas density profiles there.
The measured break radii of all but four the galaxies in
\citet{Lauer+2007} exceeds 10 parsecs.  This is well outside of the
Sedov radius, even for a very energetic jet, with an isotropic
equivalent energy of $E=4\times 10^{54}$ erg in a CNM of particularly
low density, $n_{18} \sim 1$ cm$^{-3}$.  The presence of nuclear star
cluster in the galactic center could produce another break in the
stellar density profile near the outer edge of the NSC, which is
typically located at $\sim 1-5$ pc \citep{Georgiev+2014}.  However,
even a break radius of 1 pc will reside inside the Sedov radius,
except for the combination of a powerful jet and relatively small CNM
density, $n_{18}<20$ cm$^{-3}$.  Henceforth we neglect the effects of
the stellar break radius in our analysis.


\begin{figure}
\includegraphics[width=8cm]{sedov_radius.pdf}
\caption{\label{fig:profiles} Steady-state radial profiles of the gas density of the CNM, $n(r)/n_{18}$, normalized to the value $n_{18} = n(r = 10^{18}$ cm), calculated for a black hole mass $10^{7} \,\Msun$ and gas heating parameter $v_w=600$ km s$^{-1}$.  Cusp and core stellar density profiles are shown with solid and dashed lines, respectively.  The line color denotes the isotropic equivalent energy of a jet for which the Sedov radius equals that radius from the black hole (given the swept up interior gaseous mass), assuming an initial jet Lorentz factor of $\Gamma=2$.}
\end{figure}



\subsubsection{Plausible Density Range}
\label{sec:densAllowed}

In this section, we estimate the plausible range in the normalization
of the CNM gas profile, $n_{18}$.  We assume that star formation
occurs in two bursts, an old burst of age comparable to the Hubble
time $t_{\rm h} = 10^{10}$ yr, and a ``young'' burst of variable age
$t_{\rm burst} \ll t_{\rm h}$ which contributes a fraction $f_{\rm
  burst}$.
%Both the young and old stars have the
%same (cuspy) density profile, which implies that the gas density
%profile goes as $r^{-1}$.

For a sufficiently large burst occurring less than 40 Myr ago, gas
heating is dominated by the energetic winds of massive star
winds.\footnote{Type II supernovae are also an important heating
  source, which dominates stellar winds after $\sim$6 Myr
  \citep{Voss+2009}.  However, SN II are intermittent and hence their
  contribution to the gas heating is neglected for simplicity.}  In
this case, the mass return ($\eta$) and heating parameters ($v_w$) are
calculated as described in GSM15 (their Appendix C).  Given
$\eta(t_{\rm burst},f_{\rm burst})$ and $v_w(t_{\rm burst},f_{\rm
  burst})$, we calculate $n_{18}$ following equation~\eqref{eq:n182}.

For older stellar populations, gas heating may come from a few
different sources, including Type Ia Supernovae (SNe)\footnote{Unbound
  debris streams from TDEs provide another source of heating localized
  in the galactic center (see \citep{Guillochon+2015a}), which we
  neglect since we find that its contribution to the heating is at
  most comparable to that from SNe Ia.} and AGN feedback.  We focus on
quiescent phases, during which SNe Ia dominate.  As discussed in
GSM15, SNe Ia clear out the gas external to a critical radius, $r_{\rm
  Ia}$, where the interval between successive Ia SNe equals the
dynamical (gas inflow) timescale.  For an old stellar population,
$n_{18}$ is estimated by equating $r_{\rm Ia}$ with the stagnation
radius in equation~\eqref{eq:n18}.  The Ia radius is calculated as
described in GSM15 at times $t>300 \,{\rm Myr}$ after star formation,
and is taken to be constant for $t = 40-300$ Myr.\footnote{GSM15
  incorrectly extrapolated the Ia rate valid at times $t>300 \,{\rm
    Myr}$ back to a time $t = 3$ Myr, which unphysical as no white
  dwarfs would have formed by this time.  Although its qualitative
  impact on our results is minimal, here we instead take the Ia rate
  to be 0 for $ t < 40$ Myr.}

%\footnote{The Ia rate is given
 % by $8.8 \times 10^{-13} \left(\frac{t}{3\times 10^{8} {\rm
  %      yr}}\right)^{-1.12} \, \Msun^{-1} \, {\rm yr}^{-1} $ for
  %$t>3\times 10^8$ years.} =
In GSM15 we incorrectly extrapolated this rate all of the way to 3
Myr--this is clearly un-physical as no white dwarfs would have formed
by this time. Here, we instead take the Ia rate to be 0 for times less
than 40 Myr and flat from 40-300 Myr.

Fig.~\ref{fig:param} shows how $n_{18}$ varies with the young
starburst properties, $f_{\rm burst}$ and $t_{\rm burst}$.  We find a
maximum density of $n_{18} \sim 3000$ cm$^{-3}$ is achieved for a
burst of age $t_{\rm burst} \sim 4$ Myr which forms most of the stars
in the nucleus $f_{\rm burst} \sim 1$.  In this case, both the energy
and mass budgets of the CNM are dominated by fast winds from massive
stars.  Although a large gas density is present in the immediate
aftermath of a starburst, its magnitude will decline with the wind
mass loss rate, approximately $\propto t^{-3}$.  Therefore, the gas
density would decline by an order of magnitude from its maximum
possible value after just a few Myr.

By contrast, the lowest density $\sim 0.03$ cm$^{-3}$ is achieved for
a relatively modest burst of young stars $t_{\rm burst} \approx
10^{6}$ Myr, which forms a fraction $f_{\rm burst} = 4\times 10^{-4}$
of the total stellar mass. In this case young, massive stars provide a
high heating rate, while the mass injection rate is relatively low,
with both young and old stars contributing.

Our procedure likely underestimates the value of $n_{18}$ somewhat, as
we do not include the effects of discreteness on the assumed stellar
population.  In particular, we assume that massive stars provide a
spatially homogeneous heating source, even on small scales where the
number of massive stars present may be very small.  The doubly hatched
region in Fig.~\ref{fig:param} denotes the region where less than one
massive star ($\gsim 15 \Msun$) is on average present inside of the nominal stagnation
radius (eq.~\ref{eq:rs}).  Discreteness effects are thus important for
relatively small bursts of star formation, including the case
described above which gives the minimum $n_{18}$.  If we instead
equate the stagnation radius to the radius enclosing a single star of
mass $\gsim 15 \Msun$, we find a larger value of $n_{18}\sim 0.5$
cm$^{-3}$.  The true minimum density likely resides between these
extremes, in which case min($n_{18}) \sim 0.03-0.5$ cm$^{-3}$.

The majority of the host galaxies of observed TDE flares shows
evidence for a stellar population of age $t_{\rm burst} \lesssim$ 1
Gyr \citep{French+2016}, which corresponds to the right hand side of
Fig.~\ref{fig:param}. In this region of parameter space, gas heating
rate is dominated by SN Ia and we typically expect $n_{18}\sim 100$
cm$^{-3}$. {\bf AG note that the entire stellar pop likely did not
  form 1 Gyr ago. In fact the observation merely indicate that there
  was some star formation w. in the last Gyr}

To summarize, we expect the CNM densities of quiescent galaxies on
parsec scales to vary from min($n_{18}) \sim 0.03$ to max($n_{18})\sim
10^{3}$ cm$^{-3}$, with a typical value of $n_{18}\sim 100$ cm$^{-3}$
in TDE host galaxies.

\subsubsection{Mass drop-out from star formation?}

Our model for the CNM makes predictions for the total gas density
sourced by stellar winds, including both hot and cold phases.  In the
first few Myr after a starburst, the injected stellar wind material
will be quite hot, with temperature $\gsim 10^{7}$ K due to the
thermalized wind kinetic energy.  Later, SNe Ia provide intermittent
heating, but the stellar wind material that accumulates on small
scales, between successive SNe Ia may become much cooler, with
temperature $\lesssim 10^{4}$ K, and would condense into clumps.

The radio emission from a TDE jet is expcted to be similar for a
clumpy and a smooth medium with the same average density
(e.g.~\citealt{Nakar&Granot2007}; \citealt{Mimica&Giannios2009};
\citealt{vanEerten+2009}; $\S$~\ref{sec:empirical}).  However, cold gas
may also condense into stars.  To estimate the star formation rate, we
assume that the CNM self-regulates itself to a condition of marginal
thermal stability.  This we define as the cooling time being ten times
longer than the dynamical timescale (e.g.~\citealt{McCourt+09}). For a
$\sim$1 Gyr old stellar population, we find that this requires a star
formation rate of $5.3 \times 10^{-4} \,\Msun$ yr$^{-1}$, which
implies that only $\sim 40$\% of the gas injected on small scales
would be turned into stars.  The gas density would therfore be only a
few times lower than in our estimates which neglect mass drop out from
star formation.

\begin{figure} 
  \includegraphics[width=8cm]{cnm_plot.pdf}
  \caption{\label{fig:param} Isocontours for the density at $10^{18}$
    cm (blue lines) for a $10^{7} \, \Msun$ black hole and bursts of
    star formation forming a fraction, $f_{\rm burst}$, of the stars
    $t_{\rm burst}$ years ago. The remaining stars formed a Hubble
    time ago. The old and young stars both have a cuspy density
    profile, which means that the gas density goes as $r^{-1}$.
    Hatched areas indicate regions of parameter space, where massive
    stars ($\gsim 15 \, \Msun$) dominate the gas heating rate, but we
    would expect less than one (doubly hatched) or less than ten
    (singly hatched) massive stars inside the nominal stagnation
    radius (equation~\ref{eq:rs}). In these regions of parameter space
    discreteness effects not captured by our formalism are important.}
\end{figure}


% \begin{figure}
%   \includegraphics[width=8cm]{cnm_plot_2.pdf}
%   \caption{\label{fig:param2} $n_{18}$ as a function of the mass
%     return ($\eta$) and heating ($v_w$) parameters for a $10^7 \Msun$
%     black hole. We translate the gray line from Fig.~\ref{fig:param}
%     to this parameter space. $f_{\rm burst}$ increased in the direction
%     indicated by the gray arrow. As indicated by the black solid line,
%     the area below the black solid line is thermally unstable
%     (\textit{see text for discussion}).}
% \end{figure}

\subsection{Empirical Constraints on CNM Density}
\label{sec:empirical}

In this section we translate observed constraints on the accretion rate distribution of supermassive black holes into constraints on the CNM density.  The accretion rate can be written as
\begin{equation}
\dot{M}_{\bullet} = f_{\rm in} 4 \pi r^2 \mu m_p n v,
\label{eq:mdot}
\end{equation}
where $n$ is the average density at radius $r$ and $f_{\rm in}$
is the fraction of the large scale inflow which actually reaches small scales and accretes onto the black hole. 
Inside the sonic point the velocity approaches the free-fall velocity, in which case  equation~\eqref{eq:mdot} becomes
\begin{equation}
  \dot{M}_{\bullet} = 1.1\times 10^{-5} \Mbh[,7]^{0.5} f_{\rm in}
  n_{18} \,\,\Msun {\rm yr}^{-1}.
\label{eq:mdot2}
\end{equation}
%
The corresponding Eddington ratio is

\begin{equation}
  \lambda\equiv L/L_{\rm Edd} = 4.8 \times 10^{-5}
  \left(\frac{\epsilon_{\rm rad}}{0.1}\right) \Mbh[,7]^{0.5} f_{\rm in}
  n_{18},
\label{eq:n18Edd}
\end{equation}
%
where $\epsilon_{\rm rad}$ is the radiative efficiency. \citet{Kauffmann&Heckman2009} present distributions of the OIII line luminosity, ${\rm L[OIII]}/\Mbh$ for a volume limited {\bf
  AG double check} sample of SDSS galaxies, where ${\rm L[OIII]}/\Mbh=1.7$ roughly
corresponds to Eddington ratio of unity (REF).  This bolometric correction maps the distribution of ${\rm L[OIII]}/\Mbh$ to a distribution of Eddington ratios \citep{Kauffmann&Heckman2009}.

Equation (\ref{eq:mdot2}) provides a map between the Eddington ratio and gas density $n_{18}$, provided that the radiative efficiency $\epsilon_{\rm rad}$ and accretion efficiency $f_{\rm in}$  are known.  For the former we adopt the MHD shearing box results of \citet{Sharma+2007}, who find (their Fig.~6)
\begin{align}
&\epsilon_{\rm rad} \simeq 
\begin{cases}
  0.03 \left(\frac{\dot{M}_{\bullet}}{10^{-4}\dot{M}_{\rm Edd}}\right)^{0.9} & \frac{\dot{M}_{\bullet}}{\dot{M}_{\rm edd}} \lsim 10^{-4} \\
 0.03 &  10^{-2} \gsim \frac{\dot{M}_{\bullet}}{\dot{M}_{\rm edd}}
 \gsim  10^{-4},
\end{cases}
\label{eq:efficiency}
\end{align}
where

\begin{equation} 
 \dot{M}_{\rm Edd} \equiv \frac{L_{\rm Edd}}{0.1 c^2} 
\end{equation}
%
and
there is a factor of $\sim$ 5 uncertainty in $\epsilon_{\rm rad}$ for
$\dot{M}/\dot{M}_{\rm Edd}\lsim 10^{-4}$ due to the dependence of the
results on uncertain microphysical parameters.  We linearly
interpolate between equation~\eqref{eq:efficiency} for
$\dot{M}/\dot{M}_{\rm Edd}<0.01$ and the standard thin-disk efficiency
of $\epsilon_{\rm rad}=0.1$ for $\dot{M}/\dot{M}_{\rm Edd}>0.1$.

\begin{figure}
\includegraphics[width=8cm]{fcum_n18.pdf}
\caption{\label{fig:n18Cum} Cumulative distribution of $f_{\rm in}
  n_{18}$ for black holes with mass $\Mbh\simeq 10^{7} \Msun$ based on
  the observed cumulative distributions of Eddington ratios from
  \citet{Kauffmann&Heckman2009} {\it (solid black line)}. Recall that
  $f_{\rm in}$ is the ratio of the true accretion rate onto the black
  hole to the inflow rate of free-falling gas at $10^{18}$ cm.  The
  dashed blue line is the corresponding cumulative distribution of
  $n_{18}$, for $f_{\rm in}$ from the simulations of \citet{Li+2013}
  ($\dot{M}_{\bullet}/\dot{M}_{\rm Bondi}$ in their Figure 6). The
  distribution of $f_{\rm in} n_{18}$ ($n_{18}$) is unreliable to the
  left of the solid black (dashed blue) vertical line. Such low
  densities correspond to systems with very small OIII luminosities,
  which could be coming from star formation rather than the black hole
  accretion flow.}
\end{figure}


Fig.~\ref{fig:n18Cum} shows the distribution of $f_{\rm in} n_{18}$ which results from combining the distributions of Eddington ratio from \citet{Kauffmann&Heckman2009} with equations~\eqref{eq:n18Edd} and
~\eqref{eq:efficiency}.  At first it would appear that the absence of galaxies with $f_{\rm in }n_{18} \lsim$ few cm$^{-3}$ places a lower bound on $n_{18}$ of a few cm$^{-3}$ because $f_{\rm in} \le 1$.  However, because measurements of Eddington ratios below $\sim 10^{-3}$ (shown with a vertical line in Fig.~\ref{fig:n18Cum}) are not reliable (Heckman, private communication), this allows a significant fraction ($\sim 30\%$) of galaxies to have lower gas densities.

To obtain the cumulative distribution of $n_{18}$ requires an additional prescription for $f_{\rm in}$.  We use the results of \citet{Li+2013}, who perform two-dimensional hydrodynamical simulations of axisymmetric
rotating accretion flows.  They find that when the inflow rate on large scales is highly sub-Eddington ($\dot{M}/\dot{M}_{\rm Edd} \lsim 10^{-4}$), cooling is inefficient and $f_{\rm in}\sim 0.01$. On the
other hand, when $\dot{M}/\dot{M}_{\rm Edd}\gsim 10^{-2}$, $f_{\rm in}$ approaches unity.  We use $\dot{M}_{\bullet}/\dot{M}_{\rm Bondi}$ in their Figure 6 for $f_{\rm in}$.\footnote{\citet{Li+2013} employ an alpha viscosity prescription with $\alpha=0.01$.  A higher value of $\alpha$ would likely increase the accretion fraction $f_{\rm in}$, thus decreasing our estimates of $n_{18}$ in a systematic way.}  With this choice we find that only a third of nuclei have $n_{18}>2\times 10^{3}$ cm$^{-3}$ and only 6\% have $n_{18}>10^{4}$ cm$^{-3}$.  We note, however, that existing candidate tidal disruption flares show little or no evidence for AGN emission lines (e.g. \citealt{van-Velzen+2011, Arcavi+2014}), making them unlikely to reside in the high density tail of the $n_{18}$ distribution. 

%Two potential complications to keep in mind are (i) clumpiness of the
%CNM and (ii) anisotropy. The distributions above are distributions of
%the {\it average} $n_{18}$.  Most likely, some of the nuclear gas in a
%low density hot phase, while the rest is in high density cold
%clumps/filaments.  However, while the jet is relativistic the
%light curve of a jet propagating through a clumpy medium will differ
%little from that of a smooth medium with the same average
%density. Even in the late stages when the jet becomes non-relativistic
%clumps will only make a difference if the size of the clumps is
%comparable to the size of the jet {\bf AG include some refs here}.

One potential complication is that the CNM could be distributed anisotropically around the black hole.  The cold phase could be concentrated in a ring-like structure, similar to the cirumnuclear ring surrounding SgrA$^{*}$ (e.g.~\citealt{Lau+13}).  Anisotropies may also present in the hot phase, for instance due to low density bubbles inflated by the AGN.  However, insofar as the angle of the TDE jet is large compared to these features, the quantitative impact on the radio flux is unlikely to be large.  

%For instance, X-ray observations of the nuclei of massive elliptical galaxies show densities of density on scales of $\sim
%100$ pc is $\gsim 0.1$ cm$^{-3}$

%For instance, \citet{Russell+2013} use X-ray observations to measure
%gas density and temperature profiles for a sample of massive elliptical galaxies. The measured electron density on scales of $\sim
%100$ pc is $\gsim 0.1$ cm$^{-3}$. Note that the gas density at 100 pc
%would be irrelevant for a TDE jet, but we would not expect the gas
%density to be decreasing towards the galactic center in a steady
%state.  We note that massive black holes in the \citet{Russell+2013}
%sample, with $\Mbh\sim 10^{9} \Msun$, whereas black holes tidal
%disruption events would have $\Mbh\lsim 10^8 \Msun$. This is because
%more massive black holes would not be able to disrupt (main sequence)
%stars.

%In this scenario, the
%some fraction of jets would likely be stifled by the very dense
%ring. However, such a ring would not block all jet propagation
%directions.

We conclude by noting that the empirical estimates in this section nicely complement our analytic results.  The Eddington ratios of black holes are particularly are particularly challenging to measure for quiescent nuclei.  However, cases in which AGN feedback is minimal are precisely the regime where our analytic estimates are the most trustworthy, providing a lower density floor of $n_{18} \sim 0.5$ cm$^{-3}$.

% On the other hand, the analytic estimates are specific to
% the hot phase of gas. The Eddington ratio distribution probes the
% average density (including any cold clumps). The
% distribution of Eddington ratios gives us confidence in our high
% density limit ($\sim 1000$ cm$^{-3}$).



\section{Analytic considerations}
\label{sec:jet}

\subsection{Dynamics}
\label{sec:analytic}

The mass of the CNM swept up by a jet is equal to its own rest mass at
``Sedov'' radius. For a power law gas density profile, $n= n_0 \left(r/r_0\right)^{-k}$,

\begin{equation}
  r_{\rm dec} = r_0 \left( \frac{E(3-k)}{4\pi n_{0}
      \Gamma m_{\rm p} c^2 r_0^3} \right)^{1/(3-k)}, 
  \label{eq:rdec}
\end{equation}
%
where $E$ and $\Gamma$ are the isotropic equivalent energy and Lorentz
factor of the jet. In the lab frame, The jet reaches this radius
at the deceleration time, given by

\begin{align}
 t_{\rm dec}= &c/r_{\rm dec} \nonumber\\
 \approx & \left(\frac{E}{10^{52} {\rm erg}}\right)^{1/(3-k)}
 n_0^{-1/(3-k)} \Gamma^{-1/(3-k)}{\rm year}
 \label{eq:tdec}
\end{align}
%
For our fiducial CNM density profile $n=n_{18}\left(r/10^{18}\right)^{-1} {\rm cm}$


\begin{equation}
r_{\rm sedov} = E_{54}^{1/2} \Gamma_{10}^{-1/2} n_{\rm 18}^{-1/2}\,{\rm pc}. 
\end{equation}

\noindent where $E_{54}=E/(10^{54} {\rm erg})$and $\Gamma_{10}=\Gamma/10$.

From the continuity equation, the (co-moving) density of the jet is
given by (e.g. Uhm \& Beloborodov 2007)
 \begin{align}
   n_{\rm j} =  \frac{L_{\rm j, iso}}{4 \pi r^{2}\Gamma_{\rm
       j}^{2}c^{3}m_p(1 + r \dot{\Gamma_{\rm j}}/c\Gamma_{\rm j}^{3})}
   \approx  \frac{L_{\rm j, iso}}{4 \pi r^{2}\Gamma_{\rm j}^{2}c^{3}m_p},
\end{align}
%
where $L_{\rm j, iso}$ is the isotropic equivalent luminosity of the
jet. The second term in the denominator is generally negligible as
long as the Lorentz factor of the jet changes slowly
($\dot{\Gamma}_{\rm j} \ll c\Gamma_{\rm j}^{3}/r$), as can be shown to
be true at radii $r < r_{\rm sedov}$ as long as $\Gamma_{\rm j}$
changes on timescales $\gtrsim t_{\rm j}$ ({\bf AG: What is $t_j$?
  Time for which jet? Replace $t_0$ later on w/ $t_j$?}).

A critical parameter is the ratio of the density of the jet to the
density of the CNM,

\begin{equation}
  f\approx 40\,  L_{\rm j,48} n_{18}^{-1} \Gamma_{10}^{-2} \, \left(\frac{r}{10^{18} {\rm
        cm}}\right)^{-1} 
\end{equation}
%
For a given $f$, the Lorentz factor of the shocked material may be
calculated using the relativistic shock jump condition and pressure
equality between the forward and reverse shocks. In the
ultra-relativistic limit 

\begin{equation}
\Gamma_{\rm sh} \underset{\Gamma_{\rm sh} \gg 1}= \Gamma_{\rm j}\left[1 + 2\Gamma_{\rm j}f^{-1/2}\right]^{-1/2}
\end{equation}
%
However, this expression is problematic when the outflow is mildly
relativistic or non-relativistic. In particular, it gives Lorentz
factors less than 1. A more general expression for $\Gamma_{\rm sh}$
can be obtained using equation 3 from
\citet{Beloborodov&Uhm2006}\footnote{As pointed out by
  \citet{Beloborodov&Uhm2006}, the assumption pressure equality
  between the forward and reverse shocks used to derive this
  expression could be inaccurate.}:

\begin{align}
&\frac{\Gamma_{\rm j}^2-1}{\Gamma_{43}^2-1} f^{-1}=1\\
&\Gamma_{43}=\Gamma_{\rm j} \Gamma_{\rm sh} \left(1-\beta_{\rm sh} \beta_j\right),
\label{eq:gammaShGen}
\end{align}
%
where $\Gamma_{43}$ is the Lorentz of shocked jet material in the
frame of the un-shocked jet. In the lab frame the velocity of the
reverse shock is

\begin{equation}
\beta_{\rm rs}=\frac{\beta_{\rm sh}(f)-\beta_{43}(f)/3}{1-\beta_{\rm
    sh}(f) \beta_{43}(f)/3}.
\label{eq:betars}
\end{equation} 
%
From equations~\eqref{eq:gammaShGen} and ~\eqref{eq:betars} it is
straightforward to find the radius and the Lorentz factor of the
shocked material when the reverse shock crosses the trailing edge of
the jet.

Fig.~\ref{fig:diss} shows the fraction of the slow component
($\Gamma_{\rm j}=2$) kinetic energy dissipated by the reverse shock as a
function of isotropic jet luminosity and $n_{18}$. For the purposes of
our analytic estimates, we assume a constant luminosity jet, which
shuts off after time $5 \times 10^{5}$ s. For the numerical
calculations described in the subsequent sections, the jet luminosity
declines as $t^{-5/3}$ after this time (see equation~\ref{eq:lum}).

\begin{figure}
\includegraphics[width=8cm]{diss.pdf}
\caption{\label{fig:diss} Fraction of the slow component
  ($\Gamma_{\rm j}=2$) kinetic energy dissipated by the reverse shock
  vs. the isotropic jet luminosity ($L_{\rm j,iso}$) and $n_{18}$.}
\end{figure}


\section{Jet Models}
\label{sec:numerical}

The jet-CNM interaction is simulated in one and two-dimensions, and
the resulting synchrotron emission is computed using the same
procedure described in \citet{Mimica+2015}. We assume the same angular
Lorentz factor dependence as in previous paper (i.e., $\Gamma = 10$
for the fast inner core and $\Gamma = 2$ for the slow outer sheath),
but introduce a number of modifications regarding the jet energy
when performing 1D simulations.

\subsection{1D vs. 2D Models}
\label{sec:2D}

The preferred numerical model for SwJ1644 \citep[Fig.10
in][]{Mimica+2015} was obtained using 2D simulations (the red line in
that figure). However, the light curve of a 1D version of the same
model \citep[black line in Fig. 10 in][see also section 4.2 of that
paper]{Mimica+2015} matches the 2D light curve only at early times,
when the emission from the inner relativistic jet dominates, while at
late times, when the emission from the slow outer core dominates, the
1D model overestimates the emission from the 2D model. In this work we
found a modification of the 1D model that makes its light curves
match the 2D results much more closely.

In the two component model of \citet{Mimica+2015}, the jet has a fast
inner core spans an angular interval $[0, 0.1\ {\rm rad}]$, while the
slow outer sheath occupies $[0.1\ {\rm rad}, 0.5\ {\rm rad}]$. For
both components we assume $E_{\rm ISO} = 4\times 10^{54}$ erg. The
crucial thing to notice is that, keeping $E_{\rm ISO}$ constant, the
true jet energy depends only on the angular interval: $E_{\rm jet} =
E_{\rm ISO} (\cos\theta_{\rm j,min} - \cos\theta_{\rm j,max})$. The
hydrodynamic evolution of the components is independent of angle in 1D
simulations, but the radiative transfer/light curve calculation is
sensitive to the jet geometry.

Although the sheath is injected in a relatively
narrow angular interval, at late stages of the jet evolution the bow
shock created by its interaction with CNM spans a much larger interval,
i.e. the slow component becomes almost isotropic \citep[bottom two
panels in Fig. 8 in][]{Mimica+2015}. Thus, the light curves from the
2D simulations differ from those of the 1D simulations in which the
angular size of the jet is held fixed. We note that the stage at
which the jet becomes ``isotropic'' depends on the CNM density,
i.e. the denser the CNM, the faster this is expected to happen.

Since the agreement between 1D and 2D models is good for early times
(when the fast component dominates), we modify only the slow component
in 1D models. We assume that is spans $[0.1\ {\rm rad}, \pi/2\ {\rm
  rad}]$ and lower its isotropic equivalent energy so that the true
jet energy remains unchanged with respect to the original model:

\begin{equation}\label{eq:Eiso}
 E_{\rm ISO, new} = E_{\rm ISO} \left(\frac{\cos(0.1) - \cos(0.5)}{\cos(0.1) - \cos(\pi/2)}\right)\approx 0.12 E_{\rm ISO}\, .
\end{equation}
%
We assume the same time dependence for the jet luminosity as in
\citet{Mimica+2015},

\begin{equation}\label{eq:lum}
L_{\rm j, ISO}(t) = L_{j,0}\max\left[1, (t/t_0)\right]^{-5/3}
\end{equation}
%
where $t_0 = 5\times 10^5$ seconds. Integrating equation~\ref{eq:lum}
in time from $0$ to $\infty$ would give $E_{\rm ISO}$. Thus,
$L_{j,0}=0.4\, E_{\rm ISO}/t_0$. 

We show a comparison of this modified 1D approach with the true 2D
result in Figure~\ref{fig:1D2DB}, for $n_{18}=60$ and $n_{18}=2000$
and $E_{\rm ISO}=4 \times 10^{54}$ ergs. For $n_{18}=2000$ the
agreement is excellent, while for $n_{18}=60$ the 1D models still
over-predict still over-predict the flux at times after the peak of
the light curve.

% We summarize our fiducial jet model and our grid of simulations in
% Table~\ref{tab:jetParams}

\begin{table}
\begin{threeparttable}
  \caption{\label{tab:jetParams} Summary of parameters for grid of on
    axis jet simulations.}
  \begin{tabular*}{0.95\columnwidth}{lll}
    \hline
    {Fast component} & Fiducial value & Other values \\ 
    $\Gamma_{\rm j}$ & 10 &\\
    $[\theta_{\rm min}$, $\theta_{\rm max}]$ & [0, 0.1] radians & \\
    $E_{\rm ISO}/10^{54} {\,\rm erg}$ & 4  & 0.04, 0.4\\
    $E$ & $2 \times 10^{52}$ ergs\\
    % $L_{\rm j,ISO}$ & $L_{j,0} \max\left[1, (t/t_0)\right]^{-5/3}$  \\
    % $t_0$ & $5\times 10^5$ s &\\
    % $L_{j,0}$ & $3.2 \times 10^{48}$ erg/s & \\
    \hline 
    Slow component\\
    $\Gamma_{\rm j}$ & 2 \\
    $[\theta_{\rm min}$, $\theta_{\rm max}]$ & [0.1, $\pi/2$] radians
    & \\
    $E_{\rm ISO}/10^{54} {\rm erg}$ & $4.7$ & 0.047, 0.47 \\
    $E$ & $4.7 \times 10^{53}$ erg & \\
    % $L_{\rm j, ISO}$ & $L_{j,0} \max\left[1, (t/t_0)\right]^{-5/3}$ & \\
    % $t_0$ & $5\times 10^5$ s & \\
    % $L_{j,0}$ & $3.8 \times 10^{47}$ erg/s & \\
    \hline
    Micro-physical parameters\\
    $\epsilon_e$ & 0.1 &  0.05, 0.2\\
    $\epsilon_b$ & 0.002 & 0.001, 0.005\\
    $p$ & 2.3\\
    \hline 
    Nuclear gas density \\
    $n_{18}$ & 60 & 2, 11, 345, 2000
  \end{tabular*}
% \begin{tablenotes}
% \item $^{\dagger}$  Additional values of physical parameters we tried.
% \end{tablenotes}
\end{threeparttable}
\end{table}


\begin{figure*}
\includegraphics[width=16cm]{1D_2D.pdf}
\caption{\label{fig:1D2DB} Comparison of 1D and 2D light curves for
  $n_{18}=60$ (top) and $n_{18}=2000$ (bottom) for frequencies of 1
  GHz (left) and 5 GHz (right) and an observer angle of 0.8 radians to
  the jet axis. We assume that the gas density $n\propto r^{-1}$ for
  $n_{18}=2000$, but take $n\propto r^{-1.5}$ for $n_{18}=60$ for
  computational convenience--this model was previously computed in
  \citet{Mimica+2015} and 1D results suggest that the density slope
  would have minimal impact on the results (see
  $\S$~\ref{sec:profileComp})}.
\end{figure*}

\section{Results}
\label{sec:results}
We calculate light curves for a grid of on-axis jet simulations, with
five different values of $n_{18}$ (2, 11, 60, 345, and 2000 cm$^{-3}$)
and three different values of the jet energy ($5\times 10^{51}$,
$5\times 10^{52}$, $5\times 10^{53}$ erg). The grid is summarized in
Table~\ref{tab:jetParams}.

The left-hand side of Fig.~\ref{fig:jetContours} shows contours of the
total peak luminosity (including both the fast core and slow sheath)
and the peak luminosity of just the slow component across this
grid. In general, the slow component dominates for large densities
and small frequencies.  This is further illustrated in
Fig~\ref{fig:components}, which shows the relative contributions of
the fast and slow components to the 5 GHz light curve of a $5 \times
10^{53}$ erg jet for different ambient gas densities. For
$n_{18}=2000$, the slow component dominates for nearly all times.  For
$n_{18}=2$, the fast component dominates at early times and at the
peak of the light curve, while the slow component dominates after
$\sim$1 year.

Overall, the peak luminosity of the slow component declines at low
densities, especially at 30 GHz. However, the fast component
compensates for this decline, so that the total peak luminosity is
very weakly dependent on density across the entire grid {\bf AG Not
  satisfying. Seems very coincidental and model dependent}. 

\begin{figure}
\includegraphics[width=8cm]{components.pdf}
\caption{\label{fig:components} Contributions from the fast
  (\textit{red}) and slow (\textit{blue}) to the 5 GHz light curves
  (\textit{black}) for $n_{18}=2$ (\textit{top}) and $n_{18}=2000$
  (\textit{bottom}).}
\end{figure}

% The right-hand side of Fig.~\ref{fig:jetContours} shows a comparison
% between the analytic scaling relation for the peak luminosity
% (equation~\ref{eq:peakLumGen}), and the numerical results for the
% slow component. 

The dependence of slow component peak luminosity on jet parameters and
nuclear gas density can be calculated analytically (see
equation~\ref{eq:peakLumGen}). Substituting in our fiducial values for
power law slopes of the gas density profile and energy distribution of
accelerated electrons ($p$) gives

\begin{align}
\nu L_{\nu, p}&=
\begin{cases}
  A_1 \left(\frac{E}{10^{54} {\rm ergs}}\right)^{0.59}
  \left(\frac{\epsilon_e}{0.1}\right)^{1.3}
  \left(\frac{\epsilon_b}{0.002}\right)^{0.825}\\
  \left(\frac{\nu_{\rm obs}}{5 {\rm GHz}}\right)^{0.35} n_{18}^{1.24} {\rm ergs/s} & {\rm Case \,1}\\\\
  A_2 \left(\frac{E}{10^{54} {\rm ergs}}\right)^{0.87}
  \left(\frac{\epsilon_e}{0.1}\right)^{0.61}
  \left(\frac{\epsilon_b}{0.002}\right)^{0.26}\\
  \left(\frac{\nu_{\rm obs}}{5 {\rm GHz}}\right)^{2.01} n_{18}^{-0.14}
  {\rm ergs/s} & {\rm Case \,2}
\end{cases}
\label{eq:peakLum}
\end{align}
%
where $A_1$ and $A_2$ are overall normalization constants.  Case 1
would be relevant for a low density medium, in which the jet is
optically thin at the deceleration time. In this case, the peak luminosity
is quite sensitive to density. Case 2 would be relevant for a high
density medium, in which the jet is optically thick at the
deceleration time. In this case, the peak luminosity is only weakly
dependent on the ambient gas density. This is qualitatively consistent
with the numerical results for the slow component in
Fig.~\ref{fig:jetContours}.

The right hand side of Fig.~\ref{fig:jetContours} shows comparisons of
the numerical results for the slow component with the results of
equation~\eqref{eq:peakLum}, with $A_1=7.6\times 10^{39}$ and $A_2=1.7
\times 10^{41}$ to match the results for low and high nuclear gas
densities respectively.

Equation~\eqref{eq:tpeakGen} gives the peak time for the slow
component. Substituting in our fiducial values for power law slopes of
the gas density profile and energy distribution of accelerated
electrons, gives


\begin{align}
t_p=
\begin{cases}
  t_1 E_{54}^{0.5} n_{18}^{-0.5} & {\rm Case \, 1}\\\\
  t_2 \left(\frac{E}{10^{54} {\rm ergs}}\right)^{0.32}
  \left(\frac{\epsilon_e}{0.1}\right)^{0.45}
  \left(\frac{\epsilon_b}{0.002}\right)^{0.37}\\
  \left(\frac{\nu_{\rm obs}}{5 {\rm GHz}}\right)^{-1.1} n_{18}^{0.4} &
  {\rm Case \, 2}
\end{cases}
\label{eq:tpeak}
\end{align}
%
where $t_1$ and $t_2$ are normalization constants. In general, the
light curve of a higher energy jet will peak later than the light
curve for a low energy jet. The scaling with density is more
complicated: if the emitting region is optically thick at the
deceleration time (``Case 2''), a higher ambient density results in a
later peak time. The peak occurs when the self-absorption frequency
crosses through the observing band and this occurs later for higher
ambient gas densities. Otherwise, the peak time will scale with the
deceleration time, and a larger ambient gas density will result in a
later peak (``Case 1'').

The left-hand side of Fig.~\ref{fig:ContoursTp} shows contours of the
peak time for the overall light curve and the peak time for just the
slow component. The right-hand side of Fig.~\ref{fig:ContoursTp} shows
the Case 2 peak time from equation~\eqref{eq:tpeak}, which roughly
reproduces the numerical results for high densities/low
frequencies. For lower densities, the numerical results diverge from
the analytic prediction, in part due to the contributions of the fast
component.  Furthermore, the peak time for the 30 GHz light curves
decreases with density for small $n_{18}$, as the emitting region is
optically thin at the deceleration time (``Case 1'').


\begin{figure*}
  \includegraphics[width=16cm]{lp_contours_new.pdf}
  \caption{\label{fig:jetContours} {\it {Left:}} Thick solid lines
    show peak luminosity as a function of jet energy and density at
    $10^{18}$ cm for our fiducial gas density profile, $n\propto
    r^{-1}$. Thin lines show contours of peak luminosity for the slow
    component of the jet (see~\ref{sec:numerical}). {\it Right:} Analytic
    scaling for the peak luminosity (equation~\ref{eq:peakLum})
    compared to the numerical results for the slow component.}
\end{figure*}


\begin{figure*}
  \includegraphics[width=16cm]{tp_contours_new.pdf}
  \caption{\label{fig:ContoursTp} {\it {Left:}} Thick solid lines show
    time of peak in days as a function of jet energy and density at
    $10^{18}$ cm for our fiducial gas density profile, $n\propto
    r^{-1}$. Thin lines show contours of peak time for the slow
    component of the jet (see~\ref{sec:numerical}). {\it Right:}
    Analytic scaling for the peak luminosity
    (equation~\ref{eq:peakLum}) compared to the numerical results for
    the slow component.}
\end{figure*}



% The following equation roughly reproduces the behavior of the peak
% time:

% \begin{align}
%   t_p =& 39.5 \left(\frac{E}{5 \times 10^{53} {\rm ergs}}\right)^{0.3}   \left(\frac{\epsilon_e}{0.1}\right)^{0.4}
%   \left(\frac{\epsilon_b}{0.002}\right)^{0.4}\nonumber\\
%  & \left(\frac{\nu}{5 {\rm GHz}}\right)^{-1} n_{18}^{0.35}  {\rm days}
% \label{eq:peakTime}
% \end{align}
% %
% Note that the peak time occurs later for larger $n_{18}$, as the
% transition from optically thick to optically thin occurs at later times.

%\subsection{Emission by components}


\subsection{Comparison with radio detections and upper limits.}

In the left panel of Fig.~\ref{fig:lightcurves} we show on-axis light
curves for our fiducial $5\times 10^{53}$ erg jet (comparable to
SwJ1644) and three different densities: n$_{18}$: 2, 60, and 2000
cm$^{-3}$ together with radio upper limits and detections for various
TDE candidates in Fig.~\ref{fig:lightcurves} (compiled from various
sources into Table 1 of \citealt{Mimica+2015}).  All of the 5 GHz
light curves fall above the upper limits. However, we were unable to
perform a 2D calculation for $n_{18}=2$.  For $n_{18}=60$, the 2D
light curve falls $\sim$one order of magnitude below the 1D light
curve after $\sim$ 10 years. Thus, we caution that the 1D light curves
for $n_{18}=2$ likely overestimate the late time radio luminosity. For
this density even a SwJ1644 like jet would fall below existing upper
limits. 

Multiple radio measurements from several months to a few years after a
tidal disruption flare would provide better constraints on the
presence of TDE jets. For example, suppose we have a TDE candidate
luminosity distance $d_L$ away and the radio flux density at 5 GHz is
constrained to be less than $S$. Then from equation~\eqref{eq:peakLum},
we can constrain the energy of the jet to be

\begin{equation}
E\lsim 7.6 \times 10^{48} \left(\frac{S}{50 \,\mu{\rm Jy}}\right)^{1.1}
  \left(\frac{d}{200 \,{\rm Mpc}}\right)^{2.2} {\rm ergs}'
\end{equation}


\begin{figure*} 
  \includegraphics[width=8cm]{lightcurves.pdf}
  \includegraphics[width=8cm]{lightcurves_en.pdf}
  \caption{\label{fig:lightcurves} \textit{Right-hand panel:} On-axis
    ($\theta_{\rm obs}=0$) radio light curves for our fiducial jet
    model and three different values of n$_{18}$: 2, 60, and
    2000. Solid lines correspond to the light curves from 1D jet
    simulations. When available ($n_{18}=60$ and $n_{18}$=2000), we
    have plotted light curves from the 2D jet simulations as dashed
    lines. Note we use $n\propto r^{-1}$ for $n_{18}=$ 2 and 2000, but
    $r^{-1.5}$ for $n_{18}=60$, as this model had already been
    computed and since the 1D results suggest that the density slope
    will have minimal impact on the results (see
    $\S$~\ref{sec:profileComp}).  Radio upper limits and detections
    are shown as black/gray triangles and squares respectively
    (compiled in Table 1 of \citealt{Mimica+2015}). The single upper
    limit in the top panel corresponds to 1.4 GHz. Gray triangles and
    squares in the middle panel indicate upper limits and detections
    at 3.0 GHz, while black triangles indicate upper limits at 5.0
    GHz. \textit{Left-hand panel} 1D on-axis light curves for a jet
    with a total energy of $5\times 10^{51}$ ergs, together with radio
    upper limits and the light curves for our fiducial case
    (translucent lines).}
\end{figure*}

\subsection{Effects of viewing angle}
Fig.~\ref{fig:onOff} shows a comparison off and on-axis light curves.
For $n_{18}=2000$ the light curves differ little.  For $n_{18}=60$
off-axis luminosity is lower by a factor of a few before and near the
peak. However, for late times the off- and on-axis light curves are
quite similar. This is as expected, at late times the jet becomes more
isotropic, which means the viewing angle has relatively little effect
on the observed light curve.

\begin{figure*}
\includegraphics[width=16cm]{on_off.pdf}
\caption{\label{fig:onOff} Comparison of (2D) off- and on-axis light
  curves for $n_{18}=60$ (top) and $n_{18}=2000$ (bottom) for
  frequencies of 1 GHz (left) and 5 GHz (right). The observer line of
  sight is taken to be at $\theta_{\rm obs}=0.8$ with respect to the
  jet axis. We assume an $n\propto r^{-1}$ density profile for
  $n_{18}=2000$, but take $n\propto r^{-1.5}$ for $n_{18}=60$ for
  computational convenience--this model was previously computed in
  \citet{Mimica+2015} and 1D results suggest that the density slope
  would have minimal impact on the results (see
  $\S$~\ref{sec:profileComp}).}
\end{figure*}

% \subsection{Effect of jet energy}
% In the left-hand panel of Figure~\ref{fig:lightcurves} we show radio
% light curves for a jet with energy $E=5\times 10^{51}$ ergs--two
% orders of magnitude smaller than for the high energy jet.  The peak
% luminosities are also $\sim$ two orders of magnitude lower, as the
% peak luminosity is approximately linear in jet energy. 

% For low densities ($n_{18}\sim 60$ cm$^{-3}$) the radio light curves
% for the low energy jet fall below the observed radio upper limits at
% late times. 


\subsection{Effect of gas density profile}
\label{sec:profileComp}
We have also computed the radio light curves for different shapes of
the gas density profile, while holding $n_{18}$ fixed. In
Fig.~\ref{fig:cores}, we show on-axis radio light curves for our
fiducial $n\propto r^{-1}$ and a core galaxy profile
(equation~\ref{eq:cores}), both with $n_{18}=2$.
The results are within a factor of a few of each other. The core and
cusp light curves are even closer at higher densities, and virtually
indistinguishable at $n_{18}=2000$. For larger densities, the jet only
samples small radii, where the core and cusp profiles are quite
similar (see Fig.~\ref{fig:profiles}). It is only at lower densities,
for which the Sedov radius lies outside of the flattening of the core
density profile, that noticeable differences emerge. {\bf AG why are
  there differences at early times for $n_{18}=2$ but not
  $n_{18}=2000$. Is this simply an artifact of the starting radius?}

In Fig.~\ref{fig:profs2} we compare the 1D on-axis light curves for
density two density profiles: $n\propto r^{-1}$ and $n\propto r^{-1.5}$
with $n_{18}=60$. For most times the light curves are very close.
This is as expected, since the Sedov radius, $r_{\rm sedov}\sim
10^{18}$ cm. 

Overall we conclude the slope of the stellar density will profile will
have minimal effect on the radio light curve, as long as the density
near the deceleration radius is fixed. 

\begin{figure} 
  \includegraphics[width=8cm]{fig_cores.pdf}
  \caption{\label{fig:cores} Comparison of (on-axis) light curves for
    $n_{18}=2$ and two different CNM gas density profiles: $n\sim
    r^{-1}$ and the core galaxy profile from \eqref{eq:cores} with
    $r_s=10^{18}$ cm.}
\end{figure}


\begin{figure} 
  \includegraphics[width=8cm]{profs2.pdf}
  \caption{\label{fig:profs2} Comparison of (on-axis) light curves for
    $n_{18}=60$ and two different CNM gas density profiles: $n\propto
    r^{-1}$ ({\it solid black}) and $n\propto r^{-1.5}$ ({\it dashed
      red}).}
\end{figure}



\section{Summary and Conclusions}
\label{sec:conc}

We calculate radio light curves for tidal disruption event jets
propagating through different circumnuclear (CNM) gas densities. We
simulate the jet propagation using both 1D and 2D hydrodynamic
simulations. We then post-process these to produce radio synchrotron
light curves. To isolate the effects of the density profile we take a
fixed two component jet model from \citet{Mimica+2015}, which produces
a good fit to the observed radio data in SwJ1644. We
consider a broad range of gas densities motivated by analytic
estimates of stellar wind mass injection and empirical constraints
based on observed distributions of Eddington ratios. Our conclusions
are summarized as follows.

\begin{enumerate}
\item We estimate the nuclear gas densities expected from injection of
  stellar wind material for different star formation histories. We
  find that that range of gas densities at 10$^{18}$ cm is $n_{18}
  \sim$ 0.5-2000 cm$^{-3}$.

\item The slope of the gas density profile depends on the slope of the
  stellar density profile. We expect a typical TDE host to have cuspy
  stellar density profile inside of a few pc, with $\rho_\star
  \propto r^{-1.7}$. This translates into a gas density profile $n
  \propto r^{-1}$. The radio light curve of a TDE jet is most
  sensitive to the density at the deceleration/Sedov radius (where it
  has swept up its mass in CNM gas). The light curve will be
  insensitive to changes in slope for fixed density at the
  deceleration radius.

\item We use the distribution of Eddington ratios (measured by
  \citealt{Kauffmann&Heckman2009} using $L[OIII]/\Mbh$) for a sample
  $\sim 10^{7} \Msun$ black holes from SDSS to obtain an empirical
  constraint on the average circumnuclear gas density, including all
  phases of gas. We find that $\sim90\%$ of galaxies in the sample
  have $n_{18}<10^{4} \, {\rm cm}^{-3}$, although we note that there is
  considerable uncertainty in translating an observed Eddington ratio
  to a circumnuclear gas density. TDE hosts would not fall on this
  high density tail, as TDE hosts have either weak or no detected OIII
  emission.

\item We take a jet model which fits the radio data for the SwJ1644
  transient (from \citealt{Mimica+2015}) and run it through a range of
  different density profiles. Motivated by the above results for the
  expected range of gas densities we take the density at $10^{18}$ cm,
  to be $n_{18}=2, 60,$ or 2000 cm$^{-3}$. We find bright radio
  emission at a few GHz across this entire range of densities, with
  the peak luminosity only weakly dependent on the chosen value of
  $n_{18}$.  For smaller densities the light curves peak earlier in
  time. Based on existing radio upper limits for tidal disruption
  event candidates, we show SwJ1644-like jet are absent in most TDEs
  as long as the density at $10^{18}$, $n_{18} \gsim  60$
  cm$^{-3}$. Prompt follow-up in the radio could provide tighter
  constraints on the existence of TDE jets.  
\end{enumerate}

\appendix
\section{Core Profile}
\label{app:core}
In $\S\ref{sec:profileComp}$ we compare the results of radio
light curves from jets propagating in core and cusp like gas density
profiles. See Fig.~\ref{fig:profiles} for a comparison of core and
cusp-like profiles. 

We use the following analytic expression to approximate the core
galaxy profile

\begin{align}
\begin{cases}
n=n(r_s) k(x) & 0.4 \leq x\leq 2.0\\
n = 2.0 n(r_s) (x/0.4)^{-0.95} & x < 0.4\\
n = 0.75 n(r_s) (x/2.0)^{-0.26} & x>2.\\
\end{cases}
\label{eq:cores}
\end{align}

Where, 

\begin{align}
  &x=r/r_s\\\nonumber
  &k(x)=\frac{45}{19} \frac{1}{x^{3/2}} \frac{1-x^{1.9}}{9-19
      x\frac{x^{0.9}-1}{x^{1.9}-1}}
\end{align}

To isolate the effects of the shape of the density profile, consider a
core density profile with $r_s=10^{18}$ cm, and $n_{18}=2000$: the
same as our high density cusp model.

\section{Peak Luminosities and times}
\label{app:analyt}
\citet{Leventis+2012} present analytic scaling relations for the
synchrotron flux of a spherical blast wave propagating through a
medium with a power law density profile, $n\propto r^{-k}$
(generalizing the \citealt{Nakar&Piran2011} scalings for a
blast wave propagating through a constant density medium). In this
appendix, we use these results to write down an expression for the
peak flux of the slow component of the jet.

It is very important to take synchrotron self-absorption into account,
particularly for high ambient gas densities. For the late stage
Newtonian evolution of the jet, the spectrum will be self-absorbed for
frequencies below

\begin{align}
  \nu_{\rm sa}=&C_1(p, k) E_{54}^{\frac{10 p-k p -6 k}{2 (4+p) (5-k)}} n_{18}^{\frac{30 - 5 p}{2 (4 + p) (5 - k)}}
  \epsilon_e^{\frac{2 (p-1)}{4+p}} \epsilon_b^{\frac{p+2}{2 (4+p)}}\nonumber\\
  &t^{\frac{10 - 8 k - 15 p + 4 k p}{(4 + p) (5 - k)}}, 
\label{eq:nuSa} 
\end{align}
%
where $p$, $\epsilon_e$, and $\epsilon_b$ are the standard parameters
describing the microphysics of the shocked electrons, $E_{54}$ is the
blast wave energy in units of $10^{54}$ erg, and $C_1(p, k)$ is an
overall normalization factor.  Note that equation~\eqref{eq:nuSa}
assumes that self-absorption frequency is greater than the synchrotron
peak frequency

\begin{equation}
\nu_m=C_2(p, k) E_{54}^{\frac{10-k}{2 (5-k)}} n_{18}^{-\frac{5}{2
    (5-k)}}  \epsilon_e^2  \epsilon_b^{1/2}  t^{\frac{4 k-15}{5-k}}.
\end{equation}
%
% The observing frequencies considered here ($\geq 1$ GHz), are also
% greater than $\nu_m$. 
The peak of the light curve will at the deceleration time
(equation~\ref{eq:tdec}) if the self-absorption frequency is below the
observing band then (``Case 1''). Otherwise, it will occur after the
deceleration time, when the self-absorption frequency crosses through
the observing band (``Case 2''). The peak time for these two cases is

\begin{align}
t_{\rm p}=
\begin{cases}
  t_{\rm dec}\approx \left(100 \, E_{54}\right)^{1/(3-k)}
  n_{18}^{-1/(3-k)} {\rm year} & {\rm Case\, 1}\\\\
  C_1(p, k)^{-\frac{(5 - k) (4 + p)}{10 - 8 k - 15 p + 4 k
      p}}E_{54}^{-\frac{-k p-6 k+10 p}{2 (4 k p-8 k-15 p+10)}}\\
  \times n_{18}^{-\frac{30-5 p}{2 (4 k p-8 k-15 p+10)}} \nu_{\rm
    obs}^{\frac{(5-k) (p+4)}{4 k p-8 k-15 p+10}}\\
  \times  \epsilon_b^{-\frac{(5-k) (p+2)}{2 (4 k p-8 k-15 p+10)}} \epsilon_e^{-\frac{2 (5-k) (p-1)}{4 k p-8 k-15 p+10}} & {\rm
    Case\, 2}
\end{cases}
\label{eq:tpeakGen}
\end{align}
%
Note that we assume the blast wave is modestly relativistic with
velocity comparable to $c$ and a Lorentz factor of order unity {\bf
  AG:strictly speaking not self-consistent as in the scaling relation
  the blast wave is assumed to be in the ST phase, and so
  non-relativistic}. The unabsorbed flux at the peak frequency is
given by

\begin{align}
  F_{\nu_m} =  C_3(p, k) E_{54}^{\frac{8-3 k}{2 (5-k)}}
  n_{18}^{\frac{7}{2 (5-k)}} \epsilon_b^{1/2} t^{\frac{3-2 k}{5-k}}
\label{eq:Fnum}
\end{align}
%
Extrapolating to the observer frequency gives 

\begin{align}
  \nu_{\rm obs} F_{\rm p} (\nu_{\rm obs}) &= \nu_{\rm obs}   F_{\nu_m}
  \left(\frac{\nu_{\rm obs}}{\nu_m}\right)^{-(p-1)/2}.
  \label{eq:Fpeak1}
\end{align}
%
Then from equations~\eqref{eq:tpeak}, ~\eqref{eq:Fnum},
and~\eqref{eq:Fpeak1}

\begin{align}
  \nu_{\rm obs} F_{\rm p} (\nu_{\rm obs}) \propto
  \begin{cases}
    E_{54}^{\frac{k (p+5)-12}{4 (k-3)}} n_{18}^{-\frac{3 (p+1)}{4
        (k-3)}} \nu_{\rm obs}^{\frac{3-p}{2}}
    \epsilon_b^{\frac{p+1}{4}} \epsilon_e^{p-1} & {\rm Case \,
      1}\\\\
    E_{54}^{\frac{k(-(p-2))-10 p+3}{4 k (p-2)-15 p+10}} \\ \times
    n_{18}^{\frac{11 (p-2)}{4 k (p-2)-15 p+10}} \nu_{\rm
      obs}^{\frac{14 k (p-2)-47 p+57}{4 k (p-2)-15 p+10}} \\ \times
    \epsilon_b^{\frac{k (-(p-2))+p-8}{4 k(p-2)-15 p+10}}
    \epsilon_e^{-\frac{11 (p-1)}{4 k (p-2)-15p+10}} & {\rm Case \, 2}
  \end{cases}
  \label{eq:peakLumGen}
\end{align}




\clearpage
  \footnotesize{
    \bibliographystyle{mnras}
    \bibliography{master}
  }

\end{document}
