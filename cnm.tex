\documentclass[usenatbib,fleqn]{mnras}

\makeatletter
\newlength{\abovecaptionskip}%
\setlength{\abovecaptionskip}{10\p@}
\makeatother


\usepackage{mathtools}
\usepackage{threeparttable}
 

\usepackage{amsmath,amssymb}
\usepackage{cases}
\usepackage{mathrsfs}
\usepackage{graphicx}
\usepackage{epstopdf}
%\usepackage{hyperref}
\epstopdfsetup{outdir=./figures/}
\graphicspath{{./figures/}}
\usepackage{url}
%\usepackage{aas_macros}

\newcommand\lsim{\mathrel{\rlap{\lower4pt\hbox{\hskip1pt$\sim$}}
    \raise1pt\hbox{$<$}}}
\newcommand\gsim{\mathrel{\rlap{\lower4pt\hbox{\hskip1pt$\sim$}}
    \raise1pt\hbox{$>$}}}


\newcommand       \be          {\begin{eqnarray}}
\newcommand       \ee          {\end{eqnarray}}
\newcommand{\Mbh}[1][]{M_{\bullet#1}}
\newcommand{\Menc}{M_{\rm enc}}
\renewcommand{\th}{t_h}
\newcommand{\Msun}{{\rm M_\odot}}
\newcommand{\pyear}{{\rm yr}^{-1}}
\newcommand{\rs}{r_s}

% write title (with email and institute)
\title[Influence of CNM on TDE radio emission]{The influence of
  circumnuclear environment on the radio emission from TDE jets}
\author[Generozov et al.]{ A. Generozov$^{1}$, P. Mimica$^{2}$,
  B. D. Metzger$^{1}$, D. Giannios$^{3}$, N. Stone$^{1}$,
  M.A. Aloy$^{2}$
  \\
  $^{1}$Columbia Astrophysics Laboratory, Columbia University, 550 West 120th Street, New York, NY 10027\\
  $^{2}$Departamento de Astronom\'{\i}ia y Astrof\'{\i}sica, Universidad de Valencia, E-46100 Burjassot, Spain\\
  $^{3}$Department of Physics and Astronomy, Purdue University, 525
  Northwestern Avenue, West Lafayette, IN 47907, USA}

\begin{document}
\maketitle
\begin{abstract}
  Dozens of stellar tidal disruption events (TDEs) have been
  identified at optical, UV and X-ray wavelengths.  A small fraction
  of these, most notably {\it Swift} J1644+57, produce radio
  synchrotron emission, consistent with a powerful, relativistic jet
  shocking the surrounding circumnuclear gas.  The dearth of similar
  non-thermal radio emission in the majority of TDEs may imply that
  powerful jet formation is intrinsically rare, or that the conditions
  in galactic nuclei are typically unfavorable for producing a
  detectable signal.  Here we explore the latter possibility by
  constraining the radial profile of the gas density encountered by a
  TDE jet using a one-dimensional model for the circumnuclear medium
  which includes mass and energy input from a stellar population.
  Near the jet Sedov radius of 10$^{18}$ cm, we find gas densities in
  the range of $n_{18} \sim$ 0.5$-$2000 cm$^{-3}$ across a wide range
  of plausible star formation histories.  Using one- and
  two-dimensional relativistic hydrodynamical simulations, we
  calculate the synchrotron radio light curves of TDE jets (as viewed
  both on and off-axis) across the allowed range of density profiles.
  We find that bright radio emission would be produced across the
  plausible range of nuclear gas densities by jets as powerful as {\it
    Swift} J1644+57, and we quantify the relationship between the
  radio luminosity and jet energy.  We use existing radio detections
  and upper limits to constrain the energy distribution of TDE jets.
  Radio follow up observations several months to several years after the TDE
  candidate will strongly constrain the energetics of any relativistic
  flow.
\end{abstract}
\section{Introduction}
\label{sec:intro}
When a star in a galactic nucleus is deflected too close to the
central supermassive black hole (BH), it can be torn apart by tidal
forces.  During this tidal disruption event (TDE), roughly half of the
stellar debris remains bound to the BH, while the other half is flung
outwards and unbound from the system.  The bound material, following a
potentially complex process of debris circularization
(\citealt{Kochanek1994,Guillochon+2013,Hayasaki+2013,Hayasaki+2015,Shiokawa+2015,Bonnerot+2015}),
accretes onto the BH, creating a luminous flare lasting months to
years \citep{Hills1975, Carter+1982, Rees1988}.

Many TDE flares have now been identified at optical/ultraviolet (UV)
\citep{Gezari+2008, Gezari+2009, van-Velzen+2011, Gezari+2012,
  Arcavi+2014, Chornock+2014, Holoien+2014, Vinko+2015, Holoien+2016}
and soft X-ray wavelengths \citep{Bade+1996, Grupe+1999,
  Komossa&Greiner1999, Greiner+2000, Esquej+2007, Maksym+2010,
  Saxton+2012}. Beginning with the discovery of {\it Swift} J1644+57
(hereafter SwJ1644) in 2011, three additional TDEs have been
discovered by their hard X-ray emission (\citealt{Bloom+2011,
  Levan+2011, Burrows+2011, Zauderer+2011, Cenko+2012, Pasham+2015,
  Brown+2015}).  Unlike the optical/UV/soft X-ray flares, these events
are characterized by non-thermal emission from a transient
relativistic jet beamed along our line of sight, similar to the blazar
geometry of active galactic nuclei (AGN).  In addition to their highly
variable X-ray emission, which likely originates from the base of the
jet (see e.g. \citealt{Bloom+2011, Crumley+2016}), these events are
characterized by radio synchrotron emission.  The latter, more slowly
evolving, is powered by shocks formed at the interface between the jet
and surrounding circumnuclear medium (CNM)
\citep{Bloom+2011,Giannios&Metzger2011,Metzger+2012,De-Colle+2012,Kumar+13,Mimica+2015},
analagous to the afterglow of a gamma-ray burst.

Although a handful of jetted TDE flares have been observed, the
apparent volumetric rate is a very small fraction ($\sim
10^{-5}-10^{-4}$) of the observed TDE flare rate (e.g.,
\citealt{Burrows+2011}, \citealt{Brown+2015}), and an even smaller
fraction of the theoretically predicted TDE rate
\citep{Wang&Merritt2004,Stone&Metzger2016}.  One explanation for this
discrepancy is that the majority of TDEs produce powerful jets, but
their hard X-ray emission is relativistically beamed into a small
angle $\theta_{\rm b}$ by the motion of the jet, making them visible
to only a small fraction of observers.  However, the inferred beaming
fraction $f_b \approx \theta_{b}^{2}/2 \sim 10^{-5}-10^{-4}$ would
require $\theta_{\rm b} \sim 0.01$ and hence a jet with a bulk Lorentz
factor of $\Gamma \gtrsim 1/\theta_{\rm b} \sim 100$, much higher than
inferred for AGN jets or by modeling SwJ1644
(\citealt{Metzger+2012}). This scenario 
would also require an unphysically low jet half opening angle
$\theta_j\lsim 0.01$.

The low detection rate of hard X-ray TDEs may instead indicate that powerful
jet production is intrinsically rare, or that the conditions in
the surrounding environment are unfavorable for producing bright
emission.  Jets could be rare if they require, for instance, a highly
super-Eddington accretion rate (\citealt{De-Colle+2012}), a TDE from a
deeply plunging stellar orbit (\citealt{Metzger&Stone2015}), a TDE in
a retrograde and equatorial orbit with respect to the spin of the
black hole \citep{Parfrey+2015}, or a particularly strong magnetic
flux threading the star (\citealt{Tchekhovskoy+2014,Kelley+2014}).
Alternatively, jet formation or its X-ray emission could be suppressed
if the disk undergoes Lens-Thirring precession due to a misalignment
between the angular momentum of the BH and that of the disrupted star
(\citealt{Stone&Loeb2012}).  In the latter case, however, even a
`dirty' jet could still be generated, which would produce luminous
radio emission from CNM interaction.

\citet{Bower+2013} and \citet{van-Velzen+2013} performed radio
follow-up of optical/UV and soft X-ray TDE flares on timescales of
months to decades after the outburst (see also
\citealt{Arcavi+2014}). They detect no radio afterglows definitively
associated with the host galaxy of a convincing TDE
candidate.\footnote{There were radio detections for two ROSAT flares:
  RX J1420.4+5334 and IC 3599. However, for RX J1420.4+5334 the radio
  emission was observed in a different galaxy than was originally
  associated with the flare.  IC 3599 has shown multiple outbursts in
  the recent years, calling into question whether it is a true TDE at
  all \citep{Campana+2015}. The optical transient CSS100217 (see
  \citealt{Drake+2011}) had a weak radio afterglow, but its peak
  luminosity is more consistent with a superluminous supernova than a
  TDE.} \citet{Bower+2013} and \citet{van-Velzen+2013} use a Sedov
blast wave model for the late-time radio emission to conclude that
$\lesssim 10\%$ of TDEs produce jetted emission at a level similar to
that in SwJ1644.  \citet{Mimica+2015} use two-dimensional
(axisymmetric) hydrodynamical simulations, coupled with synchrotron
radiation transport, to model the radio emission from SwJ1644 as a jet
viewed on-axis.  By extending the same calculation to off-axis viewing
angles, they showed that, regardless of viewing angle, the majority of
thermal TDE flares should have been detected if their jets were as
powerful as SwJ1644, which had a total energy of $\sim 5\times
10^{53}$ erg.


The recent TDE flare ASSASN-14li (\citealt{Holoien+2016a}) was
accompanied by transient radio emission, consistent with either a weak
relativistic jet \citep{van-Velzen+2015} or a sub-relativistic outflow
\citep{Alexander+2015,Krolik+16} of total energy $\sim
10^{48}-10^{49}$ erg.  The 90 Mpc distance of ASSASN-14li, a few
times closer than most previous TDE flares, implies that even if other
TDEs were accompanied by similar emission, their radio afterglows
would fall below existing upper limits.  The extreme contrast between
the radio emission of SwJ1644 and ASSASN-14li indicates that the
energy distribution of TDE jets is very broad.

Previous works (\citealt{Bower+2013}; \citealt{van-Velzen+2013};
\citealt{Mimica+2015}) have generally assumed that all TDE jets
encounter a similar gaseous environment as SwJ1644.  However, the
density of the circumnuclear medium (CNM) depends sensitively on the
input of mass from stellar winds and the processes responsible for
heating the gas (\citealt{Quataert2004,Generozov+2015}). 

The first goal of this paper is to constrain the range of gas
densities encountered by jetted TDEs using the semi-analytic model for
the CNM ($\S\ref{sec:cnm}$) developed in \citet{Generozov+2015}
(hereafter GSM15).  With this information in hand, in
$\S\ref{sec:results}$ we present hydrodynamical simulations of the
jet-CNM shock interaction which determine the radio synchrotron
emission across the allowed range of gaseous environments, for
different jet energies and viewing angles.  In $\S\ref{sec:param}$ we
show how the dependence of our results for the peak luminosity, and
time to radio maximum, on the jet energy and CNM density can be
reasonably understood using a simple analytic blast wave model
($\S\ref{sec:analyt}$, Appendix~\ref{app:analyt}), calibrated to the
simulation data.  Then, using extant radio detections and upper
limits, we systemtically constrain the energy distribution of TDE
jets.  One of our primary conclusions is that TDE jets as energetic as
SwJ1644 are intrinsically rare, a result with important implications
for the physics of jet launching in TDEs and other accretion flows.
Our work also lays the groundwork for collecting and employing future,
larger samples of TDEs with radio follow-up, to better constrain the
shape of the energy distribution.  We summarize and conclude in
$\S\ref{sec:conc}$.

\section{Diversity of CNM Densities}
\label{sec:cnm}

%In this section we place constraints on the gas densities in galactic
%nuclei.  In $\S\ref{sec:analy}$, we determine the possible range of
%densities resulting from mass loss by stellar winds. 

\subsection{Analytic Constraints}
\label{sec:analy}

%\subsubsection{Preliminary Considerations}

Jet radio emission is primarily sensitive to the density of ambient
gas near the Sedov radius, $r_{\rm sed}$, outside of which the jet has
swept up a gaseous mass exceeding its own. For a power law gas density
profile, $n= n_{18} \left(r/10^{18} {\rm cm}\right)^{-k}$,
\begin{align}
  r_{\rm sed} &= 10^{18} \,{\rm cm}\, \left( \frac{E(3-k)}{4\pi n_{18}
      m_{\rm p} c^2 (10^{18}\,{\rm cm})^3} \right)^{1/(3-k)}  \nonumber\\
  &\approx 3 E_{54}^{1/2} n_{\rm 18}^{-1/2}\,{\rm pc}.
  \label{eq:rdec}
\end{align}
where $E = E_{54}10^{54}$ erg is the isotropic equivalent energy and
in the final equality we have taken $k = 1$, typical of our results
described later in this section.  For a powerful jet similar to
SwJ1644, the deceleration radius is typically of order a parsec, but
it can be as small as $10^{16}$ cm for a weak jet/outflow, such as
that in ASASSN-14li.

Although an initially relativistic jet will slow to sub-relativistic
speeds at $r \sim r_{\rm sed}$, significant deceleration already sets
in at the deceleration radius (where the jet has swept up a
fraction $\sim 1/\Gamma$ of its rest mass),
\begin{equation}
  r_{\rm dec}=\frac{r_{\rm sed}}{\Gamma^{2/(3-k)}}.
  \label{eq:rdec2}
\end{equation}
According to an observer within the opening angle of the jet, the jet
reaches the Sedov and deceleration radii, respectively, at times given
by
\begin{equation}
t_{\rm sed} \simeq \frac{r_{\rm sed}}{c} \approx
10 E_{54}^{1/2}n_{18}^{-1/2} {\rm year}
 \end{equation} 
\begin{equation}
t_{\rm dec} \simeq
\frac{r_{\rm dec}}{2\Gamma^{2} c} = \frac{t_{\rm
    sed}}{2\Gamma^{2(4-k)/(3-k)}} = \frac{t_{\rm sed}}{2\Gamma^{3}},
 \label{eq:tdec}
\end{equation}
%
where in the final equality we have again taken $k = 1$.


\subsubsection{Dynamical Model of CNM}
\label{sec:model}

In the absence of large scale inflows, the dominant source of gas in
the CNM of quiescent galaxies is winds from stars in the galactic
nucleus. We bracket the range of possible nuclear gas densities using
a simple steady-state, spherically symmetric, hydrodynamic model
including mass and energy injection from stellar winds. The relevant
equations are (e.g. \citealt{Holzer+1970}; \citealt{Quataert2004})
\begin{align}
  &\frac{\partial \rho}{\partial t}+\frac{1}{r^2}\frac{\partial}{\partial r}\left(\rho r^2 v\right)=q \label{eq:drhodt}\\
  &\rho \left(\frac{\partial v}{\partial t} + v\frac{\partial
      v}{\partial r}\right) =-\frac{\partial p}{\partial r}- \rho\frac{GM_{\rm enc}}{r^{2}} -q v \label{eq:dvdt}\\
  &\rho T\left(\frac{\partial s}{\partial t} + v\frac{\partial
      s}{\partial
      r}\right)=q\left[\frac{v^2}{2}+\frac{v_w^2}{2}-\frac{\gamma_{\rm
      ad}}{\gamma_{\rm ad}-1}
    \frac{p}{\rho} \right] ,
\label{eq:model}
\end{align}
where $\rho = \mu m_p n$, $v$, $p$, and $s$ are the density, velocity,
pressure (we assume an ideal gas with $\mu = 0.62$ and adiabatic index
$\gamma_{\rm ad}$=5/3), and specific entropy of the gas,
respectively. The enclosed mass $\Menc = M_{\bullet} + M_{\star}$
includes both the black hole mass $M_{\bullet}$ and enclosed stellar
mass $M_{\star} \propto \int \rho_{\star}r^{2}dr$, where
$\rho_{\star}$ is the stellar density. 
% At the radius of the sphere of
% influence, $r_{\rm inf}$, the enclosed stellar and black masses are
% equal, $M_{\star}(r_{\rm inf})=\Mbh$. 
 We take $r_{\rm inf}=3.5
\Mbh[,7]^{0.6}$ (GSM15), where $\Mbh[,7]=\Mbh/10^7 \Msun$.

The source term $q$ is the mass injection rate per unit volume per
unit time. We take $q=\eta \rho_{\star}/t_h$, where $\eta$ is a
dimensionless efficiency parameter that depends on the properties of
the stellar population and $t_h$ is the Hubble time. The energy source
term $\propto v_w^{2}$, parameterizes the heating rate of the gas, as
physically results from stellar wind kinetic energy, supernovae, and
black hole feedback.

GSM15 present analytic approximations for the
densities and temperatures of steady state solutions to
equation~\eqref{eq:model}. We apply these results across the
physically allowed range of heating ($v_w$) and mass injection
rates ($\eta$), and obtain the corresponding range of gas densities.

\subsubsection{Stellar density profiles}
We assume a broken power law for the stellar density profile,
$\rho_{\star}$, motivated by Hubble measurements of the radial surface
brightness profiles for hundreds of nearby early type galaxies
\citep{Lauer+2007}.  The measured profile is well fit by the so-called
``Nuker'' law parameterization, i.e.~a piece-wise power law that smoothly
transitions from an inner power law slope, $\gamma$, to an outer power
law slope, $\beta$, at a break radius, $r_b$.

Most galaxies have $0<\gamma<1$, and are classified into two broad
categories: ``core" galaxies with $\gamma<0.3$ and ``cusp" galaxies with
$\gamma>0.5$. Assuming spherical symmetry and a constant mass-to-light
ratio, the inner stellar profile translates to a stellar density of
$\rho_\star\propto r^{-1-\gamma}=r^{-\delta}$. 

Cusp-like stellar density profiles are the most relevant to TDEs for
two reasons.  First, only a low mass black hole ($\Mbh\lsim 10^8
\Msun$) can disrupt a main sequence star, and low mass low mass black
holes are more commonly characterized by cusp-like profiles.
Additionally, as described in \citet{Stone&Metzger2016} a cuspy
stellar density profile results in a higher TDE rate per galaxy.  We
adopt a fiducial value of $\gamma=0.7$ ($\delta=1.7$), motivated by
the rate-weighted average value of the inner stellar density profile
for the galaxies in \citet{Stone&Metzger2016} (their Table C).


\subsubsection{Gas density profiles}
\label{sec:dProf}
Given sufficiently strong heating, a one-dimensional steady-state
model for the CNM is characterized by an inflow-outflow structure.
The velocity passes through zero at the ``stagnation radius'', $\rs$.
Mass loss from stars interior to the stagnation radius flows inwards,
while that outside of $\rs$ is unbound in an outflow from the nucleus.
Fig.~\ref{fig:profiles} shows example radial profiles of the
steady-state gas density calculated for a core and a cusp stellar
density profile. The stagnation radius is marked as a blue dot on each
profile.

As described in GSM15, the stagnation radius is approximately given by
\begin{align}
r_s \simeq f(\delta) \frac{G M_{\bullet}}{v_w^2}
  \simeq 0.4 \Mbh[,7] v_{500}^{-2} \,{\rm pc},
\label{eq:rs}
\end{align}
%
where $v_{500}\equiv v_w/500 \,{\rm km\, s^{-1}}$ and $f(\delta)$ is a
constant of order unity, which in the second equality we take equal to
its fiducial value of $f(\delta = 1.7)$=2.5. The gas density at the
stagnation radius, $n(\rs)$, is determined by the rate at which
stellar winds inject mass interior to it,
\begin{equation}
  \dot{M}=\frac{\eta M_{\rm \star}(\rs)}{t_h} \approx  2.8 \times 10^{-6} \Mbh[,7]^{0.22} \eta_{0.02} \left(\frac{r_s}{\rm
      pc}\right)^{1.3} \Msun \, {\rm yr}^{-1},
\label{eq:dotM}
\end{equation}
%
where $M_{\star}(\rs)$ is the total stellar mass enclosed within the
stagnation radius, $\eta_{0.02}=\eta/0.02$ is normalized to a value
characteristic of an old stellar population, and the second equality
again assumes our fiducial value of $\delta=1.7$.

The density at the stagnation radius, $n(\rs)$, is estimated by
equating the gas injected by stellar winds over a dynamical time at
the stagnation radius, $t_{\rm dyn} (\rs)=(\rs^3/G \Mbh)^{1/2}$, to
the gas mass enclosed at this location.  This gives
\begin{align}
  &\frac{4 \pi}{3} \rs^3 \mu m_p n(r_s) \simeq \dot{M} t_{\rm dyn}
  (\rs) \nonumber\\
  & n(r_s) \simeq 0.2 \eta_{0.02} \Mbh[,7]^{-0.28}
  \left(\frac{r_s}{\rm pc}\right)^{-0.2} {\rm cm}^{-3}.
\label{eq:nrs}
\end{align}
%
Substituting equation~\eqref{eq:rs} for $r_s$, we obtain 
\begin{equation}
n(r_s) \simeq 0.2 \, v_{500}^{0.4} \eta_{0.02} \Mbh[,7]^{-0.48} {\rm cm}^{-3},
\label{eq:nrs2}
\end{equation}
%
Near the stagnation radius, GSM15 found that the radial gas profile
has a power-law slope of $k \approx (4\delta-1)/6$, which for our
fiducial value of $\delta=1.7$ gives $n \propto r^{-1}$. The gas
profile flattens to $n\propto r^{1-\delta}$ between the stagnation
radius and the stellar break radius; however, for our fiducial value
of $\delta = 1.7$, the resulting profile $n\propto r^{1-\delta}
\approx r^{-0.7}$ is only moderately changed. Thus, for our fiducial
profile we adopt
\begin{equation} n(r)= n_{18} \left(\frac{r}{10^{18}
    {\rm cm}}\right)^{-1},
\label{eq:profile}
\end{equation}
where $n_{18}$ is the density at $r = 10^{18}$ cm.  In
$\S\ref{sec:profileComp}$ we show that our results for the jet radio
emission are not significantly altered in the case of a core-like
density profile (Appendix~\ref{app:core}) for a fixed value of
$n_{18}$ (Fig.~\ref{fig:cores}).

Combining equations \eqref{eq:nrs} and (\ref{eq:profile}), we obtain
\begin{equation}
  n_{18}\simeq 0.6 \left(\frac{r_s}{\rm pc}\right)^{0.8}
  \Mbh[,7]^{-0.28} \eta_{0.02} \, {\rm cm^{-3}},
  \label{eq:n18}
\end{equation}
%
Substituting the stagnation radius (eq.~\ref{eq:rs}) into this
expression gives
\begin{equation}
  n_{18}\simeq 0.3 \Mbh[,7]^{0.52} v_{500}^{-1.6} \eta_{0.02} \, {\rm
    cm^{-3}}.
\label{eq:n182}
\end{equation} 
%
As shown in Fig.~\ref{fig:profiles}, the gas density profile 
steepens outside the break radius $r_b$ of the stellar density
profile.  However, this will only impact the radio emission near its
maximum if $r_b$ lies inside of the Sedov radius, $r_{\rm sed}$
(eq.~\ref{eq:rdec}).  The lines in Fig.~\ref{fig:profiles} are colored
according to the combination of jet energy and CNM density $n_{18}$
which results in $r = r_{\rm sed}$ at each radius.  The measured break
radii of all but four of the \citet{Lauer+2007} galaxies exceed 10
parsecs, which greatly exceeds $r_{\rm sed}$ even in the case of a
very energetic jet ($E=4\times 10^{54}$ erg) in a low density CNM of
$n_{18} \sim 1$ cm$^{-3}$.  The presence of a nuclear star cluster
(NSC) in the galactic center could produce another break in the
stellar density profile near the outer edge of the cluster, which is
typically located at $r_{\rm nsc} \sim 1-5$ pc \citep{Georgiev+2014}.
But even in this case, only particular combinations of high $E$/low
$n_{\rm 18}$ result in $r_{\rm sed} < r_{\rm nsc}$.  We therefore
neglect the effects of an outer break in the stellar density profile
in our analysis.
% However, even a break radius of 1 pc resides inside of $r_{\rm
% sed}$, except for the combination of a powerful jet and relatively
% small CNM density, $n_{18}<30$ cm$^{-3}$.  Henceforth we neglect the
% effects of a break in the stellar density profile in our analysis.


\begin{figure}
\includegraphics[width=8cm]{sedov_radius.pdf}
\caption{\label{fig:profiles} Steady-state radial profiles of the CNM
  gas density, normalized to its value at $10^{18} {\rm cm}$,
  $n_{18}$. The profiles are calculated for a black hole mass of
  $10^{7} \,\Msun$ and a gas heating parameter of $v_w=600$ km
  s$^{-1}$.  Cusp and core stellar density profiles are shown with
  solid and dashed lines, respectively.  The line colors denote the
  ratio of isotropic equivalent jet energy to $n_{18}$ which results
  in $r = r_{\rm sed}$ at each radius.}
\end{figure}

\subsubsection{Allowed Density Range}
\label{sec:densAllowed}
We now estimate the allowed range in the normalization of the CNM gas
profile, $n_{18}$.  We assume that star formation occurs in two
bursts, an old burst of age comparable to the Hubble time $t_{\rm h} =
10^{10}$ yr, and a ``young'' burst of variable age $t_{\rm burst} \ll
t_{\rm h}$ which contributes a fraction $f_{\rm burst}$ of the stellar
mass. We assume a Salpeter IMF for both stellar populations.

For a sufficiently large burst of age $\lesssim$ 40 Myr, gas heating
is dominated by the energetic winds of massive star
winds.\footnote{Core-collapse SNe are also an important heating
  source.  In a young stellar population, the power from Core-collapse
  supernovae exceeds that from massive stellar winds after $\sim$6 Myr
  \citep{Voss+2009}. However, due to discreteness effects the heating
  from massive star winds will be more important on small scales.}  In
this case the mass return ($\eta$) and heating parameters ($v_w$) are
calculated as described in Appendix C of GSM15.\footnote{GSM15
  included terms to account for mass and energy injection from stars
  on the lower main sequence (in their equations C1 and C2). These are
  always sub-dominant and we exclude them for simplicity.}  Given
$\eta(t_{\rm burst},f_{\rm burst})$ and $v_w(t_{\rm burst},f_{\rm
  burst})$, we calculate $n_{18}$ following equation~\eqref{eq:n182}.

For an older stellar population, a few different sources contribute to
gas heating, including Type Ia Supernovae (SNe)\footnote{Unbound
  debris streams from TDEs potentially provide another source of
  heating localized in the galactic center
  (\citealt{Guillochon+2015a}), which we neglect.} and AGN feedback.
We focus on quiescent phases, during which SNe Ia dominate.  As
discussed in GSM15, SNe Ia clear out the gas external to a critical
radius, $r_{\rm Ia}$, where the interval between successive Ia SNe
equals the dynamical (gas inflow) timescale.  For an old stellar
population, $n_{18}$ is estimated by equating $r_{\rm Ia}$ with the
stagnation radius in equation~\eqref{eq:n18}.  The Ia radius is
calculated as described in GSM15 at times $t>300 \,{\rm Myr}$ after
star formation, and is taken to be constant for $t = 40-300$
Myr.\footnote{GSM15 incorrectly extrapolated the Ia rate valid at
  times $t>300 \,{\rm Myr}$ back to a time $t = 3$ Myr, which is
  unphysical as no white dwarfs would have formed by this time.
  Although its qualitative impact on our results is minimal, here we
  instead take the Ia rate to be 0 for $ t < 40$ Myr.}


Fig.~\ref{fig:param} shows how $n_{18}$ varies with the young
starburst properties, $f_{\rm burst}$ and $t_{\rm burst}$.  We find a
maximum density of $n_{18} \sim 2,000\, \Mbh[,7]^{0.5}$ cm$^{-3}$ is
achieved for a burst of age $t_{\rm burst} \sim 4$ Myr which forms
most of the stars in the nucleus ($f_{\rm burst} \sim 1$).  In this
case, both the energy and mass budgets of the CNM are dominated by
fast winds from massive stars.  Although a large gas density is
present immediately after a starburst, the density will decline with
the wind mass loss rate, approximately $\propto t^{-3}$, i.e. by an
order of magnitude within just a few Myr.
% Therefore, the gas density would decline by an order of magnitude
% from this maximum allowed value after just a few Myr.

By contrast, the lowest allowed density $\sim 0.03 \Mbh[,7]^{0.5}$
cm$^{-3}$ is achieved for a relatively modest burst of young stars
$t_{\rm burst} \approx 10^{6}$ Myr, which forms a fraction $f_{\rm
  burst} = 4\times 10^{-4}$ of the total stellar mass. In this case
the young massive stars provide a high heating rate, while the mass
injection rate is comparatively low and receives contributions from
both young and old stars.

Our procedure may underestimate the value of $n_{18}$ somewhat, as we
do not include the effects of discreteness on the assumed stellar
population.  In particular, we assume that massive stars provide a
spatially homogeneous heating source, even on small radial scales
where the number of massive stars present may be very small.  The
doubly hatched region in Fig.~\ref{fig:param} denotes the region where
less than one massive star ($\gsim 15 \Msun$) is on average present
inside of the nominal stagnation radius (eq.~\ref{eq:rs}).
Discreteness effects are thus important for relatively small bursts of
star formation, including the case described above which gives the
minimum $n_{18}$.  If we instead equate the stagnation radius to the
radius enclosing a single star of mass $\gsim 15 \Msun$, we find a
larger value of $n_{18}\sim 0.5 \Mbh[,7]^{-0.4}$ cm$^{-3}$.  The true
minimum density therefore likely lies closer to $0.5 \Mbh[,7]^{-0.4}$
cm$^{-3}$.

Finally, note that the host galaxies of most of the observed TDE
population show evidence for some star formation within the last Gyr
\citep{French+2016}.  In this region of parameter space corresponding
to the right side of Fig.~\ref{fig:param}, gas heating rate is
dominated by SN Ia and we expect $n_{18}\sim 10 \Mbh[,7]^{-0.4}-100
\Mbh[,7]^{-0.4}$ cm$^{-3}$.

In summary, the CNM densities of quiescent galaxies vary from
min($n_{18}) \sim 0.5 \Mbh[,7]^{-0.4}$ cm$^{-3}$
to max($n_{18})\sim 10^{3} \Mbh[,7]^{0.5}$ cm$^{-3}$, with a
characteristic value of $n_{18}\sim 10 \Mbh[,7]^{-0.4}-100
\Mbh[,7]^{-0.4}$ cm$^{-3}$ in the host galaxies of most observed TDEs.

\subsubsection{Mass drop-out from star formation?}

Our CNM model predicts the total gas density as sourced by
stellar winds, including both hot and cold phases.  For the first few
Myr after a starburst, the injected stellar wind material is hot
($T\gsim 10^{7}$ K) due to the thermalized wind kinetic energy.
At later times, SNe Ia provide intermittent heating, but the stellar wind
material that accumulates on small radial scales between successive SNe Ia
may be much cooler ($T \lesssim 10^{4}$ K), causing it to condense
into dense clumps.

The propagation of jets through a medium containing clumps, clouds or
stars has been studied in the context of AGNs (e.g.,
\citealt{WangWiita+2000, ChoiWiita+2007}) and microquasars (e.g.,
\citealt{Araudo+2009,Perucho+2012}). It was found that the presence of
these obstacles has an effect on the long-term jet stability, as well
as observational signatures at high energies. However, the situation
is different in the case of GRB and TDE afterglows, where the emission
is expected to be similar for a clumpy and a smooth medium with the
same average density
(e.g.~\citealt{Nakar&Granot2007,van-Eerten+2009,Mimica&Giannios2011}). The
reason is that the angular width of the outflow (especially of the
slow component) is much larger than in the case of AGNs and
microquasars. In fact, it is large enough to make the overall effect
of the presence of any inhomogeneities in the external medium
minor. An analogous effect is found in case of SN remnants sweeping a
clumpy medium \citep{Obergaulinger+2015}. We note that even the presence
of a star directly in the path of the faster core of a two-component
jet would probably not result in the disruption of the jet. 


On the other hand, a fraction of the cold gas will also condense into
stars.  To estimate the star formation rate, we assume that the CNM
self-regulates itself to a condition of marginal thermal stability,
which we define as the cooling time being ten times longer than the
dynamical timescale (\citealt{McCourt+2012}). For a stellar population
of age $\sim$1 Gyr, we find that for marginal thermal stability the
nuclear gas density would be a factor of a few lower than in our
standard estimates.


\begin{figure} 
  \includegraphics[width=8cm]{cnm_plot.pdf}
  \caption{\label{fig:param} Contours of $n_{18}$, the CNM density at
    $r = 10^{18}$ cm (blue lines), as a function of the stellar
    population in the galactic nucleus.  The star formation is
    parameterized assuming that a fraction $f_{\rm burst}$ of the
    stars form in a burst of age $t_{\rm burst}$, while the remaining
    stars formed a Hubble time ago.  We have assumed a black hole mass
    of $10^{7} \, \Msun$ and that both the young and old stars possess
    a cusp-like density profile, which produces a gas density profile
    $n \propto r^{-1}$.  Hatched areas indicate regions of parameter
    space where massive stars ($\gsim 15 \, \Msun$) dominate the gas
    heating rate, but less than one (doubly hatched) or less than ten
    (singly hatched) massive stars are present on average inside the
    nominal stagnation radius (eq.~\ref{eq:rs}).  In these regions
    discreteness effects not captured by our formalism are potentially
    important. The red line shows the approximate location of the
    Galactic Center in this parameter space (see text for details).}
\end{figure}


\subsubsection{Constraints from the Galactic Center}
\label{sec:empirical}

%\subsubsection{Galactic center} 
Due to its close proximity, it is possible to directly observe the gas
density distribution on parsec scales in the Galactic Center
(GC). \citet{Baganoff+2003} find that the hot, diffuse plasma within
10 arcseconds ($\sim 10^{18}$ cm) of Sgr A* has a root mean square
electron density of $\sim 26$ cm$^{-3}$, which implies a number
density of $n\sim 50$ cm$^{-3}$. 
% However, this estimate includes
% regions closer to the center, where the gas density is higher, so that
% the true number density at $10^{18}$ cm may be smaller by a factor of
% $\sim 2$.

In Fig.~\ref{fig:param} we show the range of two-burst star formation
models which produce heating and mass return parameters equal to those
derived from the full star formation history of the GC from
\citet{Pfuhl+2011} (their Fig.~14).  Our formalism gives values of
$n_{18}\sim 4-6$ cm$^{-3}$, which are too low compared to
observations.  Discrepency at this level is not surprising because our
model is spherically symmetric, while in reality many of the massive
stars in the GC are concentrated in two counter-rotating disks
\citep{Genzel+2003} with a possibly top heavy IMF \citep{Bartko+2010}.  The
disk stars extend from $\sim 10^{17}-10^{18}$ cm and inject $\sim
10^{-3} \Msun$ yr$^{-1}$ of stellar wind material, much more than the
$\sim 4 \times 10^{-5} \Msun$ yr$^{-1}$ expected for the global star
formation history, explaining the large density of hot gas.

In short, accurate modeling of the gas distribution in a particular
galactic nucleus, requires detailed knowledge of the distribution of
stars. Our goal here has been to bracket the range of possible nuclear gas
densities, by considering a broad range of stellar populations.

The Galactic Center also contains a cold circumnuclear ring
(e.g. \citealt{Becklin+1982}) with an opening angle of
$\sim$12$\pm3^{\circ}$ \citep{Lau+2013} and a spatially averaged
density of $\sim 10^{5}$ cm$^{-3}$ (although this varies by a few
orders of magnitude throughout the ring--see \citealt{Ferriere2012} and
references therein). Additionally, the volume from $\sim$0.4-2.5 pc
is filled with warm, ionized atomic gas with density of $\sim 900$
cm$^{-3}$ \citep{Ferriere2012}. This gas cannot be accounted for in
our model, and may originate from larger scale inflows or a disrupted
giant molecular cloud.


\begin{table}
\begin{threeparttable}
  \caption{\label{tab:jetParams} Parameters for on-axis jet simulations.}
  \begin{tabular*}{0.95\columnwidth}{lll}
\hline
& Fiducial value & Other values \\
\hline\hline
    Fast component ($\Gamma=10$) &  &  \\ 
    \hline
    $[\theta_{\rm min}$, $\theta_{\rm max}]$ & [0, 0.1] radians & \\
    $E_{\rm ISO}/10^{54}$ erg & 4  & 0.04, 0.4\\
    $E/10^{54}$ erg & 0.02 & \\
    \hline 
    Slow component ($\Gamma=2$)\\
\hline
    $[\theta_{\rm min}$, $\theta_{\rm max}]$ & [0.1, $\pi/2$] radians
    & \\
    $E_{\rm ISO}/10^{54}$ erg & $4.7$ & 0.047, 0.47 \\
    $E/10^{54}$  erg & $0.47$ & \\
    \hline
    Microphysical parameters\\
\hline
    $\epsilon_e$ & 0.1 & \\
    $\epsilon_b$ & 0.002 & \\
    $p$ & 2.3\\
    \hline 
    Nuclear gas density \\
\hline
    $n_{18}$/cm$^{-3}$ & 60 & 2, 11, 345, 2000
  \end{tabular*}
% \begin{tablenotes}
% \item $^{\dagger}$  Additional values of physical parameters we tried.
% \end{tablenotes}
\end{threeparttable}
\end{table}


\section{Synchrotron Radio Emission}

\label{sec:results}
\subsection{Numerical Set-Up}
\label{sec:numerical}
We calculate the synchrotron radio emission from the jet-CNM shock
interaction across the physically plausible range of nuclear gas
densities.  We perform both one- and two-dimensional (axisymmetric)
relativistic hydrodynamical simulations using the numerical code
MRGENESIS \citep{MimicaGianniosAloy2009}. MRGENESIS periodically
outputs snapshots with the state of the fluid in its numerical
grid. These snapshots are then used as an input to the radiative
transfer code SPEV \citep{Mimica+2009}. SPEV detects the forward shock
at the jet-CNM interface, accelerates the non-thermal electrons behind
the shock front, evolves the electron energy and spatial distribution
in time, and computes the non-thermal emission taking into account the
synchrotron self-absorption (interested readers can find many more
technical details see \citealt{Mimica+2016}). We use the same
numerical grid resolution as in \citet{Mimica+2015}.

For the jet angular structure, we adopt the preferred two-component
model for SwJ1644 from \citet{Mimica+2015}, corresponding to a fast,
inner core with Lorentz factor $\Gamma = 10$, surrounded by a slower,
$\Gamma=2$ outer sheath. A schematic depiction of the jet geometry is
shown in Fig.~\ref{fig:jetstruct}.  In our 2D simulations the fast
inner core spans an angular interval $0-0.1\ {\rm radians}$, while the
slow outer sheath extends from $0.1\ {\rm radians}$ to $0.5\ {\rm
  rad}$.  The time dependence of the jet kinetic luminosity is given
by (\citealt{Mimica+2015})
\begin{equation}\label{eq:lum}
L_{\rm j, ISO}(t) = L_{j,0}\max\left[1, (t/t_0)\right]^{-5/3},
\end{equation}
where $t_0 = 5\times 10^5$ s is the duration of peak jet power. This
is assumed to match that of the period of the most luminous X-ray
emission of SwJ1644.  Integrating equation~(\ref{eq:lum}) from $t = 0$
to $\infty$ gives the isotropic equivalent energy of the jet, $E_{\rm
  ISO}$, where $L_{j,0}=0.4\, E_{\rm ISO}/t_0$.  The ratio of the
beaming-corrected energy of the fast component is fixed to be 4\% of
that of the slow sheath. For the microphysical parameters
characterizing the fraction of the post-shock thermal energy placed
into relativistic electrons ($\epsilon_e$) and magnetic field
($\epsilon_B$), and the power-law slope of the electron energy
distribution $p$, we adopt the values from the best fit model in
\citet{Mimica+2015} (see Table~\ref{tab:jetParams}).

\begin{figure}

\includegraphics[width=8.5cm]{jetstruct.pdf}
\caption{\label{fig:jetstruct} Initial geometry of the jet used for
  our hydrodynamic simulations. We note that for 1D- two component jet
  models, we perform separate models for the inner fast core and for
  the outer sheath, which are later combined to provide the resulting
  emission. For our 1D simulation we take a slow component
  extending from 0-$\pi/2$ radians to account for the effects of jet
  spreading.}
\end{figure}

For our 1D simulations, we modify the geometry of the slow sheath to
better mimic the results of the 2D simulations.  In our 2D models the
sheath is injected within a relatively narrow angular interval;
however, at late stages of evolution the bow shock created by the
jet-CNM interaction spans a much larger angular range due to lateral
spreading.  To account for the slow component becoming almost
isotropic near peak emission in our 2D simulations \citep[bottom two
panels of Fig.~8 in][]{Mimica+2015}, we instead take the slow
component to extend from 0.1 to $\pi/2$ radians in our 1D models. We
keep the true energy of the slow component fixed at $5\times 10^{53}$
erg so that the isotropic equivalent energy of the slow component is a
factor of $[\cos(0.1)-\cos(0.5))/(\cos(0.1)-\cos(\pi/2))]\approx 0.12$
smaller than in the 2D simulations.

Figure \ref{fig:1D2DB} compares light curves calculated from this
modified 1D approach to the results of the full 2D simulations.
Despite the slow sheath being initially much broader in the 1D
simulations than in 2D, the resulting light curves agree surprisingly
well.  The agreement is particularly good at the highest densities
($n_{18}=2000$ cm$^{-3}$) because the slow component rapidly
isotropizes in 2D.  At lower densities ($n_{18}=60$ cm$^{-3}$), the
agreement is not as good, with the 1D simulations predicting
systematically higher luminosities by a factor of $\lesssim 2$.


\begin{figure}
\includegraphics[width=8cm]{1d_2d.pdf}
\caption{\label{fig:1D2DB} Comparison of light curves from 1D and 2D
  simulations for an on-axis observer ($\theta_{j} = 0$). We
  assume that the gas density $n\propto r^{-1}$.}
\end{figure}


\subsection{Analytic Estimates}
\label{sec:analyt}
The dependence of the synchrotron peak luminosity, peak time, and late
time luminosity power law slope on the ambient gas density and jet
parameters can be estimated analytically using a simple model for the
emission from a homogenous, shocked slab of gas behind a
self-similarly expanding blast wave (e.g., \citealt{Sari+98,
  Granot+02}).  The relevant results, as presented by
\citet{Leventis+2012}, are summarized in Appendix~\ref{app:analyt}.
The peak luminosity of the slow component of the jet can be estimated
from equation~(\ref{eq:peakLumGen}),
\begin{align}
\nu L_{\nu, p}&=\text{min}
\begin{dcases}
  2.7\times 10^{40} \left(\frac{E}{10^{54} {\rm ergs}}\right)^{0.59}
  \left(\frac{\epsilon_e}{0.1}\right)^{1.3}\times \\
  \left(\frac{\epsilon_b}{0.002}\right)^{0.825}\left(\frac{\nu_{\rm
        obs}}{5 {\rm GHz}}\right)^{0.35} n_{18}^{1.24}
  {\rm erg \, s^{-1}} & {\rm Opt.~Thin}\\\\
  1.1 \times 10^{42} \left(\frac{E}{10^{54} {\rm ergs}}\right)^{0.87}
  \left(\frac{\epsilon_e}{0.1}\right)^{0.61}\times\\
  \left(\frac{\epsilon_b}{0.002}\right)^{0.26}\left(\frac{\nu_{\rm
        obs}}{5 {\rm GHz}}\right)^{2.01} n_{18}^{-0.14} {\rm erg\,
    s^{-1}} & {\rm Opt.~Thick},
\end{dcases}
\label{eq:peakLum}
\end{align}
where we have adopted fiducial values for the power-law slope of the
gas density profile, $k=1$, and the electron energy distribution,
$p=2.3$.  The top and bottom lines apply, respectively, to the shocked
CNM being optically thin and optically thick at the deceleration time (as delineated by blue lines in Fig.~\ref{fig:diss}).

The peak luminosity in the optically thin case depends sensitively on
$n_{18}$, while in the optically thick regime the dependence on
density is much weaker.  The peak fluxes in equation
(\ref{eq:peakLum}) are normalized to match those derived from our
numerical results.
%From equation~\eqref{eq:tslope}, we expect the luminosity to
%decay a $L_\nu \propto t^{-1.5}$ at late times.

The time of maximum flux, for the same fiducial values ($k = 1$,
$p=2.3$), is given by equation (\ref{eq:tpeakGen}),
\begin{align}
t_p= \text{min}
\begin{dcases}
  500 E_{54}^{0.5} n_{18}^{-0.5} {\rm days} & {\rm Opt.~Thin}\\\\
  50 \left(\frac{E}{10^{54} {\rm ergs}}\right)^{0.32}
  \left(\frac{\epsilon_e}{0.1}\right)^{0.45}
  \left(\frac{\epsilon_b}{0.002}\right)^{0.37}\\
  \left(\frac{\nu_{\rm obs}}{5 {\rm GHz}}\right)^{-1.1} n_{18}^{0.4}
  {\rm days}& {\rm Opt.~Thick},
\end{dcases}
\label{eq:peakTime}
\end{align}
where again the normalizations are chosen to match our numerical
results. Note that for the optically thin case the peak time
is within a factor of two of the deceleration time
(eq.~\ref{eq:tdec}).

In general, more energetic jets produce emission which peaks later in
time.  However, the scaling of $t_p$ with $n_{18}$ is more
complicated: if the emitting region is optically thick at the
deceleration time, then the peak time increases with CNM density. In
this case the peak flux occurs when the self-absorption frequency
passes through the observing band, and this occurs later if the
nuclear gas density is higher. Otherwise, peak flux is achieved near
the deceleration time, which is a decreasing function of $n_{18}$
(eq.~\ref{eq:tdec}).  Fig.~\ref{fig:diss} shows the division between
the optically-thick and optically-thin regimes at 1 and 30 GHz in the
parameter space of jet energy and $n_{18}$.

\begin{figure}
\includegraphics[width=8cm]{diss.pdf}
\caption{\label{fig:diss} Contours of the fraction of the kinetic
  energy of the slow component of the jet ($\Gamma=2$) which is
  dissipated at the reverse shock in the parameter space of jet
  energy, $E_{\rm j}$, and CNM density, $n_{18}$.  The parameters of
  the suite of jet simulations presented in this paper are shown as
  red squares. The approximate location of SwJ1644 in the parameter
  space is also labeled.  Blue lines delineate the parameter space
  where the slow component of the jet is optically thin/thick at the
  deceleration time at 1 GHz (left line) and 30 GHz (right line).}
\end{figure}


\subsection{Numerical Light Curves}
\label{sec:numResults}
As summarized in Table~\ref{tab:jetParams} (and shown in
Fig~\ref{fig:diss}), we calculate light curves for a grid of on-axis
jet simulations for five different values of $n_{18}$ (2, 11, 60, 345,
and 2000 cm$^{-3}$) and three different values of the
(beaming-corrected) jet energy $E$ ($5\times 10^{51}$, $5\times
10^{52}$, $5\times 10^{53}$ erg).

The left panels of Fig.~\ref{fig:lightcurves} show example light
curves for different jet energies and nuclear gas densities. The peak
luminosity is roughly linearly proportional to the jet energy and is
virtually independent of the ambient density.  For high CNM densities
and low frequencies this is to be expected because the emission is
dominated by the slow component, which is optically thick at the
deceleration time. However, for high frequencies and small CNM
densities, the peak luminosity of the slow component falls off, as
shown by the lighter shaded lines in the right panels of
Fig.~\ref{fig:lightcurves}. Coincidentally, the fast component just
compensates for this decline, resulting in the total (fast + slow)
peak luminosity being weakly dependent on $n_{18}$ across the entire
parameter space.  A good approximation to this universal peak
luminosity is given by equation~\ref{eq:peakLum} for $n_{18}=2000$
cm$^{-3}$ in the optically-thick case.


Fig.~\ref{fig:lightcurves} also makes clear that the peak time
increases with the ambient gas density.  Across most of the parameter
space the peak occurs after the deceleration time, when the emitting
region transitions from optically thick to optically thin, as occurs
later for larger $n_{18}$. However, at high frequencies and low
densities the slow component is optically thin at the deceleration
time, and thus its peak time is a decreasing function of $n_{18}$. At
30 GHz, the slow component peaks later for $n_{18}$=2 cm$^{-3}$ than
for $n_{18}$=60 cm$^{-3}$.


\begin{figure*} 
  \includegraphics[width=8cm]{lightcurves.pdf}
  \includegraphics[width=8cm]{lightcurves_comp.pdf}
  \caption{\label{fig:lightcurves} \textit{Left:} Radio light curves
    as viewed on axis ($\theta_{\rm obs}=0$) for jet energies of
    $5\times 10^{53}$ erg ({\it darker-shaded lines}) and $5\times
    10^{51}$ erg ({\it lighter-shaded lines}), for values of n$_{18}=$
    2 (blue), 60 (red), and 2000 (green) cm$^{-3}$.  Solid lines show
    the result of 1D simulations, while 2D light curves are shown as
    dashed lines (when available).  A gas density profile of $n\propto
    r^{-1}$ is used for all of the light curves.  Radio upper limits
    and detections are shown as triangles and squares, respectively.
    The single upper limit in the top panel is for D3-13 at 1.4 GHz
    from \citet{Bower2011}.  Gray triangles and squares in the middle
    panel indicate upper limits and detections at 3.0 GHz from
    \citet{Bower+2013}, while black triangles indicate upper limits at
    5.0 GHz from \citet{van-Velzen+2013}.  The red triangle shows the
    6.1 GHz upper limit on PTF-09axc from \citet{Arcavi+2014}
    \textit{Right:} $5\times 10^{53}$ erg on-axis light curves from
    left column ({\it darker-shaded lines}) and corresponding slow
    component light curves ({\it lighter-shaded lines}).}
\end{figure*}

The numerical light curves are well fit with a broken power law (see
e.g.~\citealt{Leventis+2012}),
\begin{equation}
L_\nu (t) =\frac{L_{\nu, p}}{2^{-1/s}}
\left[\left(\frac{t}{t_p}\right)^{-s
    a_1}+\left(\frac{t}{t_p}\right)^{-s a_2}\right]^{-1/s}, 
\label{eq:lcAnal}\end{equation}
where $L_{\nu, p}$ and $t_p$ are the peak luminosity and time given by
equations~\eqref{eq:peakLum} and~\eqref{eq:peakTime}, respectively.
The parameter $s$ controls the sharpness of the transition between the
early-time power-law slope $a_1$ and the late-time slope $a_2$.
Fitting to the numerical light curves, we find that $s\sim 1.0$,
$a_1\sim 1.9$, and $a_2\sim -1.4$, the latter approximately agreeing
with the analytic estimate in equation~\eqref{eq:tslope}.
%%More precisely 1.94, -1.45, and 1.04 


Fig.~\ref{fig:onOff} compares the light curves for observers aligned
with the jet axis (on-axis) with those at an angle of 0.8 radians from
the jet axis (off-axis).  While the on- and off-axis light curves
agree well for $n_{18}=2000$ cm$^{-3}$, the off-axis luminosity for
$n_{18}=2$ cm$^{-3}$ is smaller by an order of magnitude at peak.
This is because the on-axis light curve for this density is dominated
by the fast component of the jet, which would not be visible for
significantly off-axis observers. However, we find that the late time
light curve is nearly independent of viewing angle.


\begin{figure}
\includegraphics[width=8cm]{on_off.pdf}
\caption{\label{fig:onOff} Comparison between on-axis (solid line) and
  off-axis (dashed line) light curves from our 2D simulations.  The
  off-axis light curves are calculated for an observer viewing angle
  of $\theta_{\rm obs}$=0.8.  We adopt a density profile of $n\propto
  r^{-1}$.}
\end{figure}

%\subsubsection{Effect of gas density profile}
\label{sec:profileComp}
The top panel of Fig.~\ref{fig:cores} shows 1D on-axis radio light
curves for our fiducial gas density profile, $n\propto r^{-1}$, and a
core galaxy profile (equation~\ref{eq:cores}), both with $n_{18}=2$
cm$^{-3}$.  The light curves differ by at most a factor of a few. The
core and cusp light curves are even closer at higher densities, and
virtually indistinguishable at $n_{18}=2000$ cm$^{-3}$. This is
because for larger ambient densities, the jet only samples small
radii, where the core and cusp profiles are similar (see
Fig.~\ref{fig:profiles}). It is only at lower densities, for which the
Sedov radius lies outside of the flattening of the core density
profile, that noticeable differences emerge.

The bottom panel of Fig.~\ref{fig:cores} compares the 1D on-axis light
curves for $n\propto r^{-1}$ and $n\propto r^{-1.5}$ gas density
profiles with $n_{18}=60$ cm$^{-3}$. For most times the light curves
agree well, which is perhaps not surprising because the density
normalization in the two models agrees at $10^{18}$ cm, which is close
to the Sedov radius for these density profiles. However, In 2D
hydrodynamical simulations, the jet structure for the $n\propto
r^{-1.5}$ profile is more prolate, which results in a much steeper
late time slope for the light curve (see Fig.~\ref{fig:prof2D}).  For the
radii of interest, we do not expect CNM density profiles steeper than
$r^{-1}$ (see sec.~\ref{sec:dProf}). Thus, we defer detailed study of
this effect to future work.


\begin{figure} 
  \includegraphics[width=8cm]{fig_cores.pdf}
  \includegraphics[width=8cm]{prof.pdf}
  \caption{\label{fig:cores} {\it Top:} Comparison between on-axis
    light curves for our fiducial $n\propto r^{-1}$ gas density
    profile corresponding to a cusp-like galaxy, and the core galaxy
    profile defined by \eqref{eq:cores} with $r_s=10^{18}$ cm. {\it
      Bottom:} Comparison between on-axis light curves calculated for
    $n\propto r^{-1}$ ({\it solid black}) and $n\propto r^{-1.5}$
    ({\it dashed red}) gas density profiles.}
\end{figure}

\begin{figure} 
  \includegraphics[width=8cm]{prof2D.pdf}
  \caption{\label{fig:prof2D} {\it Top:} Comparison between on-axis
    light curves calculated from 2D hydro simulations for $n\propto
    r^{-1}$ ({\it solid black}) and $n\propto r^{-1.5}$ ({\it dashed
      red}) gas density profiles.}
\end{figure}


\subsubsection{Reverse Shock Emission?}
Our calculations shown in Figs.~\ref{fig:1D2DB}
and~\ref{fig:lightcurves}-\ref{fig:cores} include only emission from
the forward shock (shocked CNM), while in principle the reverse shock
(shocked jet) also contributes to the radio light curve.

The fraction of the initial kinetic energy of the jet which is
dissipated by the reverse shock provides a first-order estimate of its
maximum contribution to the radio light curve.  Fig.~\ref{fig:diss}
shows contours of the fraction of the kinetic energy of the slow
component dissipated by the reverse shock as a function of the jet
energy and CNM density, $n_{18}$.  This is estimated by integrating
the shock evolution determined from the jump conditions (see
Appendix~\ref{sec:reverse} for details), approximating the jet as a
constant source of duration $t_0 = 5 \times 10^{5}$ s and Lorentz
factor $\Gamma = 2$.  The parameters defining our grid of numerical
solutions are shown in Fig.~\ref{fig:diss} as red squares.


Fig.~\ref{fig:diss} shows that for high ambient densities and/or low
energy jets, the reverse shock dissipates an order unity fraction of
the kinetic energy of the jet.  Even for our highest energy/lowest
density model ($n_{18}=2$ cm$^{-3}$ and $E_j=5\times 10^{53}$ erg) the
reverse shock will dissipate of order $~20\%$ of the jet energy.
Fig.~\ref{fig:reverse}, shows the 5 GHz light curve for this case,
separated into contributions from the forward and reverse shocks.  The
emission from the reverse shock is comparable to that from the forward
shock for the first month.  However, this somewhat overstates the
contribution of the reverse shock to the observed emission because the
latter is strongly attenuated by absorption from the front of the jet
below 10 GHz, which has not been included in the reverse shock light
curve in Fig.~\ref{fig:reverse}.  While the reverse shock dissipates
an even larger fraction of the jet energy for higher ambient density,
its emission will be even more heavily absorbed.  We conclude that the
reverse shock emission can be neglected for the high energy jets with
$E\gtrsim 10^{53}$ erg, consistent with the reverse shock not
contributing appreciably to SwJ1644 (\citealt{Metzger+2012}).

For low energy jets, we find that the jet is crushed at early times,
even for low values of $n_{18}$. In the case of very low power jets
the reverse shock structure is replaced by a number of recollimation
shocks (similar to the structure seen in
e.g. \citealt{Mimica+2009}). While this is potentially a very
interesting case since the emitting volume from recollimation shocks
can be larger than from a single reverse shock, because of a much more
complex structure we defer a more detailed study of the emission from
the reverse/recollimation shocks in the the low energy case to future
work.

As a final note of caution, even if the reverse shock dissipates most
of the bulk kinetic energy into thermal energy, the latter can be
converted back to kinetic energy through adiabatic expansion.
However, we expect that the re-expansion will be relatively isotropic
compared to the original jet, because the matter is first slowed to
mildly relativistic speeds.  The net result of a ultra-strong reverse
shock (due to a weak jet, and/or an unusually high CNM density) is
therefore likely to be the production of two quasi-spherical lobes on
either side of the black hole, centered about the deceleration radius
(\citealt{Giannios&Metzger2011}).


\begin{figure}
  \includegraphics[width=8cm]{reverse.pdf}
  \caption{\label{fig:reverse} Radio light curve from the forward
    shock (red line), reverse shock (blue), and the total light curve
    (black) for a jet of energy $5\times 10^{53}$ erg and CNM density
    $n\propto r^{-1}$ with $n_{18} = 2$ cm$^{-3}$. The reverse shock
    light curve excludes absorption from the front of the jet, which
    when included in the full calculation results in large attenuation
    of the emission, such that the total light curve is dominated by
    the forward shock.}
\end{figure}

\subsection{Parameter Space of Jet-CNM Interaction}
\label{sec:param}
The left column of Fig.~\ref{fig:jetContours} shows contours of the
peak luminosity (thick lines) as derived from our grid of numerical
on-axis models, covering the parameter space of jet energy $E$ and
density $n_{18}$.  Also shown with thin lines is the luminosity
arising from just the slow, wide angle component.  The fast, narrow
component of the jet dominates at high frequencies and low densities,
while the slow, wide component dominates for large $n_{18}$ and low
frequencies.  Remarkably, the total peak luminosity is nearly
independent of the ambient gas density; this is in part coincidental,
as the fast and slow peak fluxes individually vary across the
parameter space. However, this is not the case for an off-axis jet,
for which the peak luminosity would be comparable to the peak
luminosity of just the slow component, and thus would fall off for
smaller ambient densities for frequencies $\nu\gsim$ 1 GHz. 

The right column of Fig.~\ref{fig:jetContours} compares our numerical
results for the slow component to the analytic estimate given in
equation \eqref{eq:peakLum}.  For large $n_{18}$, the peak luminosity
scales with density, jet energy, and frequency roughly as expected in
the optically thick limit.  By contrast, for 30 GHz and low $n_{18}$,
the numerical results approach the optically thin limit.

The left column of Fig.~\ref{fig:ContoursTp} shows contours of the
time of peak flux in days, separately for the slow component (thin
lines) and the total light curve (thick lines).  Shown for comparison
in the panels in the right column is the peak time as estimated from
equation~\eqref{eq:peakTime} in the optically thick case, which
roughly reproduces the numerical results for high densities/low
frequencies. At 30 GHz, the peak time decreases with $n_{18}$ at small
values of the latter, because in this regime the jet is optically thin
prior to the deceleration time.


\begin{figure*}
  \includegraphics[width=16cm]{lp_contours_new.pdf}
  \caption{\label{fig:jetContours} {\it {Left:}} Thick lines show the
    peak radio luminosity in the parameter space of jet energy and
    ambient gas density at $10^{18}$ cm, calculated from the grid of
    on-axis jet simulations in Table~\ref{tab:jetParams}. Thin lines
    show contours of peak luminosity for the slow component light
    curve ($\S$~\ref{sec:numerical}). {\it Right:} Analytic estimate
    for the peak luminosity (dashed lines; eq.~\ref{eq:peakLum})
    compared to the numerical results for the slow component (solid
    lines).}
\end{figure*}

\begin{figure*}
  \includegraphics[width=16cm]{tp_contours_new.pdf}
  \caption{\label{fig:ContoursTp} {\it {Left:}} Thick lines show peak
    time in days in the parameter space of jet energy and ambient gas
    density at $10^{18}$ cm, calculated from the grid of on-axis jet
    simulations in Table~\ref{tab:jetParams}. Thin lines show contours
    of peak time for the slow component light curve
    (see~\ref{sec:numerical}). {\it Right:} Analytic scaling for the
    peak time ({\it dashed}, see equation~\ref{eq:peakTime}) compared
    to the numerical results for the slow component (solid)}
\end{figure*}

\subsubsection{Comparison with radio detections and upper limits.}
\label{sec:upLims}

Fig.~\ref{fig:lightcurves} compares our fiducial $5\times 10^{53}$ erg
on-axis jet model to radio detections and upper limits derived from
follow-up observations of TDE flares, as compiled in Table 1 of
\citealt{Mimica+2015}.  All of the 5 GHz light curves, corresponding
CNM densities, $n_{18}$, of 2, 60, and 2000 cm$^{-3}$, fall above the upper
limits. In agreement with the results of previous work, we conclude
that most TDEs discovered by their optical/UV or soft X-ray emission
do not produce jets as powerful as that responsible for SwJ1644
(\citealt{Bower+2013,van-Velzen+2013,Mimica+2015}), a result which is
now found to hold independent of the CNM environment.

The peak radio luminosity at frequencies $\lsim$ 1 GHz is weakly
dependent on the ambient gas density. Radio observations conducted
from several months to years after a tidal disruption flare, which
tightly constrain the peak flux of a putative jet, can therefore be
used to constrain the jet energy.  Equation~\eqref{eq:peakLum} shows
that an upper limit of $F_{\rm ul}$ on the flux density at 1 GHz of a
source at distance $d_L$ results in an upper limit on the jet energy
of
\begin{equation}
  E\lsim 4.3 \times 10^{49} \left(\frac{F_{\rm ul}}{50 \,\mu{\rm Jy}}\right)^{1.1}
  \left(\frac{d_L}{200 \,{\rm Mpc}}\right)^{2.3} {\rm erg},
\end{equation}
%
where we have taken $n_{18}=2000$ cm$^{-3}$ (but the constraint is not
overly sensitive to this choice). Radio measurements of the peak flux
following a TDE therefore serve as calorimeters of the total energy
released in a relativistic jet.

Even if the peak flux is missed, late-time measurements can still be
used to constrain the jet energy. In fact, with late time measurements
it is possible to place robust constraints on the energy of the
jet/outflow, even using higher frequency
measurements. Fig.~\ref{fig:econtours} compares our analytic fit to
the on-axis 5 GHz synchrotron light curve (eq.~\ref{eq:lcAnal}) for
different jet energies and existing radio upper limits.  The top panel
shows the light curves for a fixed value of $n_{18}$, but an
increase(decrease) in $n_{18}$ simply shifts the light curve to
later(earlier) times.  One implication of this behavior is that the
radio luminosity for an on-axis fixed energy cannot fall below the
minimum of the largest $n_{18}$ (2000 cm$^{-3}$) and smallest $n_{18}$
(0.5 cm$^{-3}$) light curves at any given time. The radio luminosity
for an off-axis jet propagating through low density environments would
be reduced compared to the on-axis case, but would still generally
fall above the expected luminosity for the same energy jet propagating
through a high density medium.


The bottom panel of Fig.~\ref{fig:econtours} shows, for a range of jet
energies, the minimum allowed synchrotron luminosity at a fixed time
(across all $n_{18}$) for comparison to the radio upper limits.  The
minimum curve which passes through each upper limit corresponds to the
maximum jet energy consistent with the upper limit.  Note that these
constraints apply to spherical outflows as well as collimated jets.



\begin{figure}
\includegraphics[width=8.5cm]{e_contours1.pdf}
\includegraphics[width=8.5cm]{e_contours2.pdf}
\caption{\label{fig:econtours} {\it Top:} Upper limits and analytic
  light curves at 5 GHz for different jet energies.  {\it Bottom:}
  Upper limits and minimum synchrotron luminosity across the allowed
  range of $n_{18}$ as a function of time for different jet energies.
  This is calculated as the minimum of the light curves calculated for
  the maximum allowed density of $n_{18} = 2000$ cm$^{-3}$ and the
  minimum allowed density of $n_{18}= 0.5$ cm$^{-3}$), as determined
  from our CNM modeling in $\S\ref{sec:densAllowed}$.}
\end{figure}

Fig.~\ref{fig:hist} shows a histogram of the maximum jet energies
consistent with the existing radio upper limits and detections of TDE
flares with radio follow-up (see also Table \ref{tab:enConstr}).  The
detected events include ASSASN-14li, SwJ1644, and SwJ2058.  For
ASSASN-14li and SwJ1644 the lightcurves are well sampled, and the
energy of the jet is relatively well constrained to be $\approx
10^{48}-10^{49}$ erg for ASSASN-14li (\citealt{van-Velzen+2015,
  Alexander+2015}) and $5\times 10^{53}$ erg for SwJ1644
(\citealt{Mimica+2015}).  For SwJ2058, we take the jet energy to be
$5\times 10^{53}$ erg, similar to its twin SwJ1644.  On one hand, the
observed radio luminosity of $\nu L_{\nu }\approx 10^{42}$ erg
s$^{-1}$ \citep{Cenko+2012} requires an outflow at least this
energetic.  On the other hand, the inferred energy of $5\times
10^{53}$ erg is comparable to the total rest mass energy available
from the disruption of a solar type star.

\begin{figure}
\includegraphics[width=8.5cm]{hist.pdf}
\caption{\label{fig:hist} Histogram of jet energies consistent with
  existing radio detections (ASSASN-14li,  SwJ1644, and
  SwJ2058) and upper limits (Table 1 of \citealt{Mimica+2015} and
  \citealt{Arcavi+2014}), as summarized in Table~\ref{tab:enConstr}.}
\end{figure}

\begin{table*}
\begin{threeparttable}
  \caption{\label{tab:enConstr} Inferred jet/outflow energies (and
    bounds) from radio detections and upper limits of optical/UV and
    soft X-ray TDE candidates. For each event detected in the radio
    there are multiple observations at different
    times/frequencies. Thus, we leave a dash in the time frequency,
    and luminosity columns and simply to refer to reference in column
    ``Ref.''}
\begin{tabular*}{1.5\columnwidth}{lllllll}
\hline
Source & $D_L$ & t & $\nu$ & $\nu L_{\nu}$ & Ref. & Energy\\
& (Mpc) & (yr) & (GHz) & ($10^{36}$ erg s$^{-1}$) & & (erg) \\
\hline
Detections \\
\hline
ASSASN-14li & 93 & - & - & - & 1 &  $10^{48}-10^{49}$\\
%CSS100217 & 700 & - & - & - & 2 & $1.7\times 10^{50}-6\times 10^{51}$\\
SwJ1644 & 1900 & - & - & - &  2  & $5\times 10^{53}$\\
SwJ2058 & 8400  & - & - & - & 3 & $5\times 10^{53}$\\ 
\hline 
Upper limits & \\
\hline
RXJ1624+7554 & 290 & 21.67 & 3.0 & 27 & 4 & $< 1.8 \times 10^{ 53 }$ \\
RXJ1242-1119 & 230 & 19.89 & 3.0 & 17 & 4 & $< 1.2 \times 10^{ 53 }$ \\
SDSSJ1323+48 & 410 & 8.61 & 3.0 & 100 & 4 & $< 1.8 \times 10^{ 53 }$ \\
SDSSJ1311-01 & 900 & 8.21 & 3.0 & 280 & 4 & $< 3.6 \times 10^{ 53 }$ \\
D1-9 & 1800 & 8.0 & 5.0 & 840 & 5 & $< 6.9 \times 10^{ 53 }$ \\
TDE1 & 660 & 5.4 & 5.0 & 130 & 5 & $< 1.1 \times 10^{ 53 }$ \\
D23H-1 & 930 & 4.8 & 5.0 & 210 & 5 & $< 1.4 \times 10^{ 53 }$ \\
PTF10iya & 1100 & 1.6 & 5.0 & 320 & 5 & $< 5.8 \times 10^{ 52 }$ \\
PS1-10jh & 840 & 0.71 & 5.0 & 320 & 5 & $< 2.4 \times 10^{ 52 }$\\
NGC5905 & 49 & 21.91 & 3.0 & 1.7 & 4 & $< 2.3 \times 10^{ 52 }$ \\
NGC5905 & 49 & 6.0 & 8.6 & 3.7 & 6 & $< 7.2 \times 10^{ 51 }$ \\
D3-13 & 2000 & 7.6 & 5.0 & 1000 & 5 & $< 7.4 \times 10^{ 53 }$ \\
D3-13 & 2000 & 1.8 & 1.4 & 1000 & 7 & $< 2.4 \times 10^{ 53 }$ \\
TDE2 & 1300 & 4.3 & 5.0 & 610 & 5 & $< 2.8 \times 10^{ 53 }$ \\
TDE2 & 1300 & 1.1 & 8.4 & 1700 & 8 & $< 1.2 \times 10^{ 53 }$ \\
SDSSJ1201+30 & 710 & 1.4 & 7.9 & 1100 & 9 & $< 1.1 \times 10^{ 53 }$ \\
PTF09axc & 550 & 5.0 & 3.5 & 700 & 10 & $< 4.0 \times 10^{ 53 }$ \\
PTF09axc & 550 & 5.0 & 6.1 & 550 & 10 & $< 2.8 \times 10^{ 53 }$\\
\end{tabular*}
\begin{tablenotes}
\item References: $(1)$ \citet{Alexander+2015, van-Velzen+2015}, $(2)$
  \citet{Berger+2012, Zauderer+2013} , $(3)$ \citet{Cenko+2012}, $(4)$
  \citet{Bower+2013}, $(5)$ \citet{van-Velzen+2013}, $(6)$
  \citet{Bade+1996, Komossa&Dahlem2001}, $(7)$
  \citet{Gezari+2008,Bower+2011}, $(8)$ \citet{van-Velzen+2011}, $(9)$
  \citet{Saxton+2012}, $(10)$ \citet{Arcavi+2014}. All upper limits
  are 5 $\sigma$. Luminosity distances are calculated using the
  identified host galaxy redshift and the best fitting Planck
  2013 cosmology ($\Omega_M=0.307$ and $H_0=67.8$ km s$^{-1}$
  Mpc$^{-1}$), as implemented in the Astropy cosmology package.
\end{tablenotes}
\end{threeparttable}
\end{table*}

\section{Summary and Conclusions}
\label{sec:conc}

We calculate radio light curves for tidal disruption event jets
propagating through different circumnuclear (CNM) gas densities. We
simulate the jet propagation using both 1D and 2D hydrodynamic
simulations. We then post-process these to produce radio synchrotron
light curves. To isolate the effects of the density profile and jet
energy we take a fixed two component jet model from
\citet{Mimica+2015}, which produces a good fit to the observed radio
data in SwJ1644. We consider a broad range of gas densities motivated
by analytic estimates of stellar wind mass injection. Our conclusions
are summarized as follows.

\begin{enumerate}
\item We estimate the nuclear gas densities expected from injection of
  stellar wind material for different star formation histories. We
  find that that range of gas densities at 10$^{18}$ cm is $n_{18}
  \sim 0.5 \Mbh[,7]^{-0.4} - 2000 \Mbh[,7]^{0.5}$ cm$^{-3}$, with an
  expected $n_{18}$ of order 10 cm$^{-3}$ for host galaxies of most
  observed TDEs.

\item The slope of the gas density profile depends on the slope of the
  stellar density profile. We expect a typical TDE host to have cuspy
  stellar density profile inside of a few pc, with $\rho_\star
  \propto r^{-1.7}$. This translates into a gas density profile $n
  \propto r^{-1}$. The radio light curve of a TDE jet is most
  sensitive to the density at the deceleration/Sedov radius (where it
  has swept up its mass in CNM gas). The light curve will be
  insensitive to changes in slope for fixed density at the
  deceleration radius.

\item We take our two component jet model and run it through a range
  of different density profiles. Motivated by the above results for
  the expected range of gas densities we take the density at $10^{18}$
  cm, $n_{18}$, to be 2, 11, 60, 345, or 2000 cm$^{-3}$. We find
  bright radio emission at a few GHz across this entire range of
  densities, with the peak luminosity only weakly dependent on the
  chosen value of $n_{18}$ for frequencies $\lsim$1 GHz. The time of
  the peak luminosity depends, however, on the density and can be as
  early as months and as late as one decade after the TDE. By
  comparing radio detections and upper limits from a set optical/UV
  and soft X-ray selected TDE, we show that most of these sources
  cannot have jets as powerful as SwJ1644. Prompt follow-up in the
  radio band, as well as regular monitoring, would provide tighter
  constraints on the existence of TDE jets. Radio afterglows can serve
  as calorimeters for off-axis jets launched by TDEs, and future
  observational efforts that capture the peak radio flux in thermally
  detected TDEs will add to the diversity of jet energies observed in
  TDE flares.  The broad range of energies (both detections and upper
  limits) already seen in TDE jets presents an interesting puzzle for
  theoretical models of jet launching.
\end{enumerate}


\section*{Acknowledgments}
We acknowledge helpful conversations with Jerry Ostriker, Luca Ciotti,
James Guillochon, and Yue Shen. PM and MAA acknowledge financial
support from the European Research Council (ERC) through the Starting
Independent Researcher Grant CAMAP-259276, and the partial support of
grants AYA2013-40979-P, AYA2015-66899-C2-1-P and
PROMETEO-II-2014-069. We thankfully acknowledge the computer
resources, technical expertise and assistance provided by the Servei
de Inform\`atica of the University of Valencia and Columbia
University's Yeti Computer Cluster. This research made use of Astropy,
a community-developed core Python package for Astronomy (Astropy
Collaboration, 2013).

\appendix
\section{Core Profile}
\label{app:core}
Fig.~\ref{fig:cores} compares the results of radio light curves from
jets propagating in core and cusp like gas density profiles
(Fig.~\ref{fig:profiles}).  We use the following analytic expression
to approximate the core galaxy CNM profile in Fig.~\ref{fig:profiles}
\begin{align}
\begin{cases}
n=n(r_s) k(x) & 0.4 \leq x\leq 2.0\\
n = 2.0 n(r_s) (x/0.4)^{-0.95} & x < 0.4\\
n = 0.75 n(r_s) (x/2.0)^{-0.26} & x>2,\\
\end{cases}
\label{eq:cores}
\end{align}
where
\begin{align}
  &x=r/r_s\\\nonumber
  &k(x)=\frac{45}{19} \frac{1}{x^{3/2}} \frac{1-x^{1.9}}{9-19
      x\frac{x^{0.9}-1}{x^{1.9}-1}}
\end{align}
To isolate the effects of the shape of the density profile, we
consider a core density profile with a stagnation radius $r_s=10^{18}$
cm and density normalization $n_{18}=2000$ cm$^{-3}$ which match those
of our high density cusp model.

\section{Peak Luminosities and times}
\label{app:analyt}
\citet{Leventis+2012} present analytic scaling relations for the
synchrotron flux of a spherical blast wave propagating through a
medium with a power law density profile, $n\propto r^{-k}$.  Here we
make use of their results to estimate the peak radio flux of the slow
(sheath) component of the jet.

During the late-time, Newtonian stage of the jet evolution,
synchrotron self absorption is important for frequencies below
\begin{align}
  \nu_{\rm sa}=&C_1(p, k) E_{54}^{\frac{10 p-k p -6 k}{2 (4+p) (5-k)}}
  n_{18}^{\frac{30 - 5 p}{2 (4 + p) (5 - k)}}
  \epsilon_e^{\frac{2 (p-1)}{4+p}} \epsilon_b^{\frac{p+2}{2 (4+p)}}\nonumber\\
  &t^{\frac{10 - 8 k - 15 p + 4 k p}{(4 + p) (5 - k)}},
\label{eq:nuSa} 
\end{align}
%
where $E = 10^{54}E_{54}$ erg is the blast wave energy and $C_1(p, k)$
is a normalization factor.  Equation~\eqref{eq:nuSa} is valid only if
self-absorption frequency is greater than the synchrotron peak
frequency,
\begin{equation}
\nu_m=C_2(p, k) E_{54}^{\frac{10-k}{2 (5-k)}} n_{18}^{-\frac{5}{2
    (5-k)}}  \epsilon_e^2  \epsilon_b^{1/2}  t^{\frac{4 k-15}{5-k}}.
\label{eq:num}
\end{equation}
%
% The observing frequencies considered here ($\geq 1$ GHz), are also
% greater than $\nu_m$.
The light curve will peak at the deceleration time (eq.~\ref{eq:tdec})
in case the emitting region is optically thin then. Otherwise, it will
occur after the deceleration time, when the self-absorption frequency
crosses through the observing band. The peak time for these two cases
is

\begin{align}
t_{\rm p} \approx
\begin{dcases}
  0.5 \left(50 (3-k) \, E_{54}\right)^{1/(3-k)} \Gamma^{(2k - 8)/(3-k)}
  n_{18}^{-1/(3-k)} {\rm yr} & {\rm Opt.\, Thin}\\\\
  C_1(p, k)^{-\frac{(5 - k) (4 + p)}{10 - 8 k - 15 p + 4 k
      p}}E_{54}^{-\frac{-k p-6 k+10 p}{2 (4 k p-8 k-15 p+10)}}\\
  \times n_{18}^{-\frac{30-5 p}{2 (4 k p-8 k-15 p+10)}} \nu_{\rm
    obs}^{\frac{(5-k) (p+4)}{4 k p-8 k-15 p+10}}\\
  \times \epsilon_b^{-\frac{(5-k) (p+2)}{2 (4 k p-8 k-15 p+10)}}
  \epsilon_e^{-\frac{2 (5-k) (p-1)}{4 k p-8 k-15 p+10}} & {\rm Opt.\,
    Thick},
\end{dcases}
\label{eq:tpeakGen}
\end{align}
%
where $\Gamma$ is the initial jet Lorentz factor. 

The unabsorbed flux at the peak frequency is given by
\begin{align}
  F_{\nu_m} =  C_3(p, k) E_{54}^{\frac{8-3 k}{2 (5-k)}}
  n_{18}^{\frac{7}{2 (5-k)}} \epsilon_b^{1/2} t^{\frac{3-2 k}{5-k}}
\label{eq:Fnum}
\end{align}
%
Extrapolating to the observer frequency gives 
\begin{align}
  \nu_{\rm obs} F_{\rm p} (\nu_{\rm obs}) &= \nu_{\rm obs}   F_{\nu_m}
  \left(\frac{\nu_{\rm obs}}{\nu_m}\right)^{-(p-1)/2}.
  \label{eq:Fpeak1}
\end{align}
%
Combining equations~\eqref{eq:num}, ~\eqref{eq:tpeakGen}, ~\eqref{eq:Fnum},
and~\eqref{eq:Fpeak1}, we find
\begin{align}
  \nu_{\rm obs} F_{\rm p} (\nu_{\rm obs}) \propto
  \begin{dcases}
    E_{54}^{\frac{k (p+5)-12}{4 (k-3)}} n_{18}^{-\frac{3 (p+1)}{4
        (k-3)}} \nu_{\rm obs}^{\frac{3-p}{2}}
    \epsilon_b^{\frac{p+1}{4}} \epsilon_e^{p-1} & {\rm Opt.\, Thin}\\\\
    E_{54}^{\frac{k(-(p-2))-10 p+3}{4 k (p-2)-15 p+10}} \\ \times
    n_{18}^{\frac{11 (p-2)}{4 k (p-2)-15 p+10}} \nu_{\rm
      obs}^{\frac{14 k (p-2)-47 p+57}{4 k (p-2)-15 p+10}} \\ \times
    \epsilon_b^{\frac{k (-(p-2))+p-8}{4 k(p-2)-15 p+10}}
    \epsilon_e^{-\frac{11 (p-1)}{4 k (p-2)-15p+10}} & {\rm Opt.\,
    Thick}
  \end{dcases}
  \label{eq:peakLumGen}
\end{align}

After peak, we expect that the flux scales as 

\begin{equation}
F_{\nu}\propto t^{\frac{21-8k-15p+4kp}{10-2k}}.
\label{eq:tslope}
\end{equation}



\section{Reverse shock}
\label{sec:reverse}
Here we estimate the fraction of the kinetic energy of the jet that is
dissipated by the reverse shock, as opposed to the forward shock whose
contribution is the focus of this paper.  From continuity, the
comoving density of a relativistic jet is given by
(e.g.~\citealt{Uhm&Beloborodov2007})
 \begin{align}
   n_{\rm j} =  \frac{L_{\rm j, iso}}{4 \pi r^{2}\Gamma_{\rm
       j}^{2}c^{3}m_p(1 + r \dot{\Gamma}/c\Gamma^{3})}
   \approx  \frac{L_{\rm j, iso}}{4 \pi r^{2}\Gamma^{2}c^{3}m_p},
\end{align}
%
where $L_{\rm j, iso}$ is the isotropic equivalent luminosity.  The
second term in the denominator can be neglected if the jet Lorentz
factor changes slowly ($\dot{\Gamma}_{\rm j} \ll c\Gamma^{3}/r$), a
condition which is satisfied at radii $r < r_{\rm dec}$ if $\Gamma$
changes slowly on a timescale $\gtrsim t_{\rm 0}$, where $t_{\rm 0}$
is the jet duration.

The common Lorentz factor of the shocked CNM and the shocked jet can
be estimated using the relativistic shock jump condition and pressure
equality between the forward and reverse shocks.  In the
ultra-relativistic limit this gives,
\begin{equation}
\Gamma_{\rm sh} \underset{\Gamma_{\rm sh} \gg 1}= \Gamma\left[1 +
  2\Gamma f^{-1/2}\right]^{-1/2},
\label{eq:gammaShSim}
\end{equation}
where
\begin{equation}
  f\approx 40\,  L_{\rm j,48} n_{18}^{-1} \Gamma_{10}^{-2} \, \left(\frac{r}{10^{18} {\rm
        cm}}\right)^{-1} 
\end{equation}
is the ratio of the density of the jet to that of the
CNM. Equation~\eqref{eq:gammaShSim} is inaccurate for mildly relativistic or
non-relativistic flows, in which case we apply the more general
expression for $\Gamma_{\rm sh}$ given by \citet{Beloborodov&Uhm2006}
(their eq.~3, see also \citealt{Mimica&Aloy2010})
\begin{equation}
\frac{\Gamma^2-1}{\Gamma_{43}^2-1} f^{-1}=1 ,
\end{equation}
where
\begin{equation}
  \Gamma_{43}=\Gamma \Gamma_{\rm sh} \left(1-\beta_{\rm sh} \beta_j\right),
\label{eq:gammaShGen}
\end{equation}
is the Lorentz of shocked jet in the frame of the unshocked jet. In
the lab frame the reverse shock moves with a velocity
\begin{equation}
\beta_{\rm rs}=\frac{\beta_{\rm sh}(f)-\beta_{43}(f)/3}{1-\beta_{\rm
    sh}(f) \beta_{43}(f)/3}.
\label{eq:betars}
\end{equation} 
%
Equations~\eqref{eq:gammaShGen} and ~\eqref{eq:betars} can be used to
determine the radius of the shocks when the reverse shock crosses the
trailing edge of the jet and the value of $\Gamma_{\rm sh}$ at this
time.  The latter allows us to calculate what fraction of the initial
kinetic energy of the jet is dissipated at the reverse shock, instead
of being transferred to the shocked external medium via the forward
shock.



\clearpage
  \footnotesize{
    \bibliographystyle{mnras}
    \bibliography{master}
  }

\end{document}
