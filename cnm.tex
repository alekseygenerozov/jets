\documentclass[usenatbib,fleqn]{mnras}

\makeatletter
\newlength{\abovecaptionskip}%
\setlength{\abovecaptionskip}{10\p@}
\makeatother


\usepackage{threeparttable}
 

\usepackage{amsmath,amssymb}
\usepackage{mathrsfs}
\usepackage{graphicx}
\usepackage{epstopdf}
%\usepackage{hyperref}
\epstopdfsetup{outdir=./figures/}
\graphicspath{{./figures/}}
\usepackage{url}
%\usepackage{aas_macros}

\newcommand\lsim{\mathrel{\rlap{\lower4pt\hbox{\hskip1pt$\sim$}}
    \raise1pt\hbox{$<$}}}
\newcommand\gsim{\mathrel{\rlap{\lower4pt\hbox{\hskip1pt$\sim$}}
    \raise1pt\hbox{$>$}}}


\newcommand       \be          {\begin{eqnarray}}
\newcommand       \ee          {\end{eqnarray}}
\newcommand{\Mbh}[1][]{M_{\bullet#1}}
\newcommand{\Menc}{M_{\rm enc}}
\renewcommand{\th}{t_h}
\newcommand{\Msun}{{\rm M_\odot}}
\newcommand{\pyear}{{\rm yr}^{-1}}
\newcommand{\rs}{r_s}

% write title (with email and institute)
\title{The influence of the cicumnuclear environment on the radio emission from TDE jets}
\author[Generozov et al.]{ A. Generozov$^{1}$, P. Mimica$^{2}$,
  B. D. Metzger$^{1}$,
  D. Giannios$^{3}$, 
  N. Stone$^{1}$,
  M.A. Aloy$^{2}$ 
  \\
  $^{1}$Columbia Astrophysics Laboratory, Columbia University, 550 West 120th Street, New York, NY 10027\\
  $^{2}$Departamento de Astronomia y Astrofisica, Universidad de Valencia, E-46100 Burjassot, Spain\\
  $^{3}$Department of Physics and Astronomy, Purdue University, 525
  Northwestern Avenue, West Lafayette, IN 47907, USA}

\begin{document}
\maketitle
\begin{abstract}
  There are now dozens of candidates for tidal disruptions of stars by
  supermassive black holes (TDEs) at optical and x-ray wavelengths. A
  small fraction of these events, (e.g. {\it Swift} J1644+57) have radio
  synchrotron emission consistent with a powerful, relativistic jet
  shocking surrounding gas. The low detection rate of such events may
  mean that powerful jets are intrinsically rare in TDEs. However, it
  could also mean that typical nuclear gas densities are unfavorable
  for producing observable radio emission. We explore this
  possibility, constraining the range of gas densities which could be
  encountered by a TDE jet. We then calculate radio light curves for
  jets across the expected range of gas densities ($\sim$ 0.5-2000
  cm$^{-3}$ at $10^{18}$ cm). We find bright radio transients across
  this range of density profiles. Existing radio upper limits are
  often taken decades after the observed flare, well after the
  expected peak of the light curve for low density
  environments. Nonetheless they exclude powerful, {\it Swift} J1644+57-like
  jets for CNM densities less than 60 cm$^{-3}$ at $10^{18}$ cm. More
  stringent constraints would be possible with prompt follow-up of
  tidal disruption event candidates and would inform our understanding
  of the conditions necessary to launch jets.
\end{abstract}
\section{Introduction}
\label{sec:intro}
When a star in a galactic nucleus is deflected too close to the
central supermassive black hole (BH), it can be torn apart by tidal
forces.  During this tidal disruption event (TDE), roughly half of the
stellar debris remains bound to the BH, while the other half is flung
outwards and unbound from the system.  The bound material, following a
potentially complex process of debris circularization
(\citealt{Guillochon+2013},\citealt{Hayasaki+2013},\citealt{Hayasaki+2015},\citealt{Shiokawa+2015},\citealt{Bonnerot+2015}),
accretes onto the BH, creating a luminous flare lasting months to
years \citep{Hills1975, Carter+1982, Rees1988}.

Many TDE flares have now been identified at UV/optical
\citep{Gezari+2008, Gezari+2009, van-Velzen+2011, Gezari+2012,
  Arcavi+2014, Chornock+2014, Holoien+2014, Vinko+2015, Holoien+2016}
and soft x-ray wavelengths \citep{Bade+1996, Grupe+1999,
  Komossa&Greiner1999, Greiner+2000, Esquej+2007, Maksym+2010,
  Saxton+2012}. Beginning with the discovery of {\it Swift} J1644+57
(henceforward Sw1644) in 2011, three additional TDEs have been
discovered by their hard x-ray emission (\citealt{Bloom+2011,
  Levan+2011, Burrows+2011, Zauderer+2011, Cenko+2012, Pasham+2015,
  Brown+2015}) {\bf Mostly same set of refs for optical/soft x-ray as
  Holoien+2016, with addition of Greiner+2000, and exclusion of
  Donley+2002 (who are looking at statistics of x-ray outbursts wo
  identifying a specific event)}.  Unlike the optical/UV and soft
X-ray TDEs, these events are the result of emission from a transient
relativistic jet beamed along our line of sight, similar to the blazar
geometry of active galactic nuclei (AGN).  In addition to their highly
variable X-ray emission, which likely originates from the base of the
jet, these events are characterized by radio synchrotron emission.
The latter, more slowly evolving, is powered by shocks formed at the
interface between the jet and surrounding circumnuclear medium (CNM)
\citep{Bloom+2011,Giannios&Metzger2011,Metzger+2012,Mimica+2015} {\bf
  other refs? did not see others in Mimica+2015}, analagous to the
afterglow of a gamma-ray burst {\bf AG change thermal to optical/UV
  and soft X-ray: X-ray emission may not be thermal (often times
  difficult to tell given spectral data.)}.

Although a handful of jetted TDE flares have been observed, their
volumetric rate is a very small fraction $\sim 10^{-5}-10^{-4}$ of the
rate of the observed TDE flare rate (e.g., \citealt{Burrows+2011},
\citealt{Brown+2015}), and an even smaller fraction of the
theoretically predicted TDE rate (\citealt{Stone&Metzger2016}).  One
potential explanation for this discrepancy is that the majority of
TDEs produce powerful jets, but their hard X-ray emission is
relativistically beamed into a small angle $\theta_{\rm b}$, making
them visible to only a small fraction of observers.  However, the
inferred beaming fraction $f_b \approx \theta_{b}^{2}/2 \sim
10^{-5}-10^{-4}$ would require $\theta_{\rm b} \sim 0.01$ and hence a
jet with a bulk Lorentz factor of $\Gamma \gtrsim 1/\theta_j \sim
100$, much higher than typically inferred for AGN jets or inferred for
Sw1644 (\citealt{Metzger+2012}).

The low detection rate of hard X-ray TDEs may instead indicate that
jets are intrinsically rare in these events, or that the conditions of the
surrounding environment are unfavorable for producing bright emission.
Jets could be rare if they require, for instance, a highly
super-Eddington accretion rate (\citealt{De-Colle+2012}), a TDE from a
deeply plunging stellar orbit (\citealt{Metzger&Stone2015}), or a
particularly strong magnetic flux threading the star
(\citealt{Tchekhovskoy+2014,Kelley+2014}).  Alternatively, jet formation
or its X-ray emission could be suppressed if the disk undergoes
Lens-Thirring precession due to a misalignment between the angular
momentum of the BH and that of the disrupted star
(\citealt{Stone&Loeb2012}).  In the latter case, however, even a `dirty'
jet could still be produced, which emits luminous non-thermal radio
synchrotron emission as it interacts the surrounding CNM.  The radio
emission from an off-axis is predicted to be relatively isotropic
\citep{Giannios&Metzger2011,Mimica+2015}.

{\bf fix} Observational campaigns have been conducted to follow-up
optical/UV and soft X-ray TDE flares at radio wavelengths on
timescales of months to decades after the outburst
(\citealt{Bower+2013}; \citealt{van-Velzen+2013}; see Table 1 of
\citet{Mimica+2015} for a compilation). Neither of these campaigns
found any radio afterglows definitively associated with the host
galaxies of strong TDE candidates.\footnote{There were radio
  detections for two ROSAT flares: RX J1420.4+5334 and IC
  3599. However, for RX J1420.4+5334 the radio emission was observed
  in a different galaxy than was originally associated with the flare.
  IC 3599 has shown multiple outbursts in the calling into question
  whether it is a true TDE at all \citep{Campana+2015}.}
\citet{Bower+2013} and \citet{van-Velzen+2013} use a simple model for
the radio emission as a Sedov blast wave, to conclude that $\lesssim
10\%$ of TDEs produce jetted TDE emission at a level similar to that
produced in {\it Swift} J1644+57.  \citet{Mimica+2015} used 2D
hydrodynamical models, coupled with a synchrotron radiation transport
calculation, to model the on-axis emission from {\it Swift}
J1644+57. They then extended this model to off-axis viewing angles,
and concluded that most previous TDEs should have been detected if
their jets were as powerful.

Previous works (\citet{Bower+2013}; \citet{van-Velzen+2013};
\citealt{Mimica+2015}) have generally assumed that all TDEs occur in a
similar environment as {\it Swift} J1644+57.  However, in general the
gas density in a galactic nucleus depends sensitively on the sources
of gas from stellar winds, and the sources of gas heating
(\citealt{Quataert2004,Generozov+2015}).  In a gamma-ray burst, the
environment the jet emerges from is relatively well-understood to be
the wind of the massive progenitor star, or the ISM of the host
galaxy.  However, the density of the CNM encountered by a TDE jet
could in principle be orders of magnitude higher.

In this paper we address the range of gas densities encountered by
jetted TDEs and estimate their synchrotron emission by means of
analytic calculations and numerical simulations.  If we find radio
emission should be detected for all plausible CNM densities, then we
are lead to the conclusion that most TDEs do not launch Sw1644-like
jets.  This result would have significant implications for the physics
of jet launching in TDEs and other accretion flows. 

In $\S\ref{sec:cnm}$ we use the formalism developed in
\citet{Generozov+2015} (hereafter GSM15) to calculate the CNM profiles
encountered by TDE jets for different assumptions about the stellar
population in the galactic nucleus.  A younger stellar population
produces significant wind mass loss from O and B stars. In contrast,
the rate of wind mass loss from an older population, which lacks these
massive stars, could be a $\sim$couple orders of magnitude smaller.
We show that the requirement of a physical stellar population limits
the gas density on a scale of $10^{18}$ cm to a range of $n_{18} \sim
0.5-2000$ cm$^{-3}$. We also show that the measured distribution of
accretion rates of low mass SMBHs in the local universe suggests that
the gas density in most galactic nuclei lie within a similar range.

As a second component of this paper, in $\S\ref{sec:jet}$ we explore
the expected synchrotron emission from {\it Swift} J1644+57-like
across the allowed range of CNM conditions.  We start in
$\S\ref{sec:analytic}$ with analytic conditions.  Then in
$\S\ref{sec:numerical}$ use both 1D and 2D hydrodynamic models to
simulate the jet propagating through the CNM and compare the results
in $\S\ref{sec:2D}$. In $\S\ref{sec:results}$ we show radio light
curves from our 1D models for a wide range of CNM densities, to
illustrate qualitatively how much varying the CNM density by itself
could change the radio light curve.  We summarize our conclusions in
$\S\ref{sec:conc}$.

\section{Range of CNM Densities}
\label{sec:cnm}

In this section we place constraints on the radial profile of the gas
density in galactic nuclei. First, we determine the possible range of
densities resulting from mass injection by stellar winds. Then we use
empirical dis

\subsection{Analytic Constraints on CNM Density}

\subsubsection{Dynamical Model}
\label{sec:model}

The dominant source of gas in the CNM of quiescent galaxies is winds
from stars in the galactic nucleus. We model the hot phase of the ISM
using the 1D spherical hydrodynamic equation with mass and energy
injection from stellar winds (e.g. \citealt{Holzer+1970};
\citealt{Quataert2004})
\begin{align}
  &\frac{\partial \rho}{\partial t}+\frac{1}{r^2}\frac{\partial}{\partial r}\left(\rho r^2 v\right)=q \label{eq:drhodt}\\
  &\rho \left(\frac{\partial v}{\partial t} + v\frac{\partial
      v}{\partial r}\right) =-\frac{\partial p}{\partial r}- \rho\frac{GM_{\rm enc}}{r^{2}} -q v \label{eq:dvdt}\\
  &\rho T\left(\frac{\partial s}{\partial t} + v\frac{\partial
      s}{\partial
      r}\right)=q\left[\frac{v^2}{2}+\frac{v_w^2}{2}-\frac{\gamma}{\gamma-1}
    \frac{p}{\rho} \right] ,
\label{eq:model}
\end{align}
where $\rho = \mu m_p n$, $v$, $p$, and $s$ are the density, velocity,
pressure (assumed to be an ideal gas with $\mu = 0.62$), and specific
entropy, respectively.  The enclosed mass $\Menc = M_{\bullet} +
M_{\star}$ includes both the black hole mass $M_{\bullet}$ and
enclosed stellar mass $M_{\star} \propto \int \rho_{\star}r^{2}dr$. At
the sphere of influence, $r_{\rm inf}$, $M_{\star}(r_{\rm inf})=\Mbh$.
We take $r_{\rm inf}=3.5 \Mbh[,7]^{0.6}$, where $\Mbh[,7]=\Mbh/10^7
\Msun$.

The mass source term $q =\eta \rho_\star/\th$ is the mass injection
rate per unit volume per unit time, where $\rho_\star$ is the stellar
density. The energy source term $\propto v_w^{2}$, parameterizes the
heating rate of the gas, as physically results from stellar wind
kinetic energy, supernovae, and black hole feedback.

\citet{Generozov+2015} present analytic approximations for the
densities and temperatures of steady state solutions to
equation~\eqref{eq:model}. We apply these results across the
physically allowed range heating rates ($v_w$) and mass injection
rates ($\eta$), and obtain the corresponding range of gas densities

\subsubsection{Stellar density profiles}
We assume a broken power law for the stellar density profile, $\rho_{\star}$.
This is motivated by \citet{Lauer+2007}, who use {\it
  Hubble Space Telescope} WFPC2 imaging to measure the radial surface
brightness profiles for hundreds of nearby early type galaxies. The
measured profile is well fit by a ``Nuker'' law parameterization:
i.e., a broken power law that smoothly transitions from an inner power law
slope, $\Gamma$, to an outer power law slope, $\beta$, at a break
radius, $r_b$.

Most galaxies have $0<\Gamma<1$, and are classified into two broad
categories: core galaxies with $\Gamma<0.3$ and cusp galaxies with
$\Gamma>0.5$. Assuming spherical symmetry and a constant mass-to-light
ratio, the inner stellar profile translates to a stellar density
$\rho_\star\propto r^{-1-\Gamma}=r^{-\delta}$. Core galaxies have
$1<\delta<1.3$, while cusp galaxies have $1.5<\delta<2$.

Galaxies with low mass black holes ($\Mbh\lsim 10^{8} \Msun$) relevant
to TDEs are more likely to have cusp-like stellar density
profiles. Additionally, as described in \citet{Stone&Metzger2016} TDEs
a cuspy stellar density profile results in a higher TDE rate. We adopt
a fiducial value of $\Gamma=0.7$ ($\delta=1.7$), the rate weighted
average for the galaxies in Table C of \citet{Stone&Metzger2016}.

 {\bf Recently changed from $\delta=1.8$, need to
  double check all expressions.}

\subsubsection{Gas density profiles}
The radio emission produced by a TDE jet is most sensitive to the gas
density at the ``Sedov'' or deceleration radius, where the jet has
swept up it's own mass in nuclear gas. For Sw1644 the deceleration
radius of the jet was $\sim 10^{18}$ cm \citep{Mimica+2015}. Thus, we
focus on determining the nuclear gas density on parsec scales.

In the presence of strong heating, one-dimensional steady-state flow
is characterized by an inflow-outflow structure. The velocity passed
through 0 at the ``stagnation radius'', $\rs$.  Mass loss from stars
interior to the stagnation radius is accreted, while that outside
$\rs$ is unbound in an outflow from the nucleus. 

Fig.~\ref{fig:profiles} shows example profiles of the steady-state gas
density calculated for a core and a cusp stellar density profile. The
stagnation radius is marked as a blue dot on each profile.

As described in GSM15 the stagnation radius can be estimated
analytically by the expression below

\begin{equation}
r_s \simeq f(\delta) \frac{G M}{v_w^2},
\label{eq:rs}
\end{equation}


\noindent where $f(\delta)$ is a function of the stellar density slope with
value of order unity.  The density at the stagnation radius, $n(\rs)$,
is determined by injection rate of stellar wind material inside of
$\rs$, $\dot{M}$. This is proportional to the stellar mass enclosed
inside of $\rs$, $M_{\star}(\rs)$:

\begin{equation}
\dot{M}=\frac{\eta M_{\rm \star}(\rs)}{t_h}
\label{eq:dotM}
\end{equation}

\noindent For our fiducial value of $\delta=1.7$,

\begin{equation}
\dot{M}= 3.0 \times 10^{-6} \Mbh[,7]^{0.22} \eta_{0.02} \left(\frac{r_s}{\rm
  pc}\right)^{1.3} \Msun \, {\rm yr}^{-1},
\end{equation}


\noindent where $\eta_{0.02}=\eta/0.02$. 

The density at the stagnation radius, $n(\rs)$, can be estimated by
equating the gas injected by stellar winds over a dynamical time at
the stagnation radius ($t_{\rm dyn} (\rs)=\sqrt{\rs^3/G \Mbh}$) to the
gas mass enclosed inside of the stagnation radius:

\begin{align}
  &\frac{4 \pi}{3} \rs^3 \mu m_p n(r_s) \simeq \dot{M} t_{\rm dyn}
  (\rs) \nonumber\\
  &n(r_s) \simeq 0.21 \eta_{0.02} \Mbh[,7]^{-0.28} \left(\frac{r_s}{\rm
      pc}\right)^{-0.2}
\label{eq:nrs}
\end{align}
 
\noindent Using equation~\eqref{eq:rs}, for $r_s$, we obtain 

\begin{equation}
n(r_s) \simeq 0.2 \, v_{500}^{0.4} \eta_{0.02} \Mbh[,7]^{-0.48} {\rm cm}^{-3},
\label{eq:nrs2}
\end{equation}

\noindent where $v_{500}=v_w/\left(500 \,\mathrm{km
    s^{-1}}\right)$. Near the stagnation radius, we find
that the radial power law slope of the gas density profile is $\sim
(4\delta-1)/6$. For our fiducial value of $\delta$=1.7, this gives
$n(r)\propto r^{-1}$ near the stagnation radius. Furthermore, the
density profile remains close to $n\propto r^{-1}$ for radii of
interest.\footnote{For cusp galaxies, the gas velocity asymptotes to a
  constant between the stagnation radius and break radius (if they are
  sufficiently far apart). In this region we expect $n\propto
  r^{1-\delta}$.}  Thus, we adopt $n(r)= n_{18} (r/10^{18} {\rm
  cm})^{-1}$ as our fiducial density profile.  We also experiment with
a core-like density profile (appendix~\ref{app:core}), and compare the
radio light curves for jets going through core and cusp-like density
profiles in $\S\ref{sec:profileComp}$.


\noindent For our fiducial $n\propto r^{-1}$ and equation~\eqref{eq:nrs}, we
obtain

\begin{equation}
  n_{18}\simeq 0.6 \left(\frac{r_s}{\rm pc}\right)^{0.8}
  \Mbh[,7]^{-0.28} {\rm cm^{-3}},
  \label{eq:n18}
\end{equation}


\noindent where $n_{18}$ is the density at $10^{18}$ cm. Substituting
equation~\eqref{eq:rs} for the stagnation radius in
equation~\eqref{eq:n18} gives

\begin{equation}
n_{18}\simeq 0.3 \Mbh[,7]^{0.52} \eta_{0.02} v_{500}^{-1.6} {\rm
  cm^{-3}}.
\label{eq:n182}
\end{equation} 
%
In general the gas density steepens at the stellar break radius, $r_b$
(see Fig.~\ref{fig:profiles}). However, such breaks will only have an
effect on the observed radio emission if $r_b$ is inside of the Sedov
radius, where the jet sweeps up its own mass in CNM gas.  The jet
mostly decelerates inside of this radius, becoming
non-relativistic. Therefore, the profile outside of this radius will be of
minimal importance in shaping the radio light curve. We plot the Sedov
radius for different energy jets on each of the profiles in
Fig.~\ref{fig:profiles}. The measured break radii of all but four the
\citet{Lauer+2007} galaxies is larger than 10 pc. This exceeds
the Sedov radius, for even a very powerful jet (isotropic equivalent
energy $E=4\times 10^{54}$ erg), interacting with a low density CNM
($n_{18}=2$). If a nuclear star cluster (NSC) is present in the
galactic center there will be a smaller effective break radius at the
outer edge of the NSC (typically 1-5 pc--see
\citealt{Georgiev+2014}). However, even for a break radius of 1 pc the
break radius will only be inside the Sedov radius for small densities
($n_{18}<20$) and powerful jets {\bf AG still feel that this is a
  relevant region of parameter space}. Henceforward we will not
consider the effects of the break radius in our analysis.


\begin{figure}
\includegraphics[width=8cm]{sedov_radius.pdf}
\caption{\label{fig:profiles} Steady-state gas density profiles for
  black hole mass $10^{7} \,\Msun$, heating parameter $v_w=600$ km
  s$^{-1}$, and cuspy ({\it solid}) and core-like ({\it dashed})
  stellar density profile. Both profiles go as $r^{-3}$ outside of 25
  pc. Note that we normalize the gas density to the density at
  $10^{18}$ cm ($n_{18}$). Colors mark the Sedov radii of modestly
  relativistic ($\Gamma=2$) jets with different isotropic equivalent
  energies. This is where the jet sweeps up its own mass in CNM gas.}
\end{figure}



\subsubsection{Plausible Density Range}
\label{sec:densAllowed}

We calculate the $n_{18}$ for stellar populations
formed via two bursts of star formation, with the older burst
occurring a Hubble time ago.  Both the young and old stars have the
same (cuspy) density profile, which implies that the gas density
profile goes as $r^{-1}$.

For a sufficiently large burst occurring less than 40 Myr ago, the gas
is heated by energy from massive star winds.\footnote{Type II
  supernovae are another important heating source and dominate stellar
  winds after $\sim$6 Myr \citep{Voss+2009}. However, heating from
  Type II SNe will be intermittent and is not included here for
  simplicity.}  In this case, the mass return ($\eta$) and heating
parameters ($v_w$), can be calculated using the procedure described in
appendix C GSM15. Then, the density at $10^{18}$ cm, $n_{18}$, follows
from equation~\eqref{eq:n182}.

For older stellar populations, heating may come from a few
different sources including Ia Supernovae or AGN feedback {\bf AG also
  unbound debris streams produced in tidal disruption events (see
  \citep{Guillochon+2015a})}. We focus on quiescent phases, where the
SNe Ia dominate. As discussed in GSM15 the Ias can clear the gas
inside of the Ia radius where the time between
successive Ia supernovae is equal to the dynamical time. In this case,
$n_{18}$ can be calculated by substituting Ia radius in
equation~\eqref{eq:n18}. The Ia radius is calculated as described in
GSM15 for times $t>300 \,{\rm Myr}$.\footnote{The Ia rate is given by $8.8
\times 10^{-13} \left(\frac{t}{3\times 10^{8} {\rm yr}}\right)^{-1.12}
\, \Msun^{-1} \, {\rm yr}^{-1} $ for $t>3\times 10^8$ years.} In GSM15
we incorrectly extrapolated this rate all of the way to 3 Myr--this is
clearly un-physical as no white dwarfs would have formed by this
time. Here, we instead take the Ia rate to be 0 for times less than 40
Myr and flat from 40-300 Myr.

Fig.~\ref{fig:param} shows how $n_{18}$ varies with star burst
properties.  We find a maximum density of $\sim 2,700$ cm$^{-3}$,
$\sim 4$ Myr after a burst of star formation forming 100\% of the
stars in the nucleus. In this case both the energy and mass budgets
are dominated by fast winds from massive stars ($\gsim 15 \, \Msun$).
Although such a large gas density would be present in the immediate
aftermath of a starburst, the wind mass loss rate (and hence the gas
density) declines as $t^{-3}$. Therefore, the gas density would
decline by an order of magnitude after just a few Myr.

The lowest density $\sim 0.03$ cm$^{-3}$ achieved for a relatively
modest burst, forming $4\times 10^{-4}$ of the stars. In this case
there is a high heating rate from the young, massive stars and a low mass
injection rate, with both young and old stars contributing.

This is likely an underestimate as we do not include discreteness
effects. In particular, we assume that massive stars provide a
spatially homogeneous heating source, even on small scales. However,
for small bursts of star formation (the doubly hatched region in
Fig.~\ref{fig:param}) there may be less than one massive star inside
of the nominal stagnation radius (equation~\ref{eq:rs}). This is the
case for the star formation history producing the minimum $n_{18}$
($\sim 0.03$ cm$^{-3}$). Using the radius enclosing one massive
($\gsim 15 \Msun$) star as the stagnation radius in
equation~\eqref{eq:n18} gives a much larger $n_{18}$ of $\sim 0.5$
cm$^{-3}$.

A typical host galaxy for TDE has evidence of some star formation
within 1 Gyr \citep{French+2016}. This corresponds the right hand side
of Fig.~\ref{fig:param}. In this case the heating rate will be
dominated by Ia supernovae and $n_{18}\sim 80$ cm$^{-3}$.

\subsubsection{Thermal properties of gas}

To above estimates correspond to the total gas density resulting from
stellar wind material, including all phases of gas. In the first few
Myr after a star burst the injected stellar wind material will all be
quite hot, with temperature $\gsim 10^{7}$ K due to the thermalized
wind kinetic energy. Later, SNe Ia provide intermittent heating, but
the stellar wind material that accumulates on small scales, between
successive Ias would be much cooler, with temperature $\sim 10^{3}$ K,
and would condense into clumps. However, as discussed in
Section~\ref{sec:empirical}, the radio emission from a TDE jet would
be quite similar for a clumpy and a smooth medium with the same
average density.



It is possible that some of these cold clumps would condense into
stars. To estimate the star formation rate, we assume that the CNM
self-regulates itself to a condition of marginal thermal stability (so
that the cooling time is ten times the dynamical time). For the case
of a TDE host with a $\sim$1 Gyr old stellar population, we find that
this requires a star formation rate of  $5.3 \times 10^{-4}
\,\Msun$ yr$^{-1}$, which means that $\sim 40$\% of the gas injected
on small scales would be turned into stars. This would leave a
substantial gas density $n_{18}=50$ cm$^{-3}$.

\begin{figure} 
  \includegraphics[width=8cm]{cnm_plot.pdf}
  \caption{\label{fig:param} Isocontours for the density at $10^{18}$
    cm (blue lines) for a $10^{7} \, \Msun$ black hole and bursts of
    star formation forming a fraction, $f_{\rm burst}$, of the stars
    $t_{\rm burst}$ years ago. The remaining stars formed a Hubble
    time ago. The old and young stars both have a cuspy density
    profile, which means that the gas density goes as $r^{-1}$.
    Hatched areas indicate regions of parameter space, where massive
    stars ($\gsim 15 \, \Msun$) dominate the gas heating rate, but we
    would expect less than one (doubly hatched) or less than ten
    (singly hatched) massive stars inside the nominal stagnation
    radius (equation~\ref{eq:rs}). In these regions of parameter space
    discreteness effects not captured by our formalism are important.}
\end{figure}


% \begin{figure}
%   \includegraphics[width=8cm]{cnm_plot_2.pdf}
%   \caption{\label{fig:param2} $n_{18}$ as a function of the mass
%     return ($\eta$) and heating ($v_w$) parameters for a $10^7 \Msun$
%     black hole. We translate the gray line from Fig.~\ref{fig:param}
%     to this parameter space. $f_{\rm burst}$ increased in the direction
%     indicated by the gray arrow. As indicated by the black solid line,
%     the area below the black solid line is thermally unstable
%     (\textit{see text for discussion}).}
% \end{figure}

\subsection{Empirical Constraints on CNM Density}
\label{sec:empirical}

In this section we translate observed constraints on the accretion
rate distribution of supermassive black holes into constraints on the CNM
density.  The accretion rate can be written as
\begin{equation}
\dot{M}_{\bullet} = f_{\rm in} 4 \pi r^2 \mu m_p n v,
\label{eq:mdot}
\end{equation}
where $n$ is the average density at radius $r$ and $f_{\rm in}$
represents the fraction of the large scale inflow, which actually
accretes onto the black hole. At small radii, the velocity approaches
free-fall value. Replacing $v$ with the free-fall velocity in
equation~\eqref{eq:mdot} gives

\begin{equation}
  \dot{M}_{\bullet} = 1.1\times 10^{-5} \Mbh[,7]^{0.5} f_{\rm in}
  n_{18} \,\,\Msun {\rm yr}^{-1}.
\end{equation}
%
The corresponding Eddington ratio is

\begin{equation}
  \lambda\equiv L/L_{\rm Edd} = 4.8 \times 10^{-5}
  \left(\frac{\epsilon_{\rm rad}}{0.1}\right) \Mbh[,7]^{0.5} f_{\rm in}
  n_{18},
\label{eq:n18Edd}
\end{equation}
%
where $\epsilon_{\rm rad}$ is the radiative
efficiency. \citet{Kauffmann&Heckman2009} present distributions of the
OIII line luminosity, ${\rm L[OIII]}/\Mbh$ for a volume limited {\bf
  AG double check} sample of SDSS galaxies.  As described by
\citet{Kauffmann&Heckman2009} ${\rm L[OIII]}/\Mbh=1.7$, roughly
corresponds to Eddington ratio of 1 {\bf AG: Look through original
  citations}. Using this bolometric correction maps the distribution
of ${\rm L[OIII]}/\Mbh$ from \citet{Kauffmann&Heckman2009} to a
distribution of Eddington ratios.

Converting the distribution Eddington ratios into a distribution of
gas densities requires a prescription for the radiative efficiency.  For
low accretion rates, we refer to \citet{Sharma+2007}, who estimate the
radiative efficiency of low luminosity AGN using MHD shearing box
simulations.  They find (see their Fig.~6)

\begin{align}
&\epsilon_{\rm rad} \simeq 
\begin{cases}
  0.03 \left(\frac{\dot{M}_{\bullet}}{10^{-4}\dot{M}_{\rm Edd}}\right)^{0.9} & \frac{\dot{M}_{\bullet}}{\dot{M}_{\rm edd}} \lsim 10^{-4} \\
 0.03 &  10^{-2} \gsim \frac{\dot{M}_{\bullet}}{\dot{M}_{\rm edd}}
 \gsim  10^{-4}
\end{cases}\nonumber\\
&\dot{M}_{\rm Edd} \equiv \frac{L_{\rm Edd}}{0.1 c^2}
\label{eq:efficiency}
\end{align}
%
with a factor of $\sim$ 5 uncertainty for small $\dot{M}/\dot{M}_{\rm
  Edd}\lsim 10^{-4}$, since the radiative efficiency depends on
uncertain microphysical parameters. We adopt
equation~\eqref{eq:efficiency} for $\dot{M}/\dot{M}_{\rm Edd}<0.01$,
and $\epsilon_{\rm rad}=0.1$ for $\dot{M}/\dot{M}_{\rm Edd}>0.1$. We
linearly interpolate between the two for intermediate values of the
Eddington ratio.

\begin{figure}
\includegraphics[width=8cm]{fcum_n18.pdf}
\caption{\label{fig:n18Cum} Cumulative distribution of $f_{\rm in}
  n_{18}$ for black holes with mass $\Mbh\simeq 10^{7} \Msun$ based on
  the observed cumulative distributions of Eddington ratios from
  \citet{Kauffmann&Heckman2009} {\it (solid black line)}. Recall that
  $f_{\rm in}$ is the ratio of the true accretion rate onto the black
  hole to the inflow rate of free-falling gas at $10^{18}$ cm.  The
  dashed blue line is the corresponding cumulative distribution of
  $n_{18}$, for $f_{\rm in}$ from the simulations of \citet{Li+2013}
  ($\dot{M}_{\rm Acc}/\dot{M}_{\rm Bondi}$ in their Figure 6). The
  distribution of $f_{\rm in} n_{18}$ ($n_{18}$) is unreliable to the
  left of the solid black (dashed blue) vertical line. Such low
  densities correspond to systems with very small OIII luminosities,
  which could be coming from star formation rather than the black hole
  accretion flow.}
\end{figure}


Fig.~\ref{fig:n18Cum} shows the distribution of $f_{\rm in} n_{18}$
resulting from combining the distributions of Eddington ratio from
\citet{Kauffmann&Heckman2009} with equations~\eqref{eq:n18Edd} and
~\eqref{eq:efficiency}.  The absence of galaxies with $f_{\rm in
}n_{18} \lsim$ few cm$^{-3}$ places a lower bound on $n_{18}$ of a few
cm$^{-3}$.  However, measurements of Eddington ratio below $\sim
10^{-3}$ (shown with a vertical line in Fig.~\ref{fig:n18Cum}) are not
reliable. This allows a moderate fraction of galaxies ($\sim 30\%$) to
have much lower gas densities.


In order to obtain the cumulative distribution for $n_{18}$, we need a
prescription for $f_{\rm in}$. We use the results \citet{Li+2013}, who
performed two-dimensional hydro-dynamical simulations of axisymmetric
rotating accretion flows. They find that when the inflow rate on large
scales is very sub-Eddington ($\dot{M}/\dot{M}_{\rm Edd} \lsim
10^{-4}$), cooling is inefficient and $f_{\rm in}\sim 0.01$. On the
other hand, when $\dot{M}/\dot{M}_{\rm Edd}\gsim 10^{-2}$, $f_{\rm
  in}$ approaches unity.  We use $\dot{M}_{\rm Acc}/\dot{M}_{\rm
  Bondi}$ in their Figure 6 for $f_{\rm in}$.\footnote{Note that
  \citet{Li+2013} use an alpha viscosity prescription with
  $\alpha=0.01$ {\bf AG double check}.  $f_{\rm in}$ would depend on
  this choice. This uncertainty in $f_{\rm in}$ translates into a
  systematic uncertainty in the distribution of $n_{18}$.}  With this
choice only a third of nuclei have $n_{18}>2,000$ and only 6\% have
$n_{18}>10^{4}$. Note that existing tidal disruption event candidates
are unlikely to be in the high density tail with either no sign or
sign of or very weak AGN emission lines
(e.g. \citealt{van-Velzen+2011, Arcavi+2014}).

Two potential complications to keep in mind are (i) clumpiness of the
CNM and (ii) anisotropy. The distributions above are distributions of
the {\it average} $n_{18}$.  Most likely, some of the nuclear gas in a
low density hot phase, while the rest is in high density cold
clumps/filaments.  However, while the jet is relativistic the
light curve of a jet propagating through a clumpy medium will differ
little from that of a smooth medium with the same average
density. Even in the late stages when the jet becomes non-relativistic
clumps will only make a difference if the size of the clumps is
comparable to the size of the jet {\bf AG include some refs here}.

The cold gas could be distributed anisotropically. For example, it
could be concentrated in a ring-like structure. In this scenario, the
some fraction of jets would likely be stifled by the very dense
ring. However, such a ring would not block all jet propagation
directions.

There may also be anisotropies present in the hot phase of gas. AGN
feedback would may blow low density bubbles in the
CNM. \citet{Russell+2013} used Chandra x-ray observations to measure
gas density and temperature profiles for a sample of massive
elliptical galaxies. The measured electron density on scales of $\sim
100$ pc is $\gsim 0.1$ cm$^{-3}$. Note that the gas density at 100 pc
would be irrelevant for a TDE jet, but we would not expect the gas
density to be decreasing towards the galactic center in a steady
state.  We note that massive black holes in the \citet{Russell+2013}
sample, with $\Mbh\sim 10^{9} \Msun$, whereas black holes tidal
disruption events would have $\Mbh\lsim 10^8 \Msun$. This is because
more massive black holes would not be able to disrupt (main sequence)
stars.


The empirical estimates in this section are nicely complemented by our
analytic results. The distribution of Eddington ratios becomes
unreliable for low luminosities.  In this case it is difficult to put
a lower bound on the density observationally. However, the analytics
give a low density floor expected from injection of stellar wind
material ($\sim 0.5$ cm$^{-3}$).

% On the other hand, the analytic estimates are specific to
% the hot phase of gas. The Eddington ratio distribution probes the
% average density (including any cold clumps). The
% distribution of Eddington ratios gives us confidence in our high
% density limit ($\sim 1000$ cm$^{-3}$).



\section{TDE jets}
\label{sec:jet}

\subsection{Analytic Considerations}
\label{sec:analytic}

The mass of the CNM swept up by a jet is equal to its own rest mass at
``Sedov'' radius. For our fiducial CNM density profile $n=n_{18}
\left(r/10^{18}\right)^{-1} {\rm cm}$ this is given by


\begin{equation}
r_{\rm sedov} = E_{54}^{1/2} \Gamma_{10}^{-1/2} n_{\rm 18}^{-1/2}\,{\rm pc}. 
\end{equation}

\noindent where $E_{54}$ is the isotropic equivalent jet energy
normalized to $10^{54}$ ergs and $\Gamma_{10}$ is the Lorentz factor
of the jet normalized to 10.

From the continuity equation, the (co-moving) density of the jet is
given by (e.g. Uhm \& Beloborodov 2007)
 \begin{align}
   n_{\rm j} =  \frac{L_{\rm j, iso}}{4 \pi r^{2}\Gamma_{\rm
       j}^{2}c^{3}m_p(1 + r \dot{\Gamma_{\rm j}}/c\Gamma_{\rm j}^{3})}
   \approx  \frac{L_{\rm j, iso}}{4 \pi r^{2}\Gamma_{\rm j}^{2}c^{3}m_p},
\end{align}
%
where $L_{\rm j, iso}$ is the isotropic equivalent luminosity of the
jet. The second term in the denominator is generally negligible as
long as the Lorentz factor of the jet changes slowly
($\dot{\Gamma}_{\rm j} \ll c\Gamma_{\rm j}^{3}/r$), as can be shown to
be true at radii $r < r_{\rm sedov}$ as long as $\Gamma_{\rm j}$
changes on timescales $\gtrsim t_{\rm j}$ ({\bf AG: What is $t_j$?
  Time for which jet? Replace $t_0$ later on w/ $t_j$?}).

A critical parameter is the ratio of the density of the jet to the
CNM

\begin{equation}
  f\approx 40\,  L_{\rm j,48} n_{18}^{-1} \Gamma_{10}^{-2} \, \left(\frac{r}{10^{18} {\rm
        cm}}\right)^{-1} 
\end{equation}
%
For a given $f$, the Lorentz factor of the shocked material may be
calculated using the relativistic shock jump condition and pressure
equality between the forward and reverse shocks. In the
ultra-relativistic limit 

\begin{equation}
\Gamma_{\rm sh} \underset{\Gamma_{\rm sh} \gg 1}= \Gamma_{\rm j}\left[1 + 2\Gamma_{\rm j}f^{-1/2}\right]^{-1/2}
\end{equation}
%
However, this expression is problematic when the outflow is mildly
relativistic or non-relativistic. In particular, it gives Lorentz
factors less than 1. A more general expression for $\Gamma_{\rm sh}$
can be obtained using equation 3 from
\citet{Beloborodov&Uhm2006}\footnote{As pointed out by
  \citet{Beloborodov&Uhm2006}, the assumption pressure equality
  between the forward and reverse shocks used to derive this
  expression could be inaccurate.}:

\begin{align}
&\frac{\Gamma_{\rm j}^2-1}{\Gamma_{43}^2-1} f^{-1}=1\\
&\Gamma_{43}=\Gamma_{\rm j} \Gamma_{\rm sh} \left(1-\beta_{\rm sh} \beta_j\right),
\label{eq:gammaShGen}
\end{align}
%
where $\Gamma_{43}$ is the Lorentz of shocked jet material in the
frame of the un-shocked jet. In the lab frame the velocity of the
reverse shock is

\begin{equation}
\beta_{\rm rs}=\frac{\beta_{\rm sh}(f)-\beta_{43}(f)/3}{1-\beta_{\rm
    sh}(f) \beta_{43}(f)/3}.
\label{eq:betars}
\end{equation} 
%
From equations~\eqref{eq:gammaShGen} and ~\eqref{eq:betars} it is
straightforward to find the radius and the Lorentz factor of the
shocked material when the reverse shock crosses the trailing edge of
the jet.

Fig.~\ref{fig:diss} shows the fraction of the slow component
($\Gamma_{\rm j}=2$) kinetic energy dissipated by the reverse shock as a
function of isotropic jet luminosity and $n_{18}$. For the purposes of
our analytic estimates, we assume a constant luminosity jet, which
shuts off after time $5 \times 10^{5}$ s. For the numerical
calculations described in the subsequent sections, the jet luminosity
declines as $t^{-5/3}$ after this time (see equation~\ref{eq:lum}).

\begin{figure}
\includegraphics[width=8cm]{diss.pdf}
\caption{\label{fig:diss} Fraction of the slow component
  ($\Gamma_{\rm j}=2$) kinetic energy dissipated by the reverse shock
  vs. the isotropic jet luminosity ($L_{\rm j,iso}$) and $n_{18}$.}
\end{figure}

\subsection{Jet Models}
\label{sec:numerical}

The jet-CNM interaction is simulated in one and two-dimensions, and
the resulting synchrotron emission is computed using the same
procedure described in \citet{Mimica+2015}. We assume the same angular
Lorentz factor dependence as in previous paper (i.e., $\Gamma = 10$
for the fast inner core and $\Gamma = 2$ for the slow outer sheath),
but introduce a number of modifications regarding the jet energy
when performing 1D simulations.

\subsection{1D vs. 2D Models}
\label{sec:2D}

The preferred numerical model for Swift J1644+57 \citep[Fig.10
in][]{Mimica+2015} was obtained using 2D simulations (the red line in
that figure). However, the light curve of a 1D version of the same
model \citep[black line in Fig. 10 in][see also section 4.2 of that
paper]{Mimica+2015} matches the 2D light curve only at early times,
when the emission from the inner relativistic jet dominates, while at
late times, when the emission from the slow outer core dominates, the
1D model overestimates the emission from the 2D model. In this work we
found a modification of the 1D model that makes its light curves
match the 2D results much more closely.

We first summarize the two-component model as presented in
\citet{Mimica+2015}. The jet has a fast inner core spans an angular
interval $[0, 0.1\ {\rm rad}]$, while the slow outer sheath occupies
$[0.1\ {\rm rad}, 0.5\ {\rm rad}]$. For both components we assume
$E_{\rm ISO} = 4\times 10^{54}$ erg. The crucial thing to notice is
that, keeping $E_{\rm ISO}$ constant, the true jet energy depends only
on the angular interval: $E_{\rm jet} = E_{\rm ISO} (\cos\theta_{\rm
  j,min} - \cos\theta_{\rm j,max})$. The hydrodynamic evolution of the
components is independent of angle in 1D simulations, but the
radiative transfer/light curve calculation is sensitive to the jet
geometry.

Although the sheath is injected in a relatively
narrow angular interval, at late stages of the jet evolution the bow
shock created by its interaction with CNM spans a much larger interval,
i.e. the slow component becomes almost isotropic \citep[bottom two
panels in Fig. 8 in][]{Mimica+2015}. Thus, the light curves from the
2D simulations differ from those of the 1D simulations in which the
angular size of the jet is held fixed. We note that the stage at
which the jet becomes ``isotropic'' depends on the CNM density,
i.e. the denser the CNM, the faster this is expected to happen.

Since the agreement between 1D and 2D models is good for early times
(when the fast component dominates), we modify only the slow component
in 1D models. We assume that is spans $[0.1\ {\rm rad}, \pi/2\ {\rm
  rad}]$ and lower its isotropic equivalent energy so that the true
jet energy remains unchanged with respect to the original model:

\begin{equation}\label{eq:Eiso}
 E_{\rm ISO, new} = E_{\rm ISO} \left(\frac{\cos(0.1) - \cos(0.5)}{\cos(0.1) - \cos(\pi/2)}\right)\approx 0.12 E_{\rm ISO}\, .
\end{equation}
%
We assume the same time dependence for the jet luminosity as in
\citet{Mimica+2015},

\begin{equation}\label{eq:lum}
L_{\rm j, ISO}(t) = L_{j,0}\max\left[1, (t/t_0)\right]^{-5/3}
\end{equation}
%
where $t_0 = 5\times 10^5$ seconds. Integrating equation~\ref{eq:lum}
in time from $0$ to $\infty$ would give $E_{\rm ISO}$. Thus,
$L_{j,0}=0.4\, E_{\rm ISO}/t_0$. 

We show a comparison of this modified 1D approach with the true 2D
result in Figure~\ref{fig:1D2DB}, for $n_{18}=60$ and $n_{18}=2000$
and $E_{\rm ISO}=4 \times 10^{54}$ ergs. For $n_{18}=2000$ the
agreement is excellent, while for $n_{18}=60$ the 1D models still
over-predict still over-predict the flux at times after the peak of
the light curve.

% We summarize our fiducial jet model and our grid of simulations in
% Table~\ref{tab:jetParams}

\begin{table}
\begin{threeparttable}
  \caption{\label{tab:jetParams} Summary of parameters for grid of on
    axis jet simulations.}
  \begin{tabular}{lll}
    \hline
    {Fast component} & Fiducial value & Other values \\ 
    $\Gamma_{\rm j}$ & 10 &\\
    $[\theta_{\rm min}$, $\theta_{\rm max}]$ & [0, 0.1] radians & \\
    $E_{\rm ISO}/10^{54} {\,\rm erg}$ & 4  & 0.04, 0.4\\
    $E$ & $2 \times 10^{52}$ ergs\\
    % $L_{\rm j,ISO}$ & $L_{j,0} \max\left[1, (t/t_0)\right]^{-5/3}$  \\
    % $t_0$ & $5\times 10^5$ s &\\
    % $L_{j,0}$ & $3.2 \times 10^{48}$ erg/s & \\
    \hline 
    Slow component\\
    $\Gamma_{\rm j}$ & 2 \\
    $[\theta_{\rm min}$, $\theta_{\rm max}]$ & [0.1, $\pi/2$] radians
    & \\
    $E_{\rm ISO}/10^{54} {\rm erg}$ & $4.7$ & 0.047, 0.47 \\
    $E$ & $4.7 \times 10^{53}$ erg & \\
    % $L_{\rm j, ISO}$ & $L_{j,0} \max\left[1, (t/t_0)\right]^{-5/3}$ & \\
    % $t_0$ & $5\times 10^5$ s & \\
    % $L_{j,0}$ & $3.8 \times 10^{47}$ erg/s & \\
    \hline
    Micro-physical parameters\\
    $\epsilon_e$ & 0.1 &  0.05, 0.2\\
    $\epsilon_b$ & 0.002 & 0.001, 0.005\\
    $p$ & 2.3\\
    \hline 
    Nuclear gas density \\
    $n_{18}$ & 60 & 2, 11, 345, 2000
  \end{tabular}
% \begin{tablenotes}
% \item $^{\dagger}$  Additional values of physical parameters we tried.
% \end{tablenotes}
\end{threeparttable}
\end{table}


\begin{figure*}
\includegraphics[width=16cm]{1D_2D.pdf}
\caption{\label{fig:1D2DB} Comparison of 1D and 2D light curves for
  $n_{18}=60$ (top) and $n_{18}=2000$ (bottom) for frequencies of 1
  GHz (left) and 5 GHz (right) and an observer angle of 0.8 radians to
  the jet axis. We assume that the gas density $n\propto r^{-1}$ for
  $n_{18}=2000$, but take $n\propto r^{-1.5}$ for $n_{18}=60$ for
  computational convenience--this model was previously computed in
  \citet{Mimica+2015} and 1D results suggest that the density slope
  would have minimal impact on the results (see
  $\S$~\ref{sec:profileComp})}.
\end{figure*}

\section{Results}
\label{sec:results}
We calculate light curves for a grid of on-axis jet simulations, with
five different values of $n_{18}$ (2, 11, 60, 345, and 2000 cm$^{-3}$)
and three different values of the jet energy ($5\times 10^{51}$,
$5\times 10^{52}$, $5\times 10^{53}$ erg). The grid is summarized in
Table~\ref{tab:jetParams}.


Fig.~\ref{fig:jetContours} shows isocontours of the peak time and the
peak luminosity for on-axis jets. The following equation roughly
reproduces the behavior of the peak luminosity across our grid:

\begin{align}
\nu L_{\nu, p}=&2.6 \times 10^{41} \left(\frac{E}{5 \times 10^{53}
    {\rm ergs}}\right)^{0.9} \left(\frac{\epsilon_e}{0.1}\right)^{0.6} \left(\frac{\epsilon_b}{0.002}\right)^{0.3}
\nonumber\\
&\left(\frac{\nu}{5 {\rm GHz}}\right)^{1.8} {\rm ergs/s}
\label{eq:peakLum}
\end{align}
%
Remarkably, the peak luminosity is only weakly dependent on density
across most of the parameter space. In the optically thin limit the
peak luminosity scales as the density $n^{0.5}$. In our case the jet
is initially optically thick and then becomes optically thin at later
times. In fact, the peak of the light curves occurs, when the total
optical depth of the jet is $\sim 1$. {\bf AG finish. Need to explain
  why the dependence on density is so weak...}

\begin{figure*}
  \includegraphics[width=8cm]{lp_contours.pdf}
  \includegraphics[width=8cm]{tp_contours.pdf}
  \caption{\label{fig:jetContours} Isocontours of peak radio
    luminosity and peak time for a jet of energy $E$ interacting with
    nuclear gas with density profile $n=n_{18}
    \left(r/10^{18}\right)^{-1}$ (\textit{solid lines}). The jet model
    is described in $\S$~\ref{sec:2D}. We also see analytic fits (see
    equations~\ref{eq:peakTime} and~\ref{eq:peakLum}) to the numerical
    results as dashed lines.}
\end{figure*}

The following equation roughly reproduces the behavior of the peak
time:

\begin{align}
  t_p =& 39.5 \left(\frac{E}{5 \times 10^{53} {\rm ergs}}\right)^{0.3}   \left(\frac{\epsilon_e}{0.1}\right)^{0.4}
  \left(\frac{\epsilon_b}{0.002}\right)^{0.4}\nonumber\\
 & \left(\frac{\nu}{5 {\rm GHz}}\right)^{-1} n_{18}^{0.35}  {\rm days}
\label{eq:peakTime}
\end{align}
%
Note that the peak time occurs later for larger $n_{18}$, as the
transition from optically thick to optically thin occurs at later times.

\subsection{Emission by components}
Our radio light curves include contributions from both the fast
$\Gamma=10$ core and the slow $\Gamma=2$ sheath. In general, the fast
component is more important for earlier times, higher frequencies, and
lower densities. 

We show the relative contributions of the fast and slow components to
the 5 GHz light curve of a $5 \times 10^{53}$ erg jet for different
ambient gas densities. For $n_{18}=2000$, the slow component dominates
for nearly all times.  For $n_{18}=2$, the fast component dominates at
early times and at the peak of the light curve, while the slow
component dominates after $\sim$1 year.

\begin{figure}
\includegraphics[width=8cm]{components.pdf}
\caption{\label{fig:components} Contributions from the fast
  (\textit{red}) and slow (\textit{blue}) to the 5 GHz light curves
  (\textit{black}) for $n_{18}=2$ (\textit{top}) and $n_{18}=2000$
  (\textit{bottom}).}
\end{figure}

We don't include emission from the reverse
shock. Generally, we expect it would be either self-absorbed or
absorbed by the forward shock. However, emission from the reverse
shock may be important for \textbf{...AG flesh out} 

\subsection{Comparison with radio detections and upper limits.}

In the left panel of Fig.~\ref{fig:lightcurves} we show on-axis light
curves for our fiducial $5\times 10^{53}$ erg jet (comparable to
Sw1644) and three different densities: n$_{18}$: 2, 60, and 2000
cm$^{-3}$ together with radio upper limits and detections for various
TDE candidates in Fig.~\ref{fig:lightcurves} (compiled from various
sources into Table 1 of \citealt{Mimica+2015}).  All of the 5 GHz
light curves fall above the upper limits. However, we were unable to
perform a 2D calculation for $n_{18}=2$.  For $n_{18}=60$, the 2D
light curve falls $\sim$one order of magnitude below the 1D light
curve after $\sim$ 10 years. Thus, we caution that the 1D light curves
for $n_{18}=2$ likely overestimate the late time radio luminosity. For
this density even a Sw1644 like jet would fall below existing upper
limits. 

Multiple radio measurements from several months to a few years after a
tidal disruption flare would provide better constraints on the
presence of TDE jets. For example, suppose we have a TDE candidate
luminosity distance $d_L$ away and the radio flux density at 5 GHz is
constrained to be less than $S$. Then from equation~\eqref{eq:peakLum},
we can constrain the energy of the jet to be

\begin{equation}
E\lsim 7.6 \times 10^{48} \left(\frac{S}{50 \,\mu{\rm Jy}}\right)^{1.1}
  \left(\frac{d}{200 \,{\rm Mpc}}\right)^{2.2} {\rm ergs}
\end{equation}


\begin{figure*} 
  \includegraphics[width=8cm]{lightcurves.pdf}
  \includegraphics[width=8cm]{lightcurves_en.pdf}
  \caption{\label{fig:lightcurves} \textit{Right-hand panel:} On-axis
    ($\theta_{\rm obs}=0$) radio light curves for our fiducial jet
    model and three different values of n$_{18}$: 2, 60, and
    2000. Solid lines correspond to the light curves from 1D jet
    simulations. When available ($n_{18}=60$ and $n_{18}$=2000), we
    have plotted light curves from the 2D jet simulations as dashed
    lines. Note we use $n\propto r^{-1}$ for $n_{18}=$ 2 and 2000, but
    $r^{-1.5}$ for $n_{18}=60$, as this model had already been
    computed and since the 1D results suggest that the density slope
    will have minimal impact on the results (see
    $\S$~\ref{sec:profileComp}).  Radio upper limits and detections
    are shown as black/gray triangles and squares respectively
    (compiled in Table 1 of \citealt{Mimica+2015}). The single upper
    limit in the top panel corresponds to 1.4 GHz. Gray triangles and
    squares in the middle panel indicate upper limits and detections
    at 3.0 GHz, while black triangles indicate upper limits at 5.0
    GHz. \textit{Left-hand panel} 1D on-axis light curves for a jet
    with a total energy of $5\times 10^{51}$ ergs, together with radio
    upper limits and the light curves for our fiducial case
    (translucent lines).}
\end{figure*}

\subsection{Effects of viewing angle}
We show a comparison off and on-axis light curves in
Fig.~\ref{fig:onOff}.  For $n_{18}=2000$ the light curves differ
little.  For $n_{18}=60$ off-axis luminosity is lower by a factor of a few
before and near the peak. However, for late times
the off- and on-axis light curves are quite similar. This is as
expected, at late times the jet becomes more isotropic, which means
the viewing angle has relatively little effect on the observed light curve.

\begin{figure*}
\includegraphics[width=16cm]{on_off.pdf}
\caption{\label{fig:onOff} Comparison of (2D) off- and on-axis light
  curves for $n_{18}=60$ (top) and $n_{18}=2000$ (bottom) for
  frequencies of 1 GHz (left) and 5 GHz (right). The observer line of
  sight is taken to be at $\theta_{\rm obs}=0.8$ with respect to the
  jet axis. We assume an $n\propto r^{-1}$ density profile for
  $n_{18}=2000$, but take $n\propto r^{-1.5}$ for $n_{18}=60$ for
  computational convenience--this model was previously computed in
  \citet{Mimica+2015} and 1D results suggest that the density slope
  would have minimal impact on the results (see
  $\S$~\ref{sec:profileComp}).}
\end{figure*}

% \subsection{Effect of jet energy}
% In the left-hand panel of Figure~\ref{fig:lightcurves} we show radio
% light curves for a jet with energy $E=5\times 10^{51}$ ergs--two
% orders of magnitude smaller than for the high energy jet.  The peak
% luminosities are also $\sim$ two orders of magnitude lower, as the
% peak luminosity is approximately linear in jet energy. 

% For low densities ($n_{18}\sim 60$ cm$^{-3}$) the radio light curves
% for the low energy jet fall below the observed radio upper limits at
% late times. 


\subsection{Effect of gas density profile}
\label{sec:profileComp}
We have also computed the radio light curves for different shapes of
the gas density profile, while holding $n_{18}$ fixed. In
Fig.~\ref{fig:cores}, we show on-axis radio light curves for our
fiducial $n\propto r^{-1}$ and a core galaxy profile
(equation~\ref{eq:cores}), both with $n_{18}=2$.
The results are within a factor of a few of each other. The core and
cusp light curves are even closer at higher densities, and virtually
indistinguishable at $n_{18}=2000$. For larger densities, the jet only
samples small radii, where the core and cusp profiles are quite
similar (see Fig.~\ref{fig:profiles}). It is only at lower densities,
for which the Sedov radius lies outside of the flattening of the core
density profile, that noticeable differences emerge. {\bf AG why are
  there differences at early times for $n_{18}=2$ but not
  $n_{18}=2000$. Is this simply an artifact of the starting radius?}

In Fig.~\ref{fig:profs2} we compare the 1D on-axis light curves for
density two density profiles: $n\propto r^{-1}$ and $n\propto r^{-1.5}$
with $n_{18}=60$. For most times the light curves are very close.
This is as expected, since the Sedov radius, $r_{\rm sedov}\sim
10^{18}$ cm. 

Overall we conclude the slope of the stellar density will profile will
have minimal effect on the radio light curve, as long as the density
near the deceleration radius is fixed. 

\begin{figure} 
  \includegraphics[width=8cm]{fig_cores.pdf}
  \caption{\label{fig:cores} Comparison of (on-axis) light curves for
    $n_{18}=2$ and two different CNM gas density profiles: $n\sim
    r^{-1}$ and the core galaxy profile from \eqref{eq:cores} with
    $r_s=10^{18}$ cm.}
\end{figure}


\begin{figure} 
  \includegraphics[width=8cm]{profs2.pdf}
  \caption{\label{fig:profs2} Comparison of (on-axis) light curves for
    $n_{18}=60$ and two different CNM gas density profiles: $n\propto
    r^{-1}$ ({\it solid black}) and $n\propto r^{-1.5}$ ({\it dashed
      red}).}
\end{figure}



\section{Summary and Conclusions}
\label{sec:conc}

We calculate radio light curves for tidal disruption event jets
propagating through different circumnuclear (CNM) gas densities. We
simulate the jet propagation using both 1D and 2D hydrodynamic
simulations. We then post-process these to produce radio synchrotron
light curves. To isolate the effects of the density profile we take a
fixed two component jet model from \citet{Mimica+2015}, which produces
a good fit to the observed radio data in Sw1644. We
consider a broad range of gas densities motivated by analytic
estimates of stellar wind mass injection and empirical constraints
based on observed distributions of Eddington ratios. Our conclusions
are summarized as follows.

\begin{enumerate}
\item We estimate the nuclear gas densities expected from injection of
  stellar wind material for different star formation histories. We
  find that that range of gas densities at 10$^{18}$ cm is $n_{18}
  \sim$ 0.5-2000 cm$^{-3}$.

\item The slope of the gas density profile depends on the slope of the
  stellar density profile. We expect a typical TDE host to have cuspy
  stellar density profile inside of a few pc, with $\rho_\star
  \propto r^{-1.7}$. This translates into a gas density profile $n
  \propto r^{-1}$. The radio light curve of a TDE jet is most
  sensitive to the density at the deceleration/Sedov radius (where it
  has swept up its mass in CNM gas). The light curve will be
  insensitive to changes in slope for fixed density at the
  deceleration radius.

\item We use the distribution of Eddington ratios (measured by
  \citealt{Kauffmann&Heckman2009} using $L[OIII]/\Mbh$) for a sample
  $\sim 10^{7} \Msun$ black holes from SDSS to obtain an empirical
  constraint on the average circumnuclear gas density, including all
  phases of gas. We find that $\sim90\%$ of galaxies in the sample
  have $n_{18}<10^{4} \, {\rm cm}^{-3}$, although we note that there is
  considerable uncertainty in translating an observed Eddington ratio
  to a circumnuclear gas density. TDE hosts would not fall on this
  high density tail, as TDE hosts have either weak or no detected OIII
  emission.

\item We take a jet model which fits the radio data for the Sw1644
  transient (from \citealt{Mimica+2015}) and run it through a range of
  different density profiles. Motivated by the above results for the
  expected range of gas densities we take the density at $10^{18}$ cm,
  to be $n_{18}=2, 60,$ or 2000 cm$^{-3}$. We find bright radio
  emission at a few GHz across this entire range of densities, with
  the peak luminosity only weakly dependent on the chosen value of
  $n_{18}$.  For smaller densities the light curves peak earlier in
  time. Based on existing radio upper limits for tidal disruption
  event candidates, we show Swift J1644+57-like jet are absent in most TDEs
  as long as the density at $10^{18}$, $n_{18} \gsim  60$
  cm$^{-3}$. Prompt follow-up in the radio could provide tighter
  constraints on the existence of TDE jets.  
\end{enumerate}

\appendix
\section{Core Profile}
\label{app:core}
In $\S\ref{sec:profileComp}$ we compare the results of radio
light curves from jets propagating in core and cusp like gas density
profiles. See Fig.~\ref{fig:profiles} for a comparison of core and
cusp-like profiles. 

We use the following analytic expression to approximate the core
galaxy profile

\begin{align}
\begin{cases}
n=n(r_s) k(x) & 0.4 \leq x\leq 2.0\\
n = 2.0 n(r_s) (x/0.4)^{-0.95} & x < 0.4\\
n = 0.75 n(r_s) (x/2.0)^{-0.26} & x>2.\\
\end{cases}
\label{eq:cores}
\end{align}

Where, 

\begin{align}
  &x=r/r_s\\\nonumber
  &k(x)=\frac{45}{19} \frac{1}{x^{3/2}} \frac{1-x^{1.9}}{9-19
      x\frac{x^{0.9}-1}{x^{1.9}-1}}
\end{align}

To isolate the effects of the shape of the density profile, consider a
core density profile with $r_s=10^{18}$ cm, and $n_{18}=2000$: the
same as our high density cusp model.

\clearpage



\clearpage
  \footnotesize{
    \bibliographystyle{mnras}
    \bibliography{master}
  }

\end{document}
