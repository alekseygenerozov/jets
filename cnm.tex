\documentclass[usenatbib,fleqn]{mnras}

\makeatletter
\newlength{\abovecaptionskip}%
\setlength{\abovecaptionskip}{10\p@}
\makeatother


\usepackage{threeparttable}
 

\usepackage{amsmath,amssymb}
\usepackage{cases}
\usepackage{mathrsfs}
\usepackage{graphicx}
\usepackage{epstopdf}
%\usepackage{hyperref}
\epstopdfsetup{outdir=./figures/}
\graphicspath{{./figures/}}
\usepackage{url}
%\usepackage{aas_macros}

\newcommand\lsim{\mathrel{\rlap{\lower4pt\hbox{\hskip1pt$\sim$}}
    \raise1pt\hbox{$<$}}}
\newcommand\gsim{\mathrel{\rlap{\lower4pt\hbox{\hskip1pt$\sim$}}
    \raise1pt\hbox{$>$}}}


\newcommand       \be          {\begin{eqnarray}}
\newcommand       \ee          {\end{eqnarray}}
\newcommand{\Mbh}[1][]{M_{\bullet}}
\newcommand{\Menc}{M_{\rm enc}}
\renewcommand{\th}{t_h}
\newcommand{\Msun}{{\rm M_\odot}}
\newcommand{\pyear}{{\rm yr}^{-1}}
\newcommand{\rs}{r_s}

% write title (with email and institute)
\title[Influence of CNM on TDE radio emission]{The influence of
  cicumnuclear environment on the radio emission from TDE jets}
\author[Generozov et al.]{ A. Generozov$^{1}$, P. Mimica$^{2}$,
  B. D. Metzger$^{1}$, D. Giannios$^{3}$, N. Stone$^{1}$,
  M.A. Aloy$^{2}$
  \\
  $^{1}$Columbia Astrophysics Laboratory, Columbia University, 550 West 120th Street, New York, NY 10027\\
  $^{2}$Departamento de Astronomia y Astrofisica, Universidad de Valencia, E-46100 Burjassot, Spain\\
  $^{3}$Department of Physics and Astronomy, Purdue University, 525
  Northwestern Avenue, West Lafayette, IN 47907, USA}

\begin{document}
\maketitle
\begin{abstract}
  There are now dozens of candidates for tidal disruptions of stars by
  supermassive black holes (TDEs) at optical and x-ray wavelengths. A
  small fraction of these events, (e.g. {\it Swift} J1644+57) have
  radio synchrotron emission consistent with a powerful, relativistic
  jet shocking surrounding gas. The low detection rate of such events
  may mean that powerful jets are intrinsically rare in TDEs. However,
  it could also mean that typical nuclear gas densities are
  unfavorable for producing observable radio emission. We explore this
  possibility, constraining the range of gas densities which could be
  encountered by a TDE jet. We then calculate radio light curves for
  jets across the expected range of gas densities ($\sim$ 0.5-2000
  cm$^{-3}$ at $10^{18}$ cm). We find bright radio transients across
  this range of density profiles. Existing radio upper limits are
  often taken decades after the observed flare, well after the
  expected peak of the light curve for low density
  environments. Nonetheless they exclude powerful, {\it Swift}
  J1644+57-like jets. More stringent constraints would be possible
  with prompt follow-up of tidal disruption event candidates and would
  inform our understanding of the conditions necessary to launch jets.
\end{abstract}
\section{Introduction}
\label{sec:intro}
When a star in a galactic nucleus is deflected too close to the
central supermassive black hole (BH), it can be torn apart by tidal
forces.  During this tidal disruption event (TDE), roughly half of the
stellar debris remains bound to the BH, while the other half is flung
outwards and unbound from the system.  The bound material, following a
potentially complex process of debris circularization
(\citealt{Guillochon+2013},\citealt{Hayasaki+2013},\citealt{Hayasaki+2015},\citealt{Shiokawa+2015},\citealt{Bonnerot+2015}),
accretes onto the BH, creating a luminous flare lasting months to
years \citep{Hills1975, Carter+1982, Rees1988}.

Many TDE flares have now been identified at optical/ultraviolet (UV)
\citep{Gezari+2008, Gezari+2009, van-Velzen+2011, Gezari+2012,
  Arcavi+2014, Chornock+2014, Holoien+2014, Vinko+2015, Holoien+2016}
and soft x-ray wavelengths \citep{Bade+1996, Grupe+1999,
  Komossa&Greiner1999, Greiner+2000, Esquej+2007, Maksym+2010,
  Saxton+2012}. Beginning with the discovery of {\it Swift} J1644+57
(hereafter SwJ1644) in 2011, three additional TDEs have been
discovered by their hard x-ray emission (\citealt{Bloom+2011,
  Levan+2011, Burrows+2011, Zauderer+2011, Cenko+2012, Pasham+2015,
  Brown+2015}).  Unlike the optical/UV and soft X-ray TDEs, these
events are the result of non-thermal emission from a transient
relativistic jet beamed along our line of sight, similar to the blazar
geometry of active galactic nuclei (AGN).  In addition to their highly
variable X-ray emission, which likely originates from the base of the
jet, these events are characterized by radio synchrotron emission.
The latter, more slowly evolving, is powered by shocks formed at the
interface between the jet and surrounding circumnuclear medium (CNM)
\citep{Bloom+2011,Giannios&Metzger2011,Metzger+2012,De-Colle+2012,Mimica+2015},
analagous to the afterglow of a gamma-ray burst.

Although a handful of jetted TDE flares have been observed, their
volumetric rate is a very small fraction $\sim 10^{-5}-10^{-4}$ of the
observed TDE flare rate (e.g., \citealt{Burrows+2011},
\citealt{Brown+2015}), and an even smaller fraction of the
theoretically predicted TDE rate
\citep{Wang&Merritt2004,Stone&Metzger2016}.  One potential explanation
for this discrepancy is that the majority of TDEs produce powerful
jets, but their hard X-ray emission is relativistically beamed into a
small angle $\theta_{\rm b}$ by the bulk motion of the jet, making them visible to only a small
fraction of observers.  However, the inferred beaming fraction $f_b
\approx \theta_{b}^{2}/2 \sim 10^{-5}-10^{-4}$ would require
$\theta_{\rm b} \sim 0.01$ and hence a jet with a bulk Lorentz factor
of $\Gamma \gtrsim 1/\theta_j \sim 100$, much higher than typically
inferred for AGN jets or inferred for SwJ1644
(\citealt{Metzger+2012}).

The low detection rate of hard X-ray TDEs may instead indicate that
jets are intrinsically rare in these events, or that the conditions of the
surrounding environment are unfavorable for producing bright emission.
Jets could be rare if they require, for instance, a highly
super-Eddington accretion rate (\citealt{De-Colle+2012}), a TDE from a
deeply plunging stellar orbit (\citealt{Metzger&Stone2015}), or a
particularly strong magnetic flux threading the star
(\citealt{Tchekhovskoy+2014,Kelley+2014}).  Alternatively, jet formation
or its X-ray emission could be suppressed if the disk undergoes
Lens-Thirring precession due to a misalignment between the angular
momentum of the BH and that of the disrupted star
(\citealt{Stone&Loeb2012}).  In the latter case, however, even a `dirty'
jet could still be produced, which emits luminous non-thermal radio
synchrotron emission as it interacts the surrounding CNM.  The radio
emission from an off-axis is predicted to be relatively isotropic
\citep{Giannios&Metzger2011,Mimica+2015}.

\citet{Bower+2013} and \citet{van-Velzen+2013} performed follow-up
radio observations of optical/UV and soft X-ray TDE flares timescales
of months to decades after the outburst (see Table 1 of
\citealt{Mimica+2015} for a compilation). They find only one dim radio
afterglow definitively associated with the host galaxy of a strong TDE
candidate (CSS100217).\footnote{There were radio detections for two
  ROSAT flares: RX J1420.4+5334 and IC 3599. However, for RX
  J1420.4+5334 the radio emission was observed in a different galaxy
  than was originally associated with the flare.  IC 3599 has shown
  multiple outbursts in the recent years, calling into question
  whether it is a true TDE at all \citep{Campana+2015}.}
\citet{Arcavi+2014} followed up the PTF-09axc flare with the flare,
finding no radio emission five years after the
event. \citet{Bower+2013} and \citet{van-Velzen+2013} use a simple
model for the radio emission as a Sedov blast wave, to conclude that
$\lesssim 10\%$ of TDEs produce jetted TDE emission at a level similar
to that produced in SwJ1644.  \citet{Mimica+2015} used 2D
hydrodynamical models, coupled with a synchrotron radiation transport
calculation, to model the on-axis emission from SwJ1644. They then
extended this model to off-axis viewing angles, and concluded that
most previous TDEs should have been detected if their jets were as
powerful.

Recently, the TDE flare (ASSASN-14li; \citealt{Holoien+2016a}) was
observed to be accompanied by transient radio emission, consistent
with either a weak relativistic jet \citep{van-Velzen+2015} or a
sub-relativistic outflow \citep{Alexander+2015} of total energy $\sim
10^{48}-10^{49}$ erg.  ASSASN-14li occurred in a nearby galaxy at 90
Mpc, approximately 10 times closer than the majority of previous
flares.  If other TDE candidates launched a similar outflow, their
radio afterglows would fall below existing upper limits.  One question
that naturally arises is whether the lack of detected radio emission
in other thermal TDEs is the result of (i) a sufficiently weak
outflows/jet, or (ii) a nuclear gas environment unfavorable for
producing luminous radio emission.

Previous works (\citealt{Bower+2013}; \citealt{van-Velzen+2013};
\citealt{Mimica+2015}) have generally assumed that all TDEs occur in a
similar environment as SwJ1644.  However, in general the
gas density in a galactic nucleus depends sensitively on the sources
of gas from stellar winds, and the sources of gas heating
(\citealt{Quataert2004,Generozov+2015}).  In a normal gamma-ray burst, the
environment the jet emerges from is relatively well-understood to be
the wind of the massive progenitor star, or the ISM of the host
galaxy.  However, the density of the CNM encountered by a TDE jet
could in principle be orders of magnitude higher.

In this paper we address the range of gas densities encountered by
jetted TDEs and estimate their synchrotron emission by means of
analytic calculations and numerical simulations.  We find that radio
emission should be detected for all plausible CNM densities, leading
to the robust conclusion that most TDEs do not launch SwJ1644-like
jets.  This result has significant implications for the physics of jet
launching in TDEs and other accretion flows.

In $\S\ref{sec:cnm}$ we use the formalism developed in
\citet{Generozov+2015} (hereafter GSM15) to calculate the CNM profiles
encountered by TDE jets for different assumptions about the stellar
population in the galactic nucleus.  A younger stellar population
produces significant wind mass loss from O and B stars. In contrast,
the rate of wind mass loss from an older population, which lacks these
massive stars, could be a $\sim$couple orders of magnitude smaller.
We show that the requirement of a physical stellar population limits
the gas density on a scale of $10^{18}$ cm to a range of $n_{18} \sim
0.5-2000$ cm$^{-3}$. We also discuss observations of the gas density
distribution in the Galactic Center. 

As a second component of this paper, in $\S\ref{sec:results}$ we
explore the expected synchrotron emission for different energy jets
across the allowed range of CNM conditions.  The jet propagation is
simulated using both 1D and 2D hydrodynamic simulations, using the
numerical setup described in $\S\ref{sec:numerical}$. The results are
post-processed to produce radio synchrotron light curves, presented in
$\S\ref{sec:numResults}$, for different jet energies, ambient gas
densities, and observer viewing angles. In $\S\ref{sec:param}$, we
show the dependence of the peak luminosity and peak time on physical
parameters can be understood with a simple, analytic blast wave
model. This model is summarized in $\S\ref{sec:analyt}$ and
Appendix~\ref{app:analyt}. Finally, we compare our light curve results
to existing radio upper limits for TDE candidates, and show that they
cannot have jets as powerful as Sw1644. We summarize our conclusions
in $\S\ref{sec:conc}$.

\section{Range of CNM Densities}
\label{sec:cnm}

In this section we place constraints on the gas densities in galactic
nuclei.  In $\S\ref{sec:analy}$, we determine the possible range of
densities resulting from mass loss by stellar winds. 

\subsection{Analytic Constraints on CNM Density}
\label{sec:analy}

\subsubsection{Preliminary Considerations}

The radio emission of a jet is primarily sensitive to the density of
ambient gas near the Sedov radius, outside of which the jet has
swept up a gaseous mass exceeding its own. For a power law gas density
profile, $n= n_{18} \left(r/r_{18}\right)^{-k}$,
\begin{equation}
  r_{\rm sed} = r_{18} \left( \frac{E(3-k)}{4\pi n_{0}
      m_{\rm p} c^2 r_{18}^3} \right)^{1/(3-k)} \approx 3 E_{54}^{1/2} n_{\rm 18}^{-1/2}\,{\rm pc}. 
  \label{eq:rdec}
\end{equation}
where $E = E_{54}10^{54}$ erg is the isotropic equivalent energy,
$r_{18} = r/10^{18}$ cm, and in the final equality we have taken $k =
1$.  For a powerful jet similar to SwJ1644, the deceleration radius is
typically of order a parsec.

An initially relativistic jet will slow to sub-relativistic speeds at
$r \sim r_{\rm sed}$.  However, significant deceleration sets in earlier, once the
jet has swept up only a fraction $\sim 1/\Gamma$ of its rest mass, as
occurs at the radius
\begin{equation}
  r_{\rm dec}=\frac{r_{\rm sed}}{\Gamma^{2/(3-k)}}.
  \label{eq:rdec2}
\end{equation}
%
The jet reaches
 the Sedov and deceleration radii, respectively, at
observer times 
\begin{equation}
t_{\rm sed} = \frac{c}{r_{\rm sed}} \approx
10 E_{54}^{-1/2}n_{18}^{1/2} {\rm year}
 \end{equation} 
\begin{equation}
t_{\rm dec} =
\frac{c}{2\Gamma^{2} r_{\rm dec}} = \frac{t_{\rm
    sed}}{2\Gamma^{3(4-k)/(3-k)}} = \frac{t_{\rm sed}}{2\Gamma^{3}},
 \label{eq:tdec}
\end{equation}
%
where in the final equality we have again taken $k = 1$.


\subsubsection{Dynamical Model of CNM}
\label{sec:model}

The dominant source of gas in the CNM of quiescent galaxies is winds
from stars in the galactic nucleus. We model the hot phase of the ISM
using the 1D spherical hydrodynamic equation with mass and energy
injection from stellar winds (e.g. \citealt{Holzer+1970};
\citealt{Quataert2004})
\begin{align}
  &\frac{\partial \rho}{\partial t}+\frac{1}{r^2}\frac{\partial}{\partial r}\left(\rho r^2 v\right)=q \label{eq:drhodt}\\
  &\rho \left(\frac{\partial v}{\partial t} + v\frac{\partial
      v}{\partial r}\right) =-\frac{\partial p}{\partial r}- \rho\frac{GM_{\rm enc}}{r^{2}} -q v \label{eq:dvdt}\\
  &\rho T\left(\frac{\partial s}{\partial t} + v\frac{\partial
      s}{\partial
      r}\right)=q\left[\frac{v^2}{2}+\frac{v_w^2}{2}-\frac{\gamma}{\gamma-1}
    \frac{p}{\rho} \right] ,
\label{eq:model}
\end{align}
where $\rho = \mu m_p n$, $v$, $p$, and $s$ are the density, velocity,
pressure (assumed to be an ideal gas with $\mu = 0.62$), and specific
entropy, respectively.  The enclosed mass $\Menc = M_{\bullet} +
M_{\star}$ includes both the black hole mass $M_{\bullet}$ and
enclosed stellar mass $M_{\star} \propto \int \rho_{\star}r^{2}dr$,
where $\rho_{\star}$ is the stellar density. At the radius of the
sphere of influence, $r_{\rm inf}$, the enclosed stellar and black
masses are equal, $M_{\star}(r_{\rm inf})=\Mbh$.  We take $r_{\rm
  inf}=3.5 \Mbh[,7]^{0.6}$ (GSM15), where $\Mbh[,7]=\Mbh/10^7 \Msun$.

The source term $q =\eta \rho_\star/\th$ is the mass injection
rate per unit volume per unit time. The energy source term $\propto
v_w^{2}$, parameterizes the heating rate of the gas, as physically
results from stellar wind kinetic energy, supernovae, and black hole
feedback.

GSM15 present analytic approximations for the
densities and temperatures of steady state solutions to
equation~\eqref{eq:model}. We apply these results across the
physically allowed range heating rates ($v_w$) and mass injection
rates ($\eta$), and obtain the corresponding range of gas densities.

\subsubsection{Stellar density profiles}
We assume a broken power law for the stellar density profile,
$\rho_{\star}$, motivated by Hubble measurements of the radial surface
brightness profiles for hundreds of nearby early type galaxies
\citep{Lauer+2007}.  The measured profile is well fit by the so-called
``Nuker'' law parameterization, i.e.~a piece-wise power law that smoothly
transitions from an inner power law slope, $\Gamma$, to an outer power
law slope, $\beta$, at a break radius, $r_b$.

Most galaxies have $0<\Gamma<1$, and are classified into two broad
categories: ``core" galaxies with $\Gamma<0.3$ and ``cusp" galaxies with
$\Gamma>0.5$. Assuming spherical symmetry and a constant mass-to-light
ratio, the inner stellar profile translates to a stellar density of
$\rho_\star\propto r^{-1-\Gamma}=r^{-\delta}$. Core galaxies have
$1<\delta<1.3$, while cusp galaxies have $1.5<\delta<2$.

Cusp-like stellar density profiles, are the most relevant to TDEs for
two reasons.  First, only a low mass black hole ($\Mbh\lsim 10^8
\Msun$) can disrupt a main sequence star, and low mass low mass black
holes are more commonly characterized by cusp-like profiles.
Additionally, as described in \citet{Stone&Metzger2016} a cuspy
stellar density profile results in a higher TDE rate per galaxy.  We
adopt a fiducial value of $\Gamma=0.7$ ($\delta=1.7$), motivated by
the rate-weighted average value of the inner stellar density profile
for the galaxies in \citet{Stone&Metzger2016} (their Table C).


\subsubsection{Gas density profiles}


Given sufficiently strong heating, a one-dimensional steady-state
model for the CNM is characterized by an inflow-outflow structure.
The velocity passes through zero at the ``stagnation radius'', $\rs$.
Mass loss from stars interior to the stagnation radius is accreted,
while that outside of $\rs$ is unbound in an outflow from the nucleus.
Fig.~\ref{fig:profiles} shows example radial profiles of the
steady-state gas density calculated for a core and a cusp stellar
density profile. The stagnation radius is marked as a blue dot on each
profile.

As described in GSM15, the stagnation radius is approximately given by
\begin{equation}
r_s \simeq f(\delta) \frac{G M}{v_w^2},
\label{eq:rs}
\end{equation}
%
where $f(\delta)$ is a constant of order unity that depends on the
slope of the stellar density profile $\delta$.  The gas density at the
stagnation radius, $n(\rs)$, is determined by the rate at which
stellar winds inject mass interior to this,
\begin{equation}
  \dot{M}=\frac{\eta M_{\rm \star}(\rs)}{t_h} \approx  2.8 \times 10^{-6} \Mbh[,7]^{0.22} \eta_{0.02} \left(\frac{r_s}{\rm
      pc}\right)^{1.3} \Msun \, {\rm yr}^{-1},
\label{eq:dotM}
\end{equation}
%
where $M_{\star}(\rs)$ is the total stellar mass enclosed within the
stagnation radius and the second equality was derived for our fiducial
value of $\delta=1.7$, where $\eta_{0.02}=\eta/0.02$ is normalized to
a fiducial value characteristic of an old stellar population.

The density at the stagnation radius, $n(\rs)$, is estimated by
equating the gas injected by stellar winds over a dynamical time at
the stagnation radius, $t_{\rm dyn} (\rs)=(\rs^3/G \Mbh)^{1/2}$, to
the gas mass enclosed at this location.  This gives
\begin{align}
  &\frac{4 \pi}{3} \rs^3 \mu m_p n(r_s) \simeq \dot{M} t_{\rm dyn}
  (\rs) \nonumber\\
  &\Rightarrow n(r_s) \simeq 0.21 \eta_{0.02} \Mbh[,7]^{-0.28} \left(\frac{r_s}{\rm
      pc}\right)^{-0.2} {\rm cm}^{-3}.
\label{eq:nrs}
\end{align}
%
Substituting equation~\eqref{eq:rs} for $r_s$, we obtain 
\begin{equation}
n(r_s) \simeq 0.2 \, v_{500}^{0.4} \eta_{0.02} \Mbh[,7]^{-0.48} {\rm cm}^{-3},
\label{eq:nrs2}
\end{equation}
where $v_{500}=v_w/\left(500 \,\mathrm{km\,\,
    s^{-1}}\right)$. 
%
Near the stagnation radius, GSM15 found that the radial gas profile
has a power-law slope of $k \approx (4\delta-1)/6$, which for our
fiducial value of $\delta$=1.7 gives $n \propto r^{-1}$. Furthermore,
the density profile remains close to $n\propto r^{-1}$ for small radii
of interest.\footnote{For cusp galaxies, the gas velocity asymptotes
  to a constant value between the stagnation radius and break radius
  (provided they are sufficiently far apart). In this region we
  instead expect $n\propto r^{1-\delta}$.}  Thus, we adopt $n(r)=
n_{18} (r/10^{18} {\rm cm})^{-1}$ as our fiducial density profile,
where $n_{18}$ is the density at $r = 10^{18}$ cm.  In
Appendix~\ref{app:core} we explore a core-like density profile, to
which we compare our results for the jetted radio emission to the
fiducial cusp case in $\S\ref{sec:profileComp}$.

Combining our assumed $n\propto r^{-1}$ profile with
equation~\eqref{eq:nrs}, we obtain
\begin{equation}
  n_{18}\simeq 0.6 \left(\frac{r_s}{\rm pc}\right)^{0.8}
  \Mbh[,7]^{-0.28} \eta_{0.02} {\rm cm^{-3}},
  \label{eq:n18}
\end{equation}
%
Substituting the stagnation radius (eq.~\ref{eq:rs}) into this
expression gives
\begin{equation}
  n_{18}\simeq 0.3 \Mbh[,7]^{0.52} v_{500}^{-1.6} \eta_{0.02} {\rm
    cm^{-3}}.
\label{eq:n182}
\end{equation} 
%
Although the gas density steepens near the break radius $r_b$ of the
stellar density profile (Fig.~\ref{fig:profiles}), this will only
affect the radio emission near its peak if $r_b$ lies inside of the
Sedov radius (eq.~\ref{eq:rdec}).  Figure \ref{fig:profiles} shows the
Sedov radius for different energy jets and the gas density profiles
there.  The measured break radii of all but four the galaxies in
\citet{Lauer+2007} exceeds 10 parsecs.  This is well outside of the
Sedov radius, even for a very energetic jet, with an isotropic
equivalent energy of $E=4\times 10^{54}$ erg in a CNM of particularly
low density, $n_{18} \sim 1$ cm$^{-3}$.

The presence of nuclear star
cluster in the galactic center could produce another break in the
stellar density profile near the outer edge of the NSC, which is
typically located at $\sim 1-5$ pc \citep{Georgiev+2014}.  However,
even a break radius of 1 pc will reside inside the Sedov radius,
except for the combination of a powerful jet and relatively small CNM
density, $n_{18}<30$ cm$^{-3}$.  Henceforth we neglect the effects of
a break in the stellar density profile in our analysis.


\begin{figure}
\includegraphics[width=8cm]{sedov_radius.pdf}
\caption{\label{fig:profiles} Steady-state radial profiles of the CNM
  gas density, $n(r)/n_{18}$, normalized to the value $n_{18} = n(r =
  10^{18}$ cm), calculated for a black hole mass $10^{7} \,\Msun$ and
  gas heating parameter $v_w=600$ km s$^{-1}$.  Cusp and core stellar
  density profiles are shown with solid and dashed lines,
  respectively.  The line color denotes the isotropic equivalent
  energy of a jet for which the Sedov radius equals that radius from
  the black hole (given the swept up interior gaseous mass).}
\end{figure}



\subsubsection{Plausible Density Range}
\label{sec:densAllowed}

In this section, we estimate the plausible range in the normalization
of the CNM gas profile, $n_{18}$.  We assume that star formation
occurs in two bursts, an old burst of age comparable to the Hubble
time $t_{\rm h} = 10^{10}$ yr, and a ``young'' burst of variable age
$t_{\rm burst} \ll t_{\rm h}$ which contributes a fraction $f_{\rm
  burst}$.
%Both the young and old stars have the
%same (cuspy) density profile, which implies that the gas density
%profile goes as $r^{-1}$.

For a sufficiently large burst occurring less than 40 Myr ago, gas
heating is dominated by the energetic winds of massive star
winds.\footnote{Although Type II supernovae are also an important heating
  source which dominates stellar winds after $\sim$6 Myr
  \citep{Voss+2009}, they are intermittent and hence their
  contribution to the gas heating is neglected for simplicity.}  In
this case, the mass return ($\eta$) and heating parameters ($v_w$) are
calculated as described in GSM15 (their Appendix C).  Given
$\eta(t_{\rm burst},f_{\rm burst})$ and $v_w(t_{\rm burst},f_{\rm
  burst})$, we calculate $n_{18}$ following equation~\eqref{eq:n182}.

For older stellar populations, gas heating may come from a few
different sources, including Type Ia Supernovae (SNe)\footnote{Unbound
  debris streams from TDEs provide another source of heating localized
  in the galactic center (\citealt{Guillochon+2015a}), which we
  neglect.} and AGN feedback.  We focus on
quiescent phases, during which SNe Ia dominate.  As discussed in
GSM15, SNe Ia clear out the gas external to a critical radius, $r_{\rm
  Ia}$, where the interval between successive Ia SNe equals the
dynamical (gas inflow) timescale.  For an old stellar population,
$n_{18}$ is estimated by equating $r_{\rm Ia}$ with the stagnation
radius in equation~\eqref{eq:n18}.  The Ia radius is calculated as
described in GSM15 at times $t>300 \,{\rm Myr}$ after star formation,
and is taken to be constant for $t = 40-300$ Myr.\footnote{GSM15
  incorrectly extrapolated the Ia rate valid at times $t>300 \,{\rm
    Myr}$ back to a time $t = 3$ Myr, which unphysical as no white
  dwarfs would have formed by this time.  Although its qualitative
  impact on our results is minimal, here we instead take the Ia rate
  to be 0 for $ t < 40$ Myr.}

%\footnote{The Ia rate is given
 % by $8.8 \times 10^{-13} \left(\frac{t}{3\times 10^{8} {\rm
  %      yr}}\right)^{-1.12} \, \Msun^{-1} \, {\rm yr}^{-1} $ for
  %$t>3\times 10^8$ years.} =
%In GSM15 we incorrectly extrapolated this rate all of the way to 3
%Myr, which is unphysical because white dwarfs have not yet formed. Here, we instead take the Ia rate to be zero for times less than 40 Myr and constant from 40$-$300 Myr.

Fig.~\ref{fig:param} shows how $n_{18}$ varies with the young
starburst properties, $f_{\rm burst}$ and $t_{\rm burst}$.  We find a
maximum density of $n_{18} \sim 3000$ cm$^{-3}$ is achieved for a
burst of age $t_{\rm burst} \sim 4$ Myr which forms most of the stars
in the nucleus $f_{\rm burst} \sim 1$.  In this case, both the energy
and mass budgets of the CNM are dominated by fast winds from massive
stars.  Although a large gas density is present in the immediate
aftermath of a starburst, its magnitude will decline with the wind
mass loss rate, approximately $\propto t^{-3}$.  Therefore, the gas
density would decline by an order of magnitude from this maximum
allowed value after just a few Myr.

By contrast, the lowest allowed density $\sim 0.03$ cm$^{-3}$ is
achieved for a relatively modest burst of young stars $t_{\rm burst}
\approx 10^{6}$ Myr, which forms a fraction $f_{\rm burst} = 4\times
10^{-4}$ of the total stellar mass. In this case the young massive
stars provide a high heating rate, while the mass injection rate is
comparatively low and receives contributions from both young and old
stars.

Our procedure may underestimate the value of $n_{18}$ somewhat, as we
do not include the effects of discreteness on the assumed stellar
population.  In particular, we assume that massive stars provide a
spatially homogeneous heating source, even on small scales where the
number of massive stars present may be very small.  The doubly hatched
region in Fig.~\ref{fig:param} denotes the region where less than one
massive star ($\gsim 15 \Msun$) is on average present inside of the
nominal stagnation radius (eq.~\ref{eq:rs}).  Discreteness effects are
thus important for relatively small bursts of star formation,
including the case described above which gives the minimum $n_{18}$.
If we instead equate the stagnation radius to the radius enclosing a
single star of mass $\gsim 15 \Msun$, we find a larger value of
$n_{18}\sim 0.5$ cm$^{-3}$.  From this we conclude that the true
minimum density likely resides between these extremes, min($n_{18})
\sim 0.03-0.5$ cm$^{-3}$.

Most of the host galaxies of observed TDE flares show evidence of some
star formation within the last Gyr \citep{French+2016}.  In this
region of parameter space corresponding to the right side of
Fig.~\ref{fig:param}, gas heating rate is dominated by SN Ia and we
expect $n_{18}\sim 10-100$ cm$^{-3}$.
%{\bf AG note that the
 % entire stellar pop likely did not form 1 Gyr ago. In fact the
  %observation merely indicate that there was some star formation w. in
 % the last Gyr}

To summarize, we expect the CNM densities of quiescent galaxies on
parsec scales to vary from min($n_{18}) \sim 0.03$ to max($n_{18})\sim
10^{3}$ cm$^{-3}$, with a typical value of perhaps $n_{18}\sim 100$ cm$^{-3}$
in the host galaxies of observed TDEs.

\subsubsection{Mass drop-out from star formation?}

Our model for the CNM predicts the total gas density sourced by
stellar winds, including both hot and cold phases.  In the first few
Myr after a starburst, the injected stellar wind material is hot
($T\gsim 10^{7}$ K) due to the thermalized wind kinetic energy.
Later, SNe Ia provide intermittent heating, but the stellar wind
material that accumulates on small scales, between successive SNe Ia
may become much cooler, with temperature $\lesssim 10^{4}$ K, and
would condense into clumps.

The radio emission from a TDE jet is expcted to be similar for a
clumpy and a smooth medium with the same average density
(e.g.~\citealt{Nakar&Granot2007,van-Eerten+2009,Mimica&Giannios2011})
{\bf AG these references show that density jumps don't create sharp
  features for a spherical, relativistic blast wave but they don't
  really address the Newtonian regime}.  However, cold gas may also
condense into stars.  To estimate the star formation rate, we assume
that the CNM self-regulates itself to a condition of marginal thermal
stability.  This we define as the cooling time being ten times longer
than the dynamical timescale (e.g.~\citealt{McCourt+09}). For a
$\sim$1 Gyr old stellar population, we find that this requires a star
formation rate of $5.3 \times 10^{-4} \,\Msun$ yr$^{-1}$, which
implies that only $\sim 40$\% of the gas injected on small scales
would be turned into stars.  The gas density would therfore be only a
few times lower than in our estimates which neglect mass drop out from
star formation.

\begin{figure} 
  \includegraphics[width=8cm]{cnm_plot.pdf}
  \caption{\label{fig:param} Isocontours of $n_{18}$, the CNM density
    at $r = 10^{18}$ cm (blue lines), as a function of the stellar
    population in the galactic nucleus.  The star formation is
    parameterized assuming that a fraction $f_{\rm burst}$ of the
    stars form in a burst of age $t_{\rm burst}$, while the remaining
    stars formed a Hubble time ago.  We have assumed a black hole of
    mass $10^{7} \, \Msun$ and that both young and old stars possess a
    cusp-like density profile, which produces a gas density profile
    that goes as $r^{-1}$.  Hatched areas indicate regions of
    parameter space where massive stars ($\gsim 15 \, \Msun$) dominate
    the gas heating rate, but less than one (doubly hatched) or less
    than ten (singly hatched) massive stars are present on average
    inside the nominal stagnation radius (eq.~\ref{eq:rs}).  In these
    regions of parameter space discreteness effects not captured by
    our formalism may be important. The red line shows the rough location of
    the Galactic Center in this parameter space.}
\end{figure}


% \begin{figure}
%   \includegraphics[width=8cm]{cnm_plot_2.pdf}
%   \caption{\label{fig:param2} $n_{18}$ as a function of the mass
%     return ($\eta$) and heating ($v_w$) parameters for a $10^7 \Msun$
%     black hole. We translate the gray line from Fig.~\ref{fig:param}
%     to this parameter space. $f_{\rm burst}$ increased in the direction
%     indicated by the gray arrow. As indicated by the black solid line,
%     the area below the black solid line is thermally unstable
%     (\textit{see text for discussion}).}
% \end{figure}

\subsection{Empirical Constraints on CNM Density}
\label{sec:empirical}

\subsubsection{Galactic center} 
Due to its close proximity, it is possible to directly observe the gas
density distribution on parsec scales in the Galactic Center
(GC). \citet{Baganoff+2003} find that the hot, diffuse plasma within
10 arcseconds ($\sim 10^{18}$ cm) of Sgr A* has a root mean square
electron density of $\sim 26$ cm$^{-3}$, which implies a number
density of $n\sim 50$ cm$^{-3}$. However, this estimate includes
regions closer to the center, where the gas density is higher, so that
the true number density at $10^{18}$ cm may be smaller by a factor of
$\sim 2$.

In Fig.~\ref{fig:param} we show the range of two-burst star formation
models which produce heating and mass return parameters equal to those
derived from the full star formation history of the GC from
\citet{Pfuhl+2011} (their Fig.~14).  Our formalism gives values of
$n_{18}\sim 4-6$ cm$^{-3}$, which are too low compared to
observations.  Discrepency at this level is not surprising because our
model in spherically symmetric, while in reality many of the massive
stars in the GC are concentrated in two counter-rotating disks
\citep{Genzel+2003} with a top heavy IMF \citep{Bartko+2011}.  The
disk stars extend from $\sim 10^{17}-10^{18}$ cm and inject $\sim
10^{-3} \Msun$ yr$^{-1}$ of stellar wind material, much more than the
$\sim 4 \times 10^{-5} \Msun$ yr$^{-1}$ expected for the global star
formation history, explaining the large density of hot gas.

In short, accurate modeling of the gas distribution in a particular
galactic nucleus, requires detailed knowledge of the distribution of
stars. Our goal here is to bracket the range of possible nuclear gas
densities, by considering a broad range of stellar populations.

% In addition, to the hot phase the galactic center also contains a cold
% circumnuclear disk with density $10^{5}$ cm$^{-3}$
% (e.g. \citealt{Becklin+1982, Morris+1999, Lau+2013}), which subtends a
% solid angle of...(CITE). However, insofar as the angle of the TDE jet
% is large compared to such features, the quantitative impact on the
% radio flux is unlikely to be large.

% {\bf AG warm phase gas? Freire+2012}

%  In part, this is due to our neglect of discreteness
% effects: the nominal stagnation radius in our model is $\sim 6\times
% 10^{16}$ cm. This is smaller than the inner radius of the disks of
% young stars in the galactic center which extend from $\sim
% 10^{17}-10^{18}$ cm \citep{Genzel+2013} and dominate the mass
% budget. Previous studies which use more realistic distributions of
% stars (e.g. \citealt{Quataert 2004, Cuadra+2006}) larger densities
% {\bf AG how much detail do I need to go into about Cuadra? Still not
%   happy with the presentation here.} 




% \subsubsection{Eddington ratio distribution}
% We now translate observed constraints on the accretion
% rate distribution of supermassive black holes into constraints on the
% CNM density.  The accretion rate can be written as
% \begin{equation}
% \dot{M}_{\bullet} = f_{\rm in} 4 \pi r^2 \mu m_p n v,
% \label{eq:mdot}
% \end{equation}
% where $n$ is the average density at radius $r$ and $f_{\rm in}$ is the
% fraction of the large scale inflow which actually reaches small scales
% and accretes onto the black hole.  Inside the sonic point the velocity
% approaches the free-fall velocity, in which case
% equation~\eqref{eq:mdot} becomes
% \begin{equation}
%   \dot{M}_{\bullet} = 1.1\times 10^{-5} \Mbh[,7]^{0.5} f_{\rm in}
%   n_{18} \,\,\Msun {\rm yr}^{-1}.
% \label{eq:mdot2}
% \end{equation}
% %
% The corresponding Eddington ratio is
% \begin{equation}
%   \lambda\equiv L/L_{\rm Edd} = 4.8 \times 10^{-5}
%   \left(\frac{\epsilon_{\rm rad}}{0.1}\right) \Mbh[,7]^{0.5} f_{\rm in}
%   n_{18},
% \label{eq:n18Edd}
% \end{equation}
% %
% where $\epsilon_{\rm rad}$ is the radiative
% efficiency. \citet{Kauffmann&Heckman2009} present distributions of the
% OIII line luminosity, ${\rm L[OIII]}/\Mbh$ for a volume limited {\bf
%   AG double check} sample of SDSS galaxies, where ${\rm
%   L[OIII]}/\Mbh=1.7$ roughly corresponds to Eddington ratio of unity
% (REF).  This bolometric correction maps the distribution of ${\rm
%   L[OIII]}/\Mbh$ to a distribution of Eddington ratios
% \citep{Kauffmann&Heckman2009}.

% Equation (\ref{eq:mdot2}) provides a map between the Eddington ratio
% and gas density $n_{18}$, provided that the radiative efficiency
% $\epsilon_{\rm rad}$ and accretion efficiency $f_{\rm in}$ are known.
% For the former we adopt the MHD shearing box results of
% \citet{Sharma+2007}, who find (their Fig.~6)
% \begin{align}
% &\epsilon_{\rm rad} \simeq 
% \begin{cases}
%   0.03 \left(\frac{\dot{M}_{\bullet}}{10^{-4}\dot{M}_{\rm Edd}}\right)^{0.9} & \frac{\dot{M}_{\bullet}}{\dot{M}_{\rm edd}} \lsim 10^{-4} \\
%  0.03 &  10^{-2} \gsim \frac{\dot{M}_{\bullet}}{\dot{M}_{\rm edd}}
%  \gsim  10^{-4},
% \end{cases}
% \label{eq:efficiency}
% \end{align}
% where
% \begin{equation} 
%  \dot{M}_{\rm Edd} \equiv \frac{L_{\rm Edd}}{0.1 c^2} 
% \end{equation}
% %
% and
% there is a factor of $\sim$ 5 uncertainty in $\epsilon_{\rm rad}$ for
% $\dot{M}/\dot{M}_{\rm Edd}\lsim 10^{-4}$ due to the dependence of the
% results on uncertain microphysical parameters.  We linearly
% interpolate between equation~\eqref{eq:efficiency} for
% $\dot{M}/\dot{M}_{\rm Edd}<0.01$ and the standard thin-disk efficiency
% of $\epsilon_{\rm rad}=0.1$ for $\dot{M}/\dot{M}_{\rm Edd}>0.1$.

% \begin{figure}
% \includegraphics[width=8cm]{fcum_n18.pdf}
% \caption{\label{fig:n18Cum} Cumulative distribution of $f_{\rm in}
%   n_{18}$ for black holes with mass $\Mbh\simeq 10^{7} \Msun$ as
%   inferred based on the distribution of measured Eddington ratios from
%   \citet{Kauffmann&Heckman2009} {\it (solid black line)}. Here $f_{\rm
%     in}$ is the ratio of the true accretion rate onto the black hole
%   to the inflow rate of free-falling gas at $10^{18}$ cm.  A dashed
%   blue line shows the corresponding cumulative distribution of
%   $n_{18}$, obtained using values of $f_{\rm in}$ motivated by the
%   simulations of \citet{Li+2013} (see text). The distribution of
%   $f_{\rm in} n_{18}$ ($n_{18}$) is unreliable to the left of the
%   solid black (dashed blue) vertical line.  Such low densities
%   correspond to galaxies with small OIII luminosities, which could be
%   the result of star formation instead of AGN activity.}
% \end{figure}


% Fig.~\ref{fig:n18Cum} shows the distribution of $f_{\rm in} n_{18}$
% which results from combining the distributions of Eddington ratio from
% \citet{Kauffmann&Heckman2009} with equations~\eqref{eq:n18Edd} and
% ~\eqref{eq:efficiency}.  It would at first appear that the absence of
% galaxies with $f_{\rm in }n_{18} \lsim$ few cm$^{-3}$ places a lower
% bound on $n_{18}$ of a few cm$^{-3}$ because $f_{\rm in} \le 1$.
% However, because measurements of Eddington ratios below $\sim 10^{-3}$
% (shown with a vertical line in Fig.~\ref{fig:n18Cum}) are not reliable
% (T.~Heckman, private communication), this allows a significant fraction
% ($\sim 30\%$) of galaxies to have lower gas densities.

% To obtain the cumulative distribution of $n_{18}$ requires an
% additional prescription for $f_{\rm in}$.  We use the results of
% \citet{Li+2013}, who perform two-dimensional hydrodynamical
% simulations of axisymmetric rotating accretion flows.  They find that
% when the inflow rate on large scales is highly sub-Eddington
% ($\dot{M}/\dot{M}_{\rm Edd} \lsim 10^{-4}$), cooling is inefficient
% and $f_{\rm in}\sim 0.01$. On the other hand, when
% $\dot{M}/\dot{M}_{\rm Edd}\gsim 10^{-2}$, $f_{\rm in}$ approaches
% unity.  We use $\dot{M}_{\bullet}/\dot{M}_{\rm Bondi}$ in their Figure
% 6 for $f_{\rm in}$.\footnote{\citet{Li+2013} employ an alpha viscosity
%   prescription with $\alpha=0.01$.  A higher value of $\alpha$ would
%   likely increase the accretion fraction $f_{\rm in}$, thus decreasing
%   our estimates of $n_{18}$ in a systematic way.}  With this choice we
% find that only a third of nuclei have $n_{18}>2\times 10^{3}$
% cm$^{-3}$ and only 6\% have $n_{18}>10^{4}$ cm$^{-3}$.  We note,
% however, that existing candidate tidal disruption flares show little
% or no evidence for AGN emission lines (e.g. \citealt{van-Velzen+2011,
%   Arcavi+2014}), making them unlikely to reside in the high density
% tail of the $n_{18}$ distribution.

% %Two potential complications to keep in mind are (i) clumpiness of the
% %CNM and (ii) anisotropy. The distributions above are distributions of
% %the {\it average} $n_{18}$.  Most likely, some of the nuclear gas in a
% %low density hot phase, while the rest is in high density cold
% %clumps/filaments.  However, while the jet is relativistic the
% %light curve of a jet propagating through a clumpy medium will differ
% %little from that of a smooth medium with the same average
% %density. Even in the late stages when the jet becomes non-relativistic
% %clumps will only make a difference if the size of the clumps is
% %comparable to the size of the jet {\bf AG include some refs here}.
% One potential complication is that the CNM could be distributed
% anisotropically around the black hole.  The cold phase could be
% concentrated in a ring-like structure, similar to the cirumnuclear
% ring surrounding SgrA$^{*}$ (e.g.~\citealt{Lau+13}).  Anisotropies
% may also present in the hot phase, for instance due to low density
% bubbles inflated by the AGN.  However, insofar as the angle of the
% TDE jet is large compared to these features, the quantitative impact
% on the radio flux is unlikely to be large.

% % For instance, X-ray observations of the nuclei of massive
% % elliptical galaxies show densities of density on scales of $\sim
% % 100$ pc is $\gsim 0.1$ cm$^{-3}$

% % For instance, \citet{Russell+2013} use X-ray observations to
% % measure
% % gas density and temperature profiles for a sample of massive
% % elliptical galaxies. The measured electron density on scales of
% % $\sim
% % 100$ pc is $\gsim 0.1$ cm$^{-3}$. Note that the gas density at 100
% % pc
% % would be irrelevant for a TDE jet, but we would not expect the gas
% % density to be decreasing towards the galactic center in a steady
% % state.  We note that massive black holes in the
% % \citet{Russell+2013}
% % sample, with $\Mbh\sim 10^{9} \Msun$, whereas black holes tidal
% % disruption events would have $\Mbh\lsim 10^8 \Msun$. This is
% % because
% % more massive black holes would not be able to disrupt (main
% % sequence)
% % stars.

% %In this scenario, the
% %some fraction of jets would likely be stifled by the very dense
% %ring. However, such a ring would not block all jet propagation
% %directions.

% The empirical estimates in this section nicely complement our
% analytic results.  The accretion rates of black holes are
% particularly are particularly challenging to measure for quiescent
% nuclei.  However, such cases, in which AGN feedback is minimal, are
% precisely the regime where our analytic estimates are the most
% trustworthy, providing a lower density floor of $n_{18} \sim 0.5$
% cm$^{-3}$.

% On the other hand, the analytic estimates are specific to the hot
% phase of gas. The Eddington ratio distribution probes the average
% density (including any cold clumps). The distribution of Eddington
% ratios gives us confidence in our high density limit ($\sim 1000$
% cm$^{-3}$).



% We show a comparison of this modified 1D approach with the true 2D
% result in Figure~\ref{fig:1D2DB}, for $n_{18}=60$ and $n_{18}=2000$
% and $E_{\rm ISO}=4 \times 10^{54}$ ergs. For $n_{18}=2000$ the
% agreement is excellent, while for $n_{18}=60$ the 1D models still
% over-predict still over-predict the flux at times after the peak of
% the light curve.



\begin{table}
\begin{threeparttable}
  \caption{\label{tab:jetParams} Parameters for on-axis jet simulations.}
  \begin{tabular*}{0.95\columnwidth}{lll}
\hline
& Fiducial value & Other values \\
\hline\hline
    Fast component ($\Gamma=10$) &  &  \\ 
    \hline
    $[\theta_{\rm min}$, $\theta_{\rm max}]$ & [0, 0.1] radians & \\
    $E_{\rm ISO}/10^{54}$ erg & 4  & 0.04, 0.4\\
    $E/10^{54}$ erg & 0.02 & \\
    \hline 
    Slow component ($\Gamma=2$)\\
\hline
    $[\theta_{\rm min}$, $\theta_{\rm max}]$ & [0.1, $\pi/2$] radians
    & \\
    $E_{\rm ISO}/10^{54}$ erg & $4.7$ & 0.047, 0.47 \\
    $E/10^{54}$  erg & $0.47$ & \\
    \hline
    Microphysical parameters\\
\hline
    $\epsilon_e$ & 0.1 &  0.05, 0.2\\
    $\epsilon_b$ & 0.002 & 0.001, 0.005\\
    $p$ & 2.3\\
    \hline 
    Nuclear gas density \\
\hline
    $n_{18}$/cm$^{-3}$ & 60 & 2, 11, 345, 2000
  \end{tabular*}
% \begin{tablenotes}
% \item $^{\dagger}$  Additional values of physical parameters we tried.
% \end{tablenotes}
\end{threeparttable}
\end{table}


\section{Synchrotron Radio Emission}

\label{sec:results}
\subsection{Numerical Set-Up}
\label{sec:numerical}
We calculate the synchrotron radio emission from the jet-CNM shock
interaction across the physically plausible range of nuclear gas
densities.  We perform both one- and two-dimensional (axisymmetric)
relativistic hydrodynamical simulations.  The synchrotron emission is
computed as described in \citet{Mimica+2015}.  {\bf BDM: a bit more
  detail about the numerical procedure}

For the jet angular structure, we adopt the preferred two-component
model for SwJ1644 from \citet{Mimica+2015}, corresponding to a fast,
inner core with Lorentz factor $\Gamma = 10$, surrounded by a slower,
$\Gamma=2$ outer sheath.  In our 2D simulations the fast inner core
spans an angular interval $0-0.1\ {\rm radians}$, while the slow outer
sheath extends from $0.1\ {\rm radians}$ to $0.5\ {\rm rad}$.  The
time dependence of the jet kinetic luminosity is given by
(\citealt{Mimica+2015})
\begin{equation}\label{eq:lum}
L_{\rm j, ISO}(t) = L_{j,0}\max\left[1, (t/t_0)\right]^{-5/3},
\end{equation}
where $t_0 = 5\times 10^5$ s is the duration of peak jet power, which
is assumed to match that of the period of the most luminous X-ray
emission of SwJ1644.  Integrating equation~(\ref{eq:lum}) from $t = 0$
to $\infty$ gives the isotropic equivalent energy of the jet, $E_{\rm
  ISO}$, where $L_{j,0}=0.4\, E_{\rm ISO}/t_0$.  The ratio of the
beaming-corrected energy of the fast component is fixed to be 4\% of
that of the slow sheath (\citealt{Mimica+2015}).

For our 1D simulations, we modify the geometry of the slow sheath to
better mimick the results of the 2D simulations.  In our 2D models the
sheath is injected within a relatively narrow angular interval;
however, at late stages of evolution the bow shock created by the
jet-CNM interaction spans a much larger angular range due to laterial
spreading.  To account for the slow component becoming almost
isotropic near peak emission in our 2D simulations \citep[bottom two
panels of Fig.~8 in][]{Mimica+2015}, we instead take the slow
component to extend from 0.1 to $\pi/2$ radians in our 1D models.

Figure \ref{fig:1D2DB} compares light curves calculated from this
modified 1D approach to the results of the full 2D simulations.
Despite the slow sheath being initially much broader in the 1D
simulations than in 2D, the resulting light curves agree surprisingly
well.  The agreement is particularly good at the highest densities
($n_{18}=2000$ cm$^{-3}$) because the slow component rapidly
isotropizes in 2D.  component rapidly spreads out and isotropizes in
2D.  At lower densities ($n_{18}=60$ cm$^{-3}$), the agreement is not
at good, with the 1D simulations predicting systematically higher
luminosities by a factor of $\lesssim 2$.

%Nonetheless, using a
%broadened slow component in 1D leads to better agreement with 2D
%results {\bf AG: Is this actually true for n18=60?}.

% Fig~\ref{fig:components}, which shows the relative contributions of
% the fast and slow components to the 5 GHz light curve of a $5 \times
% 10^{53}$ erg jet for different ambient gas densities. For
% $n_{18}=2000$ cm$^{-3}$, the slow component dominates for nearly all
% times.  For $n_{18}=2$ cm$^{-3}$, the fast component dominates at
% early times and at the peak of the light curve, while the slow
% component dominates after $\sim$1 year. In general, the fast component
% is more important for early times, higher frequencies and smaller
% ambient gas densities.

\begin{figure}
\includegraphics[width=8cm]{1d_2d.pdf}
\caption{\label{fig:1D2DB} Comparison of light curves from 1D and 2D
  simulations for an observer angle of 0.8 radians to the jet axis. We
  assume that the gas density $n\propto r^{-1}$ for $n_{18}=2000$
  cm$^{-3}$, but take $n\propto r^{-1.5}$ for $n_{18}=60$ cm$^{-3}$
  for computational convenience--this model was previously computed in
  \citet{Mimica+2015} and our 1D results suggest that the light curve
  is insensitive to reasonable variations in the density power-law
  slope for a fixed value of $n_{18}$ ($\S$~\ref{sec:profileComp})}.
\end{figure}

%\subsection{Light curves}
%\label{sec:lightcurves}
%In this section, we investigate the effects of the gas density profile, jet energy, and observer viewing angle on the observed synchrotron radio light curve. 

\subsection{Analytic Estimates}
\label{sec:analyt}
In this section, we investigate the effects of the gas density
profile, jet energy, and observer viewing angle on the observed
synchrotron radio light curve.  We begin with a few analytic
estimates.

The dependence of the synchrotron peak luminosity, peak time, and late
time luminosity power law slope on the ambient gas density and jet
parameters can be estimated analytically using a simple model for the
emission from a homogenous, shocked slab of gas behind a
self-similarly expanding blast wave (e.g., \citealt{Sari+98,
  Granot+02}).  The relevant results, as presented by
\citet{Leventis+2012}, are summarized in Appendix~\ref{app:analyt}.
The peak luminosity of the slow component of the jet can be estimated
from equation~(\ref{eq:peakLumGen}),
\begin{align}
\nu L_{\nu, p}&=\text{min}
\begin{cases}
  2.7\times 10^{40} \left(\frac{E}{10^{54} {\rm ergs}}\right)^{0.59}
  \left(\frac{\epsilon_e}{0.1}\right)^{1.3}\times \\
  \left(\frac{\epsilon_b}{0.002}\right)^{0.825}\left(\frac{\nu_{\rm
        obs}}{5 {\rm GHz}}\right)^{0.35} n_{18}^{1.24}
  {\rm erg \, s^{-1}} & {\rm Optically\, thin}\\\\
  1.1 \times 10^{42} \left(\frac{E}{10^{54} {\rm ergs}}\right)^{0.87}
  \left(\frac{\epsilon_e}{0.1}\right)^{0.61}\times\\
  \left(\frac{\epsilon_b}{0.002}\right)^{0.26}\left(\frac{\nu_{\rm
        obs}}{5 {\rm GHz}}\right)^{2.01} n_{18}^{-0.14} {\rm erg\,
    s^{-1}} & {\rm Optically\, thick},
\end{cases}
\label{eq:peakLum}
\end{align}
where we have adopted fiducial values for the power-law slope of the
gas density profile, $k=1$, and the electron energy distribution,
$p=2.3$.  The top and bottom lines apply, respectively, to the shocked
CNM being optically thin and optically thick at the deceleration time.
The peak luminosity in the optically thin case depends sensitively on
$n_{18}$, while in the optically thick regime the dependence on
density is much weaker.  We have normalized the peak fluxes in
equation (\ref{eq:peakLum}) to match those derived from our numerical
results. From equation~\eqref{eq:tslope}, we expect the luminosity to
go a $L_\nu \propto t^{-1.5}$ at late times.

The time of maximum flux, for the same fiducial values ($k = 1$,
$p=2.3$), is given by equation (\ref{eq:tpeakGen}),
\begin{align}
t_p= \text{min}
\begin{cases}
  500 E_{54}^{0.5} n_{18}^{-0.5} {\rm days} & {\rm Optically\, Thin}\\\\
  49.9 \left(\frac{E}{10^{54} {\rm ergs}}\right)^{0.32}
  \left(\frac{\epsilon_e}{0.1}\right)^{0.45}
  \left(\frac{\epsilon_b}{0.002}\right)^{0.37}\\
  \left(\frac{\nu_{\rm obs}}{5 {\rm GHz}}\right)^{-1.1} n_{18}^{0.4}
  {\rm days}& {\rm Optically\, Thick},
\end{cases}
\label{eq:peakTime}
\end{align}
where again the normalizations are chosen to match our numerical
results. In general, a more energetic jet produces emission which
peaks later in time.  However, the scaling of $t_p$ with $n_{18}$ is
more complicated: if the emitting region is optically thick at the
deceleration time, the peak time increases with CNM density. In this
case the peak occurs when the self-absorption frequency passes through
the observing band, and occurs later if the nuclear gas density is
higher. Otherwise, the peak time occurs near the deceleration time,
which is a decreasing function of $n_{18}$ (eq.~\ref{eq:tdec}).
Fig.~\ref{fig:diss} shows the rough division line optically thick and
optically thin regimes at 1 and 30 GHz.

\begin{figure}
\includegraphics[width=8cm]{diss.pdf}
\caption{\label{fig:diss} Black lines fraction of the slow component
  ($\Gamma=2$) kinetic energy dissipated by the reverse shock in
  parameter space jet energy and $n_{18}$. Note that for late stages
  of evolution the jet becomes nearly isotropic so that $E_{\rm
    j}\approx E_{\rm j, iso}$. Red squares indicate the on-axis jet
  simulations we have performed. Blue lines show where the slow
  component of the jet optically thin/thick at the deceleration time.}
\end{figure}


\subsubsection{Numerical Light Curves}
\label{sec:numResults}
As summarized in Table~\ref{tab:jetParams} (and shown in
Fig~\ref{fig:diss}), we calculate light curves for a grid of on-axis
jet simulations for five different values of $n_{18}$ (2, 11, 60, 345,
and 2000 cm$^{-3}$) and three different values of the
(beaming-corrected) jet energy $E$ ($5\times 10^{51}$, $5\times
10^{52}$, $5\times 10^{53}$ erg).

The left-hand panel of Fig.~\ref{fig:lightcurves} shows example light
curves for different energy jets and nuclear gas densities. The peak
luminosity is close to linear in jet energy and is virtually
independent of the ambient density.  This is as expected for high
densities and low frequencies: the emission is dominated by the slow
component, which is optically thick at the deceleration time.  For
high frequencies and small CNM densities, the peak luminosity of the
slow component falls off (see right-hand panel of
Fig.~\ref{fig:lightcurves}). Coincidentally, the fast component just
compensates for this decline in the slow component luminosity,
resulting in the total peak luminosity being nearly independent of
$n_{18}$ across the parameter space. In fact, the optically thick peak
luminosity in equation~\ref{eq:peakLum} provides a good approximation
for the total peak luminosity for a fixed value of $n_{18}=2000$
cm$^{-3}$.

Fig.~\ref{fig:lightcurves} also shows that the peak time is a
decreasing function of the ambient gas density. The peak occurs after
the deceleration time when the emitting region transitions from
optically thick to optically thin, and this transition occurs later
for larger ambient gas densities. However, at high frequencies and low
densities the slow component is optically thin at the deceleration
time, and thus its peak time is an increasing function of nuclear gas
density. In particular, at 30 GHz, the slow component light curves
peaks later for $n_{18}$=2 cm$^{-3}$ than for $n_{18}$=60 cm$^{-3}$.


\begin{figure*} 
  \includegraphics[width=8cm]{lightcurves.pdf}
  \includegraphics[width=8cm]{lightcurves_comp.pdf}
  \caption{\label{fig:lightcurves} \textit{Left-hand panel:} On-axis
    ($\theta_{\rm obs}=0$) radio light curves for jet energies of
    $5\times 10^{53}$ erg ({\it opaque}) and $5\times 10^{51}$ erg
    ({\it translucent}) and three different values of n$_{18}$: 2, 60,
    and 2000 cm$^{-3}$.  Solid lines correspond to the light curves
    from 1D jet simulations. When available ($n_{18}=60$ cm$^{-3}$ and
    $n_{18}=2000$ cm$^{-3}$), we have plotted light curves from the 2D
    jet simulations as dashed lines. Note we use $n\propto r^{-1}$ for
    $n_{18}=$ 2 and 2000, but $r^{-1.5}$ for $n_{18}=60$ cm$^{-3}$, as
    this model had already been computed and since the 1D results
    suggest that the density slope will have minimal impact on the
    results (see $\S$~\ref{sec:profileComp}).  Radio upper limits and
    detections are shown as triangles and squares respectively. The
    single upper limit in the top panel is for D3-13 at 1.4 GHz (from
    \citet{Bower+2011}). Gray triangles and squares in the middle
    panel indicate upper limits and detections at 3.0 GHz (from
    \citealt{Bower+2013}), while black triangles indicate upper limits
    at 5.0 GHz (from \citealt{van-Velzen+2013}).  This data is
    summarized Table 1 of~\citealt{Mimica+2015}. The single red
    triangle corresponds a upper limit at 6.1 GHz for the PTF-09axc
    flare from \citet{Arcavi+2014} \textit{Right-hand panel:} Total
    ({\it opaque}) and slow component ({\it translucent}) on-axis
    light curves for a $5\times 10^{53}$ erg jet.}
\end{figure*}

The numerical light curves can be well fit with a broken power law of
the form:

\begin{equation}
L_\nu (t) = L_{\nu, p} \frac{1}{2^{-1/s}}
\left[\left(\frac{t}{t_p}\right)^{-s
    a_1}+\left(\frac{t}{t_p}\right)^{-s a_1}\right]^{-1/s}, 
\label{eq:lcAnal}
\end{equation}

where $L_{\nu, p}$ and $t_p$ is the peak luminosity and time given are
the peak luminosity and time given by equations~\eqref{eq:peakLum}
and~\eqref{eq:peakTime}, respectively. The light curve has a power law
slope of $a_1$ at early times and transitions to power law slope $a_2$
at late times. The sharpness of this transition is encapsulated in
$s$. We find that $a_1\sim 1.9$, $a_2\sim -1.4$, and $s\sim 1.0$, best
reproduce our numerical results. Note the late time power law slope is
close to the analytic estimate in equation~\eqref{eq:tslope}.
%%More precisely 1.94, -1.45, and 1.04 


% \subsubsection{Effects of viewing angle}
Fig.~\ref{fig:onOff} compares the light curves for observers aligned
with the jet axis (on-axis) with those at an angle of 0.8 radians from
the jet axis (off-axis).  While the $n_{18}=2000$ cm$^{-3}$ light
curves differ little, for $n_{18}=60$ cm$^{-3}$ the off-axis
luminosity is smaller by a factor of a few before and near the
peak. However, at late times the off- and on-axis light curves agree
well. This is as expected: at late times the jet becomes more
isotropic, which means the viewing angle has relatively little effect
on the observed light curve {\bf AG note that the slow component is
  relatively wide to begin with in the 2D simulation. Very off-axis
  jets will be ~an order of magnitude dimmer for early times}.

\begin{figure}
\includegraphics[width=8cm]{on_off.pdf}
\caption{\label{fig:onOff} Comparison of (2D) on-axis and off-axis
  light curves. For the off-axis light curve the observer is an angle
  $\theta_{\rm obs}$=0.8 radians away from the jet axis. We assume an
  $n\propto r^{-1}$ density profile for $n_{18}=2000$ cm$^{-3}$, but
  take $n\propto r^{-1.5}$ for $n_{18}=60$ cm$^{-3}$ for computational
  convenience--this model was previously computed in
  \citet{Mimica+2015} and 1D results suggest that the density slope
  would have minimal impact on the results (see
  $\S$~\ref{sec:profileComp}).}
\end{figure}

%\subsubsection{Effect of gas density profile}
\label{sec:profileComp}
The top panel of Fig.~\ref{fig:cores} shows on-axis radio light curves
for our fiducial gas density profile, $n\propto r^{-1}$, and a core
galaxy profile (equation~\ref{eq:cores}), both with $n_{18}=2$
cm$^{-3}$.  The light curves differ by at most a factor of a few. The
core and cusp light curves are even closer at higher densities, and
virtually indistinguishable at $n_{18}=2000$ cm$^{-3}$. This is
because for larger ambient densities, the jet only samples small
radii, where the core and cusp profiles are similar (see
Fig.~\ref{fig:profiles}). It is only at lower densities, for which the
Sedov radius lies outside of the flattening of the core density
profile, that noticeable differences emerge. {\bf AG why are there
  differences at early times for $n_{18}=2$ cm$^{-3}$ but not
  $n_{18}=2000$ cm$^{-3}$. Is this simply an artifact of the starting
  radius?}

The bottom panel of Fig.~\ref{fig:cores} compares the 1D on-axis light
curves for $n\propto r^{-1}$ and $n\propto r^{-1.5}$ gas density
profiles with $n_{18}=60$. For most times the light curves are very
close.  This is as expected, since the Sedov radius, $r_{\rm
  sedov}\sim 10^{18}$ cm. Overall we conclude that the radio light
curve is insensitive to the precise gas density profile, for a fixed
value of the gas density near the deceleration/Sedov radius.

\begin{figure} 
  \includegraphics[width=8cm]{fig_cores.pdf}
  \includegraphics[width=8cm]{profs2.pdf}
  \caption{\label{fig:cores} {\it Top Panel:} Comparison of (on-axis)
    light curves for our fiducial $n\propto r^{-1}$ gas density
    profile and the core galaxy profile from \eqref{eq:cores} with
    $r_s=10^{18}$ cm. {\it Bottom Panel:} Comparison of (on-axis)
    light curves for $n\propto r^{-1}$ ({\it solid black}) and
    $n\propto r^{-1.5}$ ({\it dashed red}) gas density profiles.}
\end{figure}


\subsubsection{Reverse Shock Emission?}
Our calculations shown in Figs.~\ref{fig:1D2DB}$-$\ref{fig:cores}
include only emission from the forward shock (shocked CNM).  In
principal the reverse shock (shocked jet) may also contribute to the
radio light curve.
% However, as we describe in this section its contribution will be
% negligible, at least for high energy jets.

The fraction of the initial kinetic energy of the jet which is
dissipated by the reverse shock provides a first-order estimate of its
maximum contribution to the radio light curve.  Fig.~\ref{fig:diss}
shows the fraction of the slow component ($\Gamma=2$) kinetic energy
dissipated by the reverse shock as a function of isotropic jet
luminosity and CNM denisty $n_{18}$, as estimated by integrating the
shock evolution determined from the jump conditions (see
Appendix~\ref{sec:reverse} for details).  For this estimate, we
approximate the jet as a constant source of duration $t_0 = 5 \times
10^{5}$ s.  The parameters of our numerical solutions are shown as red
squares in Fig.~\ref{fig:diss}.


Fig.~\ref{fig:diss} shows that for high ambient densities and/or low
energy jets, the reverse shock dissipates an order unity fraction of
the kinetic energy of the jet.  However, this is somewhat deceptive
because for high energy jets the reverse shock emission will be
strongly attenuated by self absorption for observer frequencies below
10 GHz.  This is illustrated in Fig.~\ref{fig:reverse}, which shows
the 5 GHz light curve separated into contributions from the forward
and reverse shocks, respectively, calculated for $E = 5\times 10^{53}$
erg and $n_{18} = 2$.  Although the reverse shock contribution is
comparable to that of the forward shock at early times, it contributes
subdominantly to the total as compared to the forward shock due to
self-absorption near the front of the jet (which is not included in
the reverse shock light curve in Fig.~\ref{fig:reverse}).  While the
reverse shock dissipates an even larger fraction of the jet energy for
higher ambient density, its emission will be even more heavily
absorbed.  We conclude that reverse shock emission can be neglected
for the high energy jets with $E\gtrsim 10^{53}$ erg.

For low energy jets, we find that jet is crushed at early times, even
for low values of $n_{18}$, making the reverse shock structure more
challenging analyze in our simulations.  We defer a more detailed
study of the reverse shock in this case to future work.  {\bf BDM:
  Pere should add more here describing the issues with low energy
  jets}

As a final note, even if the reverse shock dissipates most of the
kinetic energy into thermal energy, this energy is still available to
be recoverted to kinetic energy through adiabatic expansion.  However,
the jet re-expansion will be relatively isotropic because the bulk
Lorentz factor of the shocked gas is now mildly relativistic.  The net
result will likely be to produce two quasi-spherical `mushroom clouds' on
either side of the black hole (\citealt{Giannios&Metzger2011}).


\begin{figure}
  \includegraphics[width=8cm]{reverse.pdf}
  \caption{\label{fig:reverse} Reverse shock, forward shock, and total
    light curves for a $5\times 10^{53}$ erg jet propagating through
    our fiducial $n\propto r^{-1}$ medium with density normalization,
    $n_{18}=2$ cm$^{-3}$. The reverse shock emission is absorbed, so
    that effectively only the forward shock contributes to the total
    light curve.}
\end{figure}

\subsection{Parameter Space of Jet-CNM Interaction}
\label{sec:param}
The left side of Fig.~\ref{fig:jetContours} shows contours of the peak
luminosity from our numerical models covering the parameter space of
jet energy $E$ and density $n_{18}$ (thick lines), as well as the
luminosity originating exclusively from the slow component (thin
lines).  The slow component dominates the total luminosity for large
$n_{18}$ and small frequencies, while the fast component dominates for
high frequencies and low ambient gas densities.  Remarkably, the total
peak luminosity is nearly independent of the ambient gas density; this
is entirely coincidental, as the fast and slow peak fluxes
individually vary across the parameter space.
% {\bf AG Not satisfying. Seems very coincidental and model
% dependent. Also the
% fast component seems to have a very strong density dependence.}

The right panel of Fig.~\ref{fig:jetContours} compares the numerical
results for the slow component with our analytic estimate
(eq.~\eqref{eq:peakLum}).  For large $n_{18}$, the peak luminosity
scales with density, jet energy, and frequency roughly as expected in
the optically thick limit.  By contrast, for 30 GHz and low ambient
gas densities, the numerical results asymptotically approach optically
thin limit.

The left panel of Fig.~\ref{fig:ContoursTp} shows contours of the peak
time in days, separately for the slow component (solid lines) and the
total light curve (dashed lines).  Shown for comparison on the right
side is the optically thick peak time estimate from
equation~\eqref{eq:peakTime}, which roughly reproduces the numerical
results for high densities/low frequencies. For low densities, the
numerical results diverge from the analytic prediction, in part due to
the growing contribution of the fast component.  Furthermore, the peak
time for the 30 GHz light curves decreases with density for small
$n_{18}$, due to the emitting region being optically thin at the
deceleration time.




\begin{figure*}
  \includegraphics[width=16cm]{lp_contours_new.pdf}
  \caption{\label{fig:jetContours} {\it {Left:}} Thick lines show the
    peak radio luminosity in the parameter space of jet energy and
    ambient gas density at $10^{18}$ cm, calculated from the grid of
    on-axis jet simulations in Table~\ref{tab:jetParams}. Thin lines
    show contours of peak luminosity for the slow component light
    curve ($\S$~\ref{sec:numerical}). {\it Right:} Analytic estimate
    for the peak luminosity (dashed lines; eq.~\ref{eq:peakLum})
    compared to the numerical results for the slow component (solid
    lines).}
\end{figure*}

\begin{figure*}
  \includegraphics[width=16cm]{tp_contours_new.pdf}
  \caption{\label{fig:ContoursTp} {\it {Left:}} Thick lines show peak
    time in the parameter space of jet energy and ambient gas density
    at $10^{18}$ cm, calculated from the grid of on-axis jet
    simulations in Table~\ref{tab:jetParams}. Thin lines show contours
    of peak luminosity for the slow component light curve
    (see~\ref{sec:numerical}). {\it Right:} Analytic scaling for the
    peak luminosity ({\it dashed}, see equation~\ref{eq:peakTime})
    compared to the numerical results for the slow component (solid
    lines)}
\end{figure*}

\subsubsection{Comparison with radio detections and upper limits.}
\label{sec:upLims}
The left panel of Fig.~\ref{fig:lightcurves} show radio upper limits
and detections for various TDE candidates (compiled from various
sources into Table 1 of \citealt{Mimica+2015}) together with on-axis
light curves for our fiducial $5\times 10^{53}$ erg jet (comparable to
SwJ1644) and three different ambient density normalizations: n$_{18}$:
2, 60, and 2000 cm$^{-3}$. All of the 5 GHz light curves fall above
the upper limits, in agreement with the results of previous works
concluding that thermal TDEs do not possess jets as powerful as that
responsible for SwJ1644
(\citealt{Bower+2013},\citealt{van-Velzen+2013},\citealt{Mimica+2015}),
a result which we now find holds to be independent of the CNM
environment.

{\bf AG--Pere could you check the late time slope of the 2D light
  curves again? You said part of the discrepancy is due to
  contributions of the fast component. But from the right panel of
  Fig. 4, it is clear that the slow component dominates the 1D light
  curve for times $\lsim$ 1 yr.}  One caveat is that we were unable to
perform a 2D calculation for $n_{18}=2$ cm$^{-3}$ to late times due to
numerical issues.  Since for $n_{18}=60$ cm$^{-3}$ the 2D light curve
fell roughly one order of magnitude below the 1D light curve after
$\sim$ 10 years, we caution that the 1D light curves for $n_{18}=2$
cm$^{-3}$ likely overestimate the late time radio luminosity.  For
very low densities is therefore still possible that even a
SwJ1644-like jet would fall below existing upper limits in some cases.

Multiple radio measurements from several months to a few years after a
tidal disruption flare would provide better constraints on the
presence of TDE jets. For example, suppose we have a TDE candidate
with a radio upper limit. Equation~\eqref{eq:peakLum}, provides a
constraint on the total energy of the jet,
\begin{equation}
E\lsim 6 \times 10^{48} \left(\frac{S}{50 \,\mu{\rm Jy}}\right)^{1.1}
  \left(\frac{d_L}{200 \,{\rm Mpc}}\right)^{2.3} {\rm erg},
\end{equation}
%
where $S$ is upper limit for the flux density and $d_L$ is the
luminosity distance to the source, and we have set $n_{18}=2000$
cm$^{-3}$.

{\bf This section implicitly assumes that late time slopes of the 1D
  light curves can be trusted.}  The jet energy can still be
constrained from late time measurements. Fig.~\ref{fig:econtours},
shows a comparison between the analytic fit for the synchrotron light
curve (eq.~\ref{eq:lcAnal}) evaluated at different jet energies and
existing radio upper limits at 5 GHz for TDE candidates. The top panel
shows the light curves for a fixed density normalization. Changing the
density normalization simply shifts the light curve in time. Thus, the
radio luminosity at any given time is bounded by the luminosities for
the minimum and maximum plausible $n_{18}$: 0.5 cm$^{-3}$ and 2000
cm$^{-3}$ respectively. The bottom panel of Fig.~\ref{fig:econtours}
shows the curves of the minimum possible synchrotron luminosity for
different jet energies along with radio upper limits. The minimum
synchrotron luminosity curve going through each upper limit
corresponds to the maximum possible jet energy consistent with
it. Note that these constraints would apply to spherical outflows as
well as jets.



\begin{figure}
\includegraphics[width=8.5cm]{e_contours1.pdf}
\includegraphics[width=8.5cm]{e_contours2.pdf}
\caption{\label{fig:econtours} {\it Top:} 5 GHz Upper limits and light
  curves for different jet energies. {\it Bottom:} Upper limits and
  minimum possible synchrotron luminosity as a function of time and
  jet energy. This is minimum of the light curves for the minimum
  plausible $n_{18}$ ($\sim 0.5$ cm$^{-3}$) and the maximum plausible
  $n_{18}$ ($\sim 2000$ cm$^{-3}$).}
\end{figure}

Fig.~\ref{fig:hist} shows a histograms of maximum jet energy
consistent with existing radio upper limits and detections of TDE
candidates. The latter group consists of 4 events: ASSASN-14li,
CSS100217, SwJ1644, and SwJ2058. For ASSASN-14li and SwJ1644 the radio
light curves are well sampled, and the energy of the jet is well
constrained: $10^{48}-10^{49}$ erg (\citealt{van-Velzen+2015,
  Alexander+2015}) for ASSASN-14li and $5\times 10^{53}$ erg. For
CSS100217 there is only a single radio detection, which requires the
outflow energy to be $1.7\times 10^{50}-6\times 10^{51}$ erg. For
SwJ2058 we take the jet energy to be $5\times 10^{53}$ erg. On the one
hand the observed radio luminosity ($\nu L_{\nu }\approx 10^{42}$ erg
s$^{-1}$ \citep{Cenko+2012}) requires an outflow at least this
energetic. On the other hand, this is comparable to the total energy
available from the disruption of a solar type star. {\bf AG double
  check SwJ2058}.

\begin{figure}
\includegraphics[width=8.5cm]{hist.pdf}
\caption{\label{fig:hist} Histogram of maximum plausible jet energies
  consistent with existing radio upper limits and radio detections
  (ASSASN-14li, CSS100217, SwJ1644, and SwJ2058). The radio upper
  limits are taken from the compilation in Table 1 of
  \citet{Mimica+2015}, to which we add the upper limit for the
  PTF-09axc flare from \citet{Arcavi+2014}. Energy constraints for
  each event are summarized in Table~\ref{tab:enConstr}}
\end{figure}

\begin{table*}
\begin{threeparttable}
  \caption{\label{tab:enConstr} Inferred jet/outflow energies (and bounds) from radio detections and upper limits of optical/UV
    and soft x-ray TDE candidates.}
\begin{tabular*}{1.5\columnwidth}{lllllll}
\hline
Source & $D_L$ & t & $\nu$ & $\nu L_{\nu}$ & Ref. & Energy\\
& (Mpc) & (yr) & (GHz) & ($10^{36}$ erg s$^{-1}$) & & (erg) \\
\hline
Detections \\
\hline
ASSASN-14li & & & & & &  $10^{48}-10^{49}$\\
CSS100217 & & & & & & $1.7\times 10^{50}-6\times 10^{51}$\\
SwJ1644 & & & & & & $5\times 10^{53}$\\
SwJ2058 & & & & & & $5\times 10^{53}$\\ 
\hline 
Upper limits & \\
\hline
NGC5905 & 52.0 & 21.91 & 3.0 & 2 & 1 & $< 2.6 \times 10^{ 52 }$ \\
RXJ1624+7554 & 265.0 & 21.67 & 3.0 & 12 & 1 & $< 9.9 \times 10^{ 52 }$ \\
RXJ1242-1119 & 208.0 & 19.89 & 3.0 & 8 & 1 & $< 6.7 \times 10^{ 52 }$ \\
SDSSJ1323+48 & 365.0 & 8.61 & 3.0 & 48 & 1 & $< 1.0 \times 10^{ 53 }$ \\
SDSSJ1311-01 & 750.0 & 8.21 & 3.0 & 115 & 1 & $< 1.9 \times 10^{ 53 }$ \\
D1-9 & 1700.0 & 8.0 & 5.0 & 800 & 2 & $< 6.6 \times 10^{ 53 }$ \\
D3-13 & 2000.0 & 7.6 & 5.0 & 960 & 2 & $< 7.2 \times 10^{ 53 }$ \\
TDE1 & 645.0 & 5.4 & 5.0 & 120 & 2 & $< 1.0 \times 10^{ 53 }$ \\
D23H-1 & 910.0 & 4.8 & 5.0 & 200 & 2 & $< 1.3 \times 10^{ 53 }$ \\
TDE2 & 1280.0 & 4.3 & 5.0 & 590 & 2 & $< 2.7 \times 10^{ 53 }$ \\
PTF10iya & 1130.0 & 1.6 & 5.0 & 300 & 2 & $< 5.5 \times 10^{ 52 }$ \\
PS1-10jh & 820.0 & 0.71 & 5.0 & 300 & 2 & $< 2.3 \times 10^{ 52 }$ \\
NGC5905 & 75.0 & 6.0 & 8.6 & 9 & 3 & $< 1.4 \times 10^{ 52 }$ \\
D3-13 & 2000.0 & 1.8 & 1.4 & 1000 & 4 & $< 2.4 \times 10^{ 53 }$ \\
TDE2 & 1280.0 & 1.1 & 8.4 & 1650 & 5 & $< 1.1 \times 10^{ 53 }$ \\
SDSSJ1201+30 & 700.0 & 1.4 & 7.9 & 1000 & 7 & $< 1.0 \times 10^{ 53 }$ \\
PTF09axc & 540.0 & 5.0 & 3.5 & 670 & 8 & $< 2.8 \times 10^{ 53 }$ \\
PTF09axc & 540.0 & 5.0 & 6.1 & 530 & 8 & $< 1.9 \times 10^{ 53 }$ \\
\end{tabular*}
\begin{tablenotes}
\item References: $(1)$ \citet{Bower+2013}, $(2)$ \citet{van-Velzen+2013},
$(3)$ \citet{Bade+1996, Komossa2002},
$(4)$ \citet{Gezari+2008,Bower+2011}, $(5)$ \citet{van-Velzen+2011},
$(6)$ \citet{Drake+2011}, $(7)$ \citet{Saxton+2012}, $(8)$
\citet{Arcavi+2014}. All upper limits are 5 $\sigma$.
\end{tablenotes}
\end{threeparttable}
\end{table*}

\section{Summary and Conclusions}
\label{sec:conc}

We calculate radio light curves for tidal disruption event jets
propagating through different circumnuclear (CNM) gas densities. We
simulate the jet propagation using both 1D and 2D hydrodynamic
simulations. We then post-process these to produce radio synchrotron
light curves. To isolate the effects of the density profile we take a
fixed two component jet model from \citet{Mimica+2015}, which produces
a good fit to the observed radio data in SwJ1644. We
consider a broad range of gas densities motivated by analytic
estimates of stellar wind mass injection and empirical constraints
based on observed distributions of Eddington ratios. Our conclusions
are summarized as follows.

\begin{enumerate}
\item We estimate the nuclear gas densities expected from injection of
  stellar wind material for different star formation histories. We
  find that that range of gas densities at 10$^{18}$ cm is $n_{18}
  \sim$ 0.5-2000 cm$^{-3}$.

\item The slope of the gas density profile depends on the slope of the
  stellar density profile. We expect a typical TDE host to have cuspy
  stellar density profile inside of a few pc, with $\rho_\star
  \propto r^{-1.7}$. This translates into a gas density profile $n
  \propto r^{-1}$. The radio light curve of a TDE jet is most
  sensitive to the density at the deceleration/Sedov radius (where it
  has swept up its mass in CNM gas). The light curve will be
  insensitive to changes in slope for fixed density at the
  deceleration radius.

\item We take a jet model which fits the radio data for the SwJ1644
  transient (from \citealt{Mimica+2015}) and run it through a range of
  different density profiles. Motivated by the above results for the
  expected range of gas densities we take the density at $10^{18}$ cm,
  to be $n_{18}$ 2, 11, 60, 345, or 2000 cm$^{-3}$. We find bright radio
  emission at a few GHz across this entire range of densities, with
  the peak luminosity only weakly dependent on the chosen value of
  $n_{18}$.  For smaller densities the light curves peak earlier in
  time. By comparing radio detections and upper limits from a set
  optical/UV and soft x-ray selected TDE, we show that these sources
  cannot have jets as powerful as SwJ1644. Prompt follow-up in the
  radio could provide tighter constraints on the existence of TDE
  jets.
\end{enumerate}

\appendix
\section{Core Profile}
\label{app:core}
Fig.~\ref{fig:cores} compares the results of radio light curves from jets propagating in core and cusp like gas density
profiles (Fig.~\ref{fig:profiles}).  We use the following analytic expression to approximate the core
galaxy profile
\begin{align}
\begin{cases}
n=n(r_s) k(x) & 0.4 \leq x\leq 2.0\\
n = 2.0 n(r_s) (x/0.4)^{-0.95} & x < 0.4\\
n = 0.75 n(r_s) (x/2.0)^{-0.26} & x>2,\\
\end{cases}
\label{eq:cores}
\end{align}
where
\begin{align}
  &x=r/r_s\\\nonumber
  &k(x)=\frac{45}{19} \frac{1}{x^{3/2}} \frac{1-x^{1.9}}{9-19
      x\frac{x^{0.9}-1}{x^{1.9}-1}}
\end{align}
To isolate the effects of the shape of the density profile, we consider a
core density profile with a stagnation radius $r_s=10^{18}$ cm and density normalization $n_{18}=2000$
cm$^{-3}$ which match those of our high density cusp model.

\section{Peak Luminosities and times}
\label{app:analyt}
\citet{Leventis+2012} present analytic scaling relations for the
synchrotron flux of a spherical blast wave propagating through a
medium with a power law density profile, $n\propto r^{-k}$.  Here we
make use of their results to estimate the peak radio flux of the slow
(sheath) component of the jet.

During the late-time, Newtonian stage of the jet evolution,
synchrotron self absorption is important for frequencies below
\begin{align}
  \nu_{\rm sa}=&C_1(p, k) E_{54}^{\frac{10 p-k p -6 k}{2 (4+p) (5-k)}} n_{18}^{\frac{30 - 5 p}{2 (4 + p) (5 - k)}}
  \epsilon_e^{\frac{2 (p-1)}{4+p}} \epsilon_b^{\frac{p+2}{2 (4+p)}}\nonumber\\
  &t^{\frac{10 - 8 k - 15 p + 4 k p}{(4 + p) (5 - k)}}, 
\label{eq:nuSa} 
\end{align}
%
where $E = 10^{54}E_{54}$ erg is the blast wave energy and $C_1(p, k)$
is a normalization factor.  Equation~\eqref{eq:nuSa} is valid only if
self-absorption frequency is greater than the synchrotron peak
frequency,
\begin{equation}
\nu_m=C_2(p, k) E_{54}^{\frac{10-k}{2 (5-k)}} n_{18}^{-\frac{5}{2
    (5-k)}}  \epsilon_e^2  \epsilon_b^{1/2}  t^{\frac{4 k-15}{5-k}}.
\label{eq:num}
\end{equation}
%
% The observing frequencies considered here ($\geq 1$ GHz), are also
% greater than $\nu_m$.
The light curve will peak at the deceleration time (eq.~\ref{eq:tdec})
in case the emitting region is optically thin then. Otherwise, it will
occur after the deceleration time, when the self-absorption frequency
crosses through the observing band. The peak time for these two cases
is

\begin{align}
t_{\rm p} \approx
\begin{cases}
  \frac{t_{\rm dec}}{2 \Gamma^2} \approx \left(100 \,
    E_{54}\right)^{1/(3-k)}  \Gamma^{(2k - 8)/(3-k)}
  n_{18}^{-1/(3-k)} {\rm yr} & {\rm Optically thin}\\\\
  C_1(p, k)^{-\frac{(5 - k) (4 + p)}{10 - 8 k - 15 p + 4 k
      p}}E_{54}^{-\frac{-k p-6 k+10 p}{2 (4 k p-8 k-15 p+10)}}\\
  \times n_{18}^{-\frac{30-5 p}{2 (4 k p-8 k-15 p+10)}} \nu_{\rm
    obs}^{\frac{(5-k) (p+4)}{4 k p-8 k-15 p+10}}\\
  \times \epsilon_b^{-\frac{(5-k) (p+2)}{2 (4 k p-8 k-15 p+10)}}
  \epsilon_e^{-\frac{2 (5-k) (p-1)}{4 k p-8 k-15 p+10}} & {\rm
    Optically thick},
\end{cases}
\label{eq:tpeakGen}
\end{align}
%
where $\Gamma$ is the initial jet Lorentz factor. 
%The additional
%factor of $1/2 \Gamma^2$ for Case 1 is a correction accounting for the
%fact that the observed light curve will be compressed in time due to
%light travel time effects.

The unabsorbed flux at the peak frequency is given by
\begin{align}
  F_{\nu_m} =  C_3(p, k) E_{54}^{\frac{8-3 k}{2 (5-k)}}
  n_{18}^{\frac{7}{2 (5-k)}} \epsilon_b^{1/2} t^{\frac{3-2 k}{5-k}}
\label{eq:Fnum}
\end{align}
%
Extrapolating to the observer frequency gives 
\begin{align}
  \nu_{\rm obs} F_{\rm p} (\nu_{\rm obs}) &= \nu_{\rm obs}   F_{\nu_m}
  \left(\frac{\nu_{\rm obs}}{\nu_m}\right)^{-(p-1)/2}.
  \label{eq:Fpeak1}
\end{align}
%
Combining equations~\eqref{eq:num}, ~\eqref{eq:tpeakGen}, ~\eqref{eq:Fnum},
and~\eqref{eq:Fpeak1}, we find
\begin{align}
  \nu_{\rm obs} F_{\rm p} (\nu_{\rm obs}) \propto
  \begin{cases}
    E_{54}^{\frac{k (p+5)-12}{4 (k-3)}} n_{18}^{-\frac{3 (p+1)}{4
        (k-3)}} \nu_{\rm obs}^{\frac{3-p}{2}}
    \epsilon_b^{\frac{p+1}{4}} \epsilon_e^{p-1} & {\rm Optically\, thin}\\\\
    E_{54}^{\frac{k(-(p-2))-10 p+3}{4 k (p-2)-15 p+10}} \\ \times
    n_{18}^{\frac{11 (p-2)}{4 k (p-2)-15 p+10}} \nu_{\rm
      obs}^{\frac{14 k (p-2)-47 p+57}{4 k (p-2)-15 p+10}} \\ \times
    \epsilon_b^{\frac{k (-(p-2))+p-8}{4 k(p-2)-15 p+10}}
    \epsilon_e^{-\frac{11 (p-1)}{4 k (p-2)-15p+10}} & {\rm Optically\,
    thick}
  \end{cases}
  \label{eq:peakLumGen}
\end{align}

After peak, we expect that the flux scales as 

\begin{equation}
F_{\nu}\propto t^{\frac{21-8k-15p+4kp}{10-2k}}
\label{eq:tslope}
\end{equation}



\section{Reverse shock}
\label{sec:reverse}
Here we estimate the fraction of the kinetic energy of the jet that is
dissipated by the reverse shock, as opposed to the forward shock whose
contribution is the focus of this paper.  From continuity, the
comoving density of a relativistic jet is given by
(e.g.~\citealt{Beloborodov&Uhm2006})
 \begin{align}
   n_{\rm j} =  \frac{L_{\rm j, iso}}{4 \pi r^{2}\Gamma_{\rm
       j}^{2}c^{3}m_p(1 + r \dot{\Gamma}/c\Gamma^{3})}
   \approx  \frac{L_{\rm j, iso}}{4 \pi r^{2}\Gamma^{2}c^{3}m_p},
\end{align}
%
where $L_{\rm j, iso}$ is the isotropic equivalent luminosity.  The
second term in the denominator can be neglected if the jet Lorentz
factor changes slowly ($\dot{\Gamma}_{\rm j} \ll c\Gamma^{3}/r$), a
condition which is satisfied at radii $r < r_{\rm dec}$ if $\Gamma$
changes slowly on a timescale $\gtrsim t_{\rm 0}$, where $t_{\rm 0}$
is the jet duration.

The common Lorentz factor of the shocked CNM and the shocked jet can
be estimated using the relativistic shock jump condition and pressure
equality between the forward and reverse shocks.  In the
ultra-relativistic limit this gives,
\begin{equation}
\Gamma_{\rm sh} \underset{\Gamma_{\rm sh} \gg 1}= \Gamma\left[1 + 2\Gamma f^{-1/2}\right]^{-1/2},
\end{equation}
where
\begin{equation}
  f\approx 40\,  L_{\rm j,48} n_{18}^{-1} \Gamma_{10}^{-2} \, \left(\frac{r}{10^{18} {\rm
        cm}}\right)^{-1} 
\end{equation}
is the ratio of the density of the jet to that of the CNM.  This
expression is inaccurate for mildly relativistic or non-relativistic,
in which case we apply the more general expression for $\Gamma_{\rm
  sh}$ given by \citet{Beloborodov&Uhm2006} (their Fig.~6)
\begin{equation}
\frac{\Gamma^2-1}{\Gamma_{43}^2-1} f^{-1}=1 ,
\end{equation}
where
\begin{equation}
  \Gamma_{43}=\Gamma \Gamma_{\rm sh} \left(1-\beta_{\rm sh} \beta_j\right),
\label{eq:gammaShGen}
\end{equation}
is the Lorentz of shocked jet in the frame of the unshocked jet. In
the lab frame the reverse shock moves with a velocity
\begin{equation}
\beta_{\rm rs}=\frac{\beta_{\rm sh}(f)-\beta_{43}(f)/3}{1-\beta_{\rm
    sh}(f) \beta_{43}(f)/3}.
\label{eq:betars}
\end{equation} 
%
Equations~\eqref{eq:gammaShGen} and ~\eqref{eq:betars} can be used to
determine the radius of the shocks when the reverse shock crosses the
trailing edge of the jet and the value of $\Gamma_{\rm sh}$ at this
time.  The latter allows us to calculate what fraction of the initial
kinetic energy of the jet is dissipated at the reverse shock, instead
of being transferred to the shocked external medium via the forward
shock.



\clearpage
  \footnotesize{
    \bibliographystyle{mnras}
    \bibliography{master}
  }

\end{document}
